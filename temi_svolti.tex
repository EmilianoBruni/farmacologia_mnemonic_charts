\part{Temi svolti}

\section{$\beta$--bloccanti}

I farmaci $\beta$--bloccanti agiscono sul recettore adrenergico~$\beta$, recettore metabotropo a proteina~G di tipo~\ce{G_s} prevalentemente stimolatorio presente nel cuore, adipociti, apparato iuxaglomerulare~($\beta_1$), nel muscolo liscio~($\beta_2$) e nella vescica~($\beta_3$).

Il recettore attiva la cascata di segnalazione intracellulare tramite aumento di cAMP.

Gli effetti dei farmaci $\beta$--bloccanti agiscono nell'aparato cardiocircolatorio sia inibendo il sistema renica---angiotensina con riduzione del tono arteriolare con conseguente diminuzione della pressione e del post--carico, sia come effetto diretto inotropo e cronotropo negativo sul muscolo cardiaco.

Il \index{propanololo}propanololo agisce su titti i recettori di questa famiglia ed è usato principalmente nell'angina, nell'infarto e nell'insufficienza cardiaca per ridurre le richieste metaboliche del miocardio, nelle aritmie, come farmaco di classe~II per ridurre il potenziale d'aziene e aumentare il periodo refrattario~AV.

Il \index{metoprololo}metoprololo viene usato nelle emicranie e nel tremore muscolare.

Il \index{labetololo}labetololo (anche $\alpha$--bloccante) è usato per bloccare la cascata adrenergica introdotta dal feocromocitoma. Il \index{timololo} viene usato per la gestione del glaucoma.

I $\beta$--bloccanti lipofilici tipo il \index{propanololo}propanololo sono assunti per os e ben assorbiti, con intenso metabolismo epatico.

I $\beta$--bloccanti idrofilici tipo l'\index{atenololo}atenololo, non sono ben assorbiti per os. 

Eventuali effetti collaterali sono il blocco della conduzione~AV soprattutto in unione con i~\ce{Ca^2+}--antagonisti, reazioni broncocostrittive e aumento della glicemia.

\section{Farmaci anti--psicotici}

La schizzofrenia è una persistente alienazione del pensiero che da sia disturbi positivi, con caratteristiche psicologiche aggiunte quali delirio, allucinazioni e aggressività, sia sintomi negativi, con caratteristiche psicologiche perse quali isolamento sociale, apatia e mancanza di iniziativa.

Le teorie fisiopatologiche di questo disturbo sono tre. Una teoria dopaminergica che spiega i sintomi positivi che deriva la patologia da una iperstimolazione dei recettori~\ce{D_2}/\ce{D_4}; una teoria glutammatergica che deriva la patologi da bassi livelli di glutammato e da una conseguente iperstimolazione del recettore~NMDAM; una teoria serotoninergica derivata dall'osservazione che gli antagonisti serotoninergici sono antipsicotici e che l'LSD, un agonista~5-HT, fa venire i sintomi positivi.

I farmaci usati si dividono in tipici a atipici. I tipici quali i fenotiazidici come la \index{clorpromazina}clorpromazina e i butirrofenonici come l'\index{aloperidolo}aloperidolo (Serenase), agiscono bloccando i recettori dopaminergici diminuendo i sintomi positivi e, i più recenti, anche quelli negativi ma hanno effetti collaterali sul sistema extrapiramidale come distonie acute e tardive. 

Gli atipici non hanno effetti sulla via extrapiramidale in quando bloccano selettivamente la via mesolimbica (della gratificazione) ignorando la via nigrostriata e sono quindi usati principalmente se gli effetti collaterali dei tipici sono eccessivi. Tali farmaci bloccano anche i recettori $\alpha$--adrenergici e i~5-HT e sono la \index{clorapina}clorapina (una dibenzodiazepina) e l'\index{olanzapina}olanzapina.

Il difetto di tutti i farmaci anti--psicotici descritti è che impiegano settimane prima del loro effetto terapeutico e questo è un segno che vi deve essere un qualche altro effetto secondario ad agire come, ad esempio, l'aumento dei~\ce{D_2} a livello limbico.

\section{Le displidemie o iperlipidemie}

La displidemia indica un elevato livello di lipidi nel sangue. 

I lipidi presenti nel sangue arrivano da una via esogena e da una via endogena.

Dalla via esogena, dal cibo presente nel lume intestinale, i lipidi vengono internalizzati da un recettore chiamato NPC1L1 presente sull'orletto a spazola degli enterociti e qui esterificati e inglobati in chilomiconi che, attraverso il sangue, raggiungono muscolo, tessuto adiposo e fegato.

La via endogena prevede la sintesi nel fegato da parte, tra l'altro, di un enzima, l'HMG-CoA reduttasi che risulta catalizzare la tappa limitante della sintesi dei grassi.

I grassi vengono poi inglobate da lipoproteine a formare micelle classifficate sulla base della densità in HDL-C, LDL-C, VLDL.

Le displidemie possono essere primarie o secondarie per diabete mellito, alcolemia, insufficienza renale cronica o per effetto collaterale da farmaci.

I farmaci che agiscono sulla via endogena sono le statine (\index{simvastina}simvastina) che inibisce la HMG-CoA reduttasi, i fibrati (\index{benzofibrato}benzofibrato) che attivano un gruppo di geni che trascrivono per le lipasi, le apoA1 (quindi HDL) e apoA5 che a sua volta stimola la produzione di lipasi.

I farmaci che inibiscono l'assorbimento di colesterolo sono l'\index{ezetimide}ezetimide che blocca il recettore NPC1L1 e, un po' in disuso, le resine leganti gli acidi biliari che, voluminose e di cattivo gusto, sequestrano gli acidi biliari a livello del lume intestinale evitandone il riassorbimento ma causano diarrea per iperosmolarità del contenuto intestinale.

Da citare che le statine sono anche usate nella prevenzione dell'infarto del miocardio e nella prevenzione di placche aterosclerotiche in pazienti con LDL alto.

\newpage