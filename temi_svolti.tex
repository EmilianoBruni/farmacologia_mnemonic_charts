\part{Esami}
\section{Temi svolti}

\subsection{$\beta$--bloccanti}

I farmaci $\beta$--bloccanti agiscono sul recettore adrenergico~$\beta$, recettore metabotropo a proteina~G di tipo~\ce{G_s} prevalentemente stimolatorio presente nel cuore, adipociti, apparato iuxaglomerulare~($\beta_1$), nel muscolo liscio~($\beta_2$) e nella vescica~($\beta_3$).

Il recettore attiva la cascata di segnalazione intracellulare tramite aumento di cAMP.

Gli effetti dei farmaci $\beta$--bloccanti agiscono nell'aparato cardiocircolatorio sia inibendo il sistema renica---angiotensina con riduzione del tono arteriolare con conseguente diminuzione della pressione e del post--carico, sia come effetto diretto inotropo e cronotropo negativo sul muscolo cardiaco.

Il \index{propanololo}propanololo agisce su titti i recettori di questa famiglia ed è usato principalmente nell'angina, nell'infarto e nell'insufficienza cardiaca per ridurre le richieste metaboliche del miocardio, nelle aritmie, come farmaco di classe~II per ridurre il potenziale d'aziene e aumentare il periodo refrattario~AV.

Il \index{metoprololo}metoprololo viene usato nelle emicranie e nel tremore muscolare.

Il \index{labetalolo}labetalolo (anche $\alpha$--bloccante) è usato per bloccare la cascata adrenergica introdotta dal feocromocitoma. Il \index{timololo} viene usato per la gestione del glaucoma.

I $\beta$--bloccanti lipofilici tipo il \index{propanololo}propanololo sono assunti per os e ben assorbiti, con intenso metabolismo epatico.

I $\beta$--bloccanti idrofilici tipo l'\index{atenololo}atenololo, non sono ben assorbiti per os. 

Eventuali effetti collaterali sono il blocco della conduzione~AV soprattutto in unione con i~\ce{Ca^2+}--antagonisti, reazioni broncocostrittive e aumento della glicemia.

\subsection{Farmaci anti--psicotici}

La schizzofrenia è una persistente alienazione del pensiero che da sia disturbi positivi, con caratteristiche psicologiche aggiunte quali delirio, allucinazioni e aggressività, sia sintomi negativi, con caratteristiche psicologiche perse quali isolamento sociale, apatia e mancanza di iniziativa.

Le teorie fisiopatologiche di questo disturbo sono tre. Una teoria dopaminergica che spiega i sintomi positivi che deriva la patologia da una iperstimolazione dei recettori~\ce{D_2}/\ce{D_4}; una teoria glutammatergica che deriva la patologi da bassi livelli di glutammato e da una conseguente iperstimolazione del recettore~NMDAM; una teoria serotoninergica derivata dall'osservazione che gli antagonisti serotoninergici sono antipsicotici e che l'LSD, un agonista~5-HT, fa venire i sintomi positivi.

I farmaci usati si dividono in tipici a atipici. I tipici quali i fenotiazidici come la \index{clorpromazina}clorpromazina e i butirrofenonici come l'\index{aloperidolo}aloperidolo (Serenase), agiscono bloccando i recettori dopaminergici diminuendo i sintomi positivi e, i più recenti, anche quelli negativi ma hanno effetti collaterali sul sistema extrapiramidale come distonie acute e tardive. 

Gli atipici non hanno effetti sulla via extrapiramidale in quando bloccano selettivamente la via mesolimbica (della gratificazione) ignorando la via nigrostriata e sono quindi usati principalmente se gli effetti collaterali dei tipici sono eccessivi. Tali farmaci bloccano anche i recettori $\alpha$--adrenergici e i~5-HT e sono la \index{clorapina}clorapina (una dibenzodiazepina) e l'\index{olanzapina}olanzapina.

Il difetto di tutti i farmaci anti--psicotici descritti è che impiegano settimane prima del loro effetto terapeutico e questo è un segno che vi deve essere un qualche altro effetto secondario ad agire come, ad esempio, l'aumento dei~\ce{D_2} a livello limbico.

\subsection{Farmaci antianemici}

L'anemia è una riduzione della massa eritrocitaria nel sangue misurabile come riduzione dei valori di Hb. Le cause principali sono carenza di ferro, talassemie e emoglobinopatie con caratteristiche microcitiche (MCV < 80 fl), emolitiche, aplastiche o emorragiche con caratteristiche normocitiche ( 100 < MCV < 80 fl) e deficit di \ce{B_12} o folati con caratteristiche macrocitiche (MCV > 100 fl).

La carenza di ferro si ha nel caso di amoraggie croniche, aumento del fabbisogno o diminuzione nell'assorbimento e si cura con sali ferrosi per os come il solfato ferroso o parenterale com il ferro destrano nel caso di intolleranza alla terapia per os. L'assorbimento segue le stesse vie del ferro alimentare. Le reazioni avverse vanno da disfunzioni intestinali quali diarrea, vomito e nausea fino alla gastrite necrotizzante nell'intossicazione acuta. Per via parenterale vi possono essere anche casi di reazione anafilattica.

Nel caso di sovradosaggi può essere utile l'impiego di chelanti del ferro come la \index{desferrioxamina}desferrioxamina.

Nel caso di anemia megaloblastica, dato che i sintomi del deficit di \ce{B_12} e di folati sono molto simili, va valutato dapprima l'eventuale deficit di \ce{B_12} eliminando dubbi sulla presenza di anemia perniciosa da deficit del fattore primario, malassorbimento primitivo, gravidanza e eliminazione dalla dieta della vitamina \ce{B_12}. Il deficit di tale vitamina, al contrario di quella dei folati, causa anche problemi neurologici che non si risolvono una volta risolto il deficit.

Le reazioni avverse alla vitamina~\ce{B_12} sono la trombosi, policitemia e insufficienza cardiaca fino allo shock anafilattico. Non si registrano grosse reazioni avverse all'uso dei folati.

Nel caso di anemie aplastiche, disordini midollari e insufficienza renale sono utili i fattori di crescita emopoietici come l'\index{eritropoietina}eritropoietina e la \index{darbepoetina $\alpha$}darbepoetina $\alpha$ con reazioni avverse un eventuale aumento dell'ematocrito e aumentata viscosità oltre che al prurito.

\subsection{Farmaci per il trattamento dell'obesità}

L'obesità è una malattia multifattoriale e poligenica in cui l'apporto calorico nel lungo periodo è superiore al consumo energetico causando un aumento del BMI, l'indice di massa corporea.

I principali fattori che entrano nella regolazione del cibo e del consumo energetico sono la leptina, la colecistochina (CCK), l'insulina, il sistema nervoso simpatico e fattori psico--socio--economici.

La leptina è sintetizzata dalle cellule adipose e il suo aumento dovrebbe portare ad un effetto anoressizzante ma nei pazienti obesi tale effetto è mancante per qualche forma di resistenza dovuta a degradazione, a difetto del trasportatore o inefficacia dei recettori.

La sintesi della leptina è regolata positivamente da glucocorticoidi, insulina. Una regolazione negativa è data da agonisti $\beta$--adrenergici.

La CCK agisce sul rilascio di bile, stimolando la secrezione di insulina e attiva la stimolazione vagale portando un effetto di sazietà.

L'insulina stimola la leptina ma nell'obeso, essendo insensibile ciò causa ipertensione.

Il sistema nervoso simpatico invece causa un effetto termogenico grazie alla fosforilazione ossidativa disaccoppiata nelle cellule brune e relativo aumento de consumi energetici.

L'obesità causa patologie secondarie quali il diabete mellito la cui terapia, l'insulina, causa un ulteriore aumento di assunzione di cibo, malattie cardiovascolari, tumori ormoni dipendenti, probelmi digestivi e respiratori e osteoartriti.

I farmaci usati sono la \index{sibutramina}sibutramina che inibisce la ricaptazione della serotonina e noradrenalina (IRSN) agendo negativamente sui siti regolanti l'appetito con aumento della sazietà, diminuzione del BMI, diminuzione di LDL e aumento di HDL. La sibutramina ha però controindicazioni quali un aumentato rischio cardiovascolare, costipazione e insonnia.

Un altro farmaco usato per trattare l'obesità è l'\index{orlistat}orlistat che blocca il sito delle lipasi gastriche e pancreatiche bloccando la degradazione dei grassi e l'assorbimento che vengono quindi eliminate dalle feci con steatorrea, crampi addominali e flatuenza.

Rimedi chirurgici sono il bypass e il bendaggio gastrico. Attività fisica e dieta controllata sono i primi approcci terapeutici imprescindibili anche se coadiuvati da eventuale terapia farmacologica.

\subsection{Le displidemie o iperlipidemie}

La displidemia indica un elevato livello di lipidi nel sangue. 

I lipidi presenti nel sangue arrivano da una via esogena e da una via endogena.

Dalla via esogena, dal cibo presente nel lume intestinale, i lipidi vengono internalizzati da un recettore chiamato NPC1L1 presente sull'orletto a spazola degli enterociti e qui esterificati e inglobati in chilomiconi che, attraverso il sangue, raggiungono muscolo, tessuto adiposo e fegato.

La via endogena prevede la sintesi nel fegato da parte, tra l'altro, di un enzima, l'HMG-CoA reduttasi che risulta catalizzare la tappa limitante della sintesi dei grassi.

I grassi vengono poi inglobate da lipoproteine a formare micelle classifficate sulla base della densità in HDL-C, LDL-C, VLDL.

Le displidemie possono essere primarie o secondarie per diabete mellito, alcolemia, insufficienza renale cronica o per effetto collaterale da farmaci.

I farmaci che agiscono sulla via endogena sono le statine (\index{simvastina}simvastina) che inibisce la HMG-CoA reduttasi, i fibrati (\index{benzofibrato}benzofibrato) che attivano un gruppo di geni che trascrivono per le lipasi, le apoA1 (quindi HDL) e apoA5 che a sua volta stimola la produzione di lipasi.

I farmaci che inibiscono l'assorbimento di colesterolo sono l'\index{ezetimide}ezetimide che blocca il recettore NPC1L1 e, un po' in disuso, le resine leganti gli acidi biliari che, voluminose e di cattivo gusto, sequestrano gli acidi biliari a livello del lume intestinale evitandone il riassorbimento ma causano diarrea per iperosmolarità del contenuto intestinale.

Da citare che le statine sono anche usate nella prevenzione dell'infarto del miocardio e nella prevenzione di placche aterosclerotiche in pazienti con LDL alto.

\subsection{Farmaci anti epilettici}

L'epilessia è una anomala scarica parossistica dei neuroni corticali dovuta, in alcuni casi, ad un deficit di inibizione~\ce{GABA_A}--mediato e ipereccitabilità glutammato--mediata.

Le crisi si classificano in parziali o convulsioni con locus encefalico specifico e generalizzate. Le generalizzate possono essere caratterizzate da assenza o piccolo male, tonico--cloniche o grande male e miocloniche. Le parziali possono evolvere in grande male. Esiste anche una categoria a se stante per gli spasmi infantili.

Le convulsioni vengono trattate con \index{carbamazepina}carbamazepina e \index{valproato}valproato oppure con \index{clonazepam}clonazepam e \index{fenitoina}fenitoina. 

Le crisi tonico--cloniche vengono trattate con \index{carbamazepina}carbamazepina o \index{valproato}valproato o \index{fenitoina}fenitoina.

Le assenze con \index{etosuccimide}etosuccimide o \index{valproato}valproato.

Le crisi miotoniche con \index{diazepam}diazepam.

Gli spasmi infantili con \index{corticotropina}corticotropina.

Nelle emergenze si usa \index{diazepam}diazepam o altra benzodiazepina insieme a la \index{fosfofenitoina}fosfofenitoina, una molecola simil--\index{fenitoina}fenitoina da usare in IM o IV.

La \index{carbamazepina}carbamazepina è un antidepressivo triciclico che ininbisce i canali~\ce{Na^+} con alta scarica di frequenza evitando cosi il blocco dei neuroni nello stato normale. Blocca anche i canali~\ce{Ca^2+}. Interagisce accellerando il metabolismo di \index{fenitoina}fenitoina e \index{warfarin}warfarin. Sconsigliato in pazienti sotto \index{MAOI}MAOI. 

La \index{fenitoina}fenitoina ha azione simile alla \index{carbamazepina}carbamazepina ma può causare le assenze per cui non va usata in questa patologia. Viene trasportata dall'albumina ma \index{valproato}valproato e \index{sulfonamidi}sulfonamidi hanno maggiore affinità per cui aumentano la concentrazione di farmaco libero. Sono stati rilevati possibili effetti teratogeni.

Il \index{valproato}valproato aumenta i livelli di GABA inibendo due enzimi inattivanti questo neurotrasmettitore. Ha bassa tossicità e non ha effetti sedativi con scari effetti collaterali. Spazza la \index{fenitoina}fenitoina dalle proteine plasmatiche e inibisce il metabolismo di \index{fentobarbital}fentobarbital, usato come cura delle epilessie in età pediatrica, \index{fenitoina}fenitoina, \index{carbamazepina}carbamazepina.

L'\index{etosuccimide}etosuccimide inibisce i canali del~\ce{Ca^2+} a bassa soglia responsabili delle correnti nelle assenze ma può esacerbare crisi tonico--cloniche. Può portare inoltre a nausea, vertigini e reazioni di ipersensibilità.

\subsection{Calcio antagonisti}

I calcio antagonisti sono una classe di farmaci che, bloccando i canali del calcio, possono produrre effetti utili sulla muscolatura, soprattutto quella cardiaca grazie ad un ovvio effetto inotropo negativo che può essere usato per ridurre le richieste di ossigeno da parte del miocardio migliorando i sintomi anginosi.

Attualmente quasi tutti i farmaci \ce{Ca^2+}--antagonisti agiscono sui canali L del calcio presenti su cuore, muscolo scheletrico e liscio e neuroni.

La \index{nifedipina}nifedipina è una diidropiridina che agisce principalmente sulla muscolatura liscia arteriolare producendo una riduzione della pressione arteriosa. Non agendo sul circolo venoso risente in misura minore di effetti di ipotensione ortostatica presente, ad esempio, con i $\beta$--bloccanti. 

Scarso è l'effetto della \index{nifedipina}nifedipina sul nodo seno--atriale e sul nodo atrio--ventricolare dove invece il \index{diltiazem}diltiazem, una benzotiazepina, e il \index{verapamil}verapamil sono molto più efficaci e sono ampiamente utilizzati nel trattamento dell'angina.

Questi hanno anche un effetto anti--adrenergico che contribuisce a una generale vasodilatazione periferica anche se minore delle diidropiridine per cui sono da preferirsi nel caso di pazienti con ipotensione.

La tossicità di questa classe di farmaci è una diretta conseguenza della loro azione terapeurica con una diminuzione eccessiva di inotropismo che può portare a bradicardie, scompensi e arresti cardiaci.

La \index{nifedipina}nifedipina è efficace anche nel bloccare il parto prematuro anche se, per la sua maggiore tossicità, si preferisce l'uso dell'\index{atosiban}atosiban, un antagonista dell'ossitocina.

\newpage