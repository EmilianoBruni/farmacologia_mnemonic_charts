\part{Esami}
\section{Temi svolti}

\subsection{$\beta$--bloccanti}

I farmaci $\beta$--bloccanti agiscono sul recettore adrenergico~$\beta$, recettore metabotropo a proteina~G di tipo~\ce{G_s} prevalentemente stimolatorio presente nel cuore, adipociti, apparato iuxaglomerulare~($\beta_1$), nel muscolo liscio~($\beta_2$) e nella vescica~($\beta_3$).

Il recettore attiva la cascata di segnalazione intracellulare tramite aumento di cAMP.

Gli effetti dei farmaci $\beta$--bloccanti agiscono nell'aparato cardiocircolatorio sia inibendo il sistema renica---angiotensina con riduzione del tono arteriolare con conseguente diminuzione della pressione e del post--carico, sia come effetto diretto inotropo e cronotropo negativo sul muscolo cardiaco.

Il \index{propanololo}propanololo agisce su titti i recettori di questa famiglia ed è usato principalmente nell'angina, nell'infarto e nell'insufficienza cardiaca per ridurre le richieste metaboliche del miocardio, nelle aritmie, come farmaco di classe~II per ridurre il potenziale d'aziene e aumentare il periodo refrattario~AV.

Il \index{metoprololo}metoprololo viene usato nelle emicranie e nel tremore muscolare.

Il \index{labetalolo}labetalolo (anche $\alpha$--bloccante) è usato per bloccare la cascata adrenergica introdotta dal feocromocitoma. Il \index{timololo} viene usato per la gestione del glaucoma.

I $\beta$--bloccanti lipofilici tipo il \index{propanololo}propanololo sono assunti per os e ben assorbiti, con intenso metabolismo epatico.

I $\beta$--bloccanti idrofilici tipo l'\index{atenololo}atenololo, non sono ben assorbiti per os. 

Eventuali effetti collaterali sono il blocco della conduzione~AV soprattutto in unione con i~\ce{Ca^2+}--antagonisti, reazioni broncocostrittive e aumento della glicemia.

\subsection{Farmaci anti--psicotici}

La schizzofrenia è una persistente alienazione del pensiero che da sia disturbi positivi, con caratteristiche psicologiche aggiunte quali delirio, allucinazioni e aggressività, sia sintomi negativi, con caratteristiche psicologiche perse quali isolamento sociale, apatia e mancanza di iniziativa.

Le teorie fisiopatologiche di questo disturbo sono tre. Una teoria dopaminergica che spiega i sintomi positivi che deriva la patologia da una iperstimolazione dei recettori~\ce{D_2}/\ce{D_4}; una teoria glutammatergica che deriva la patologi da bassi livelli di glutammato e da una conseguente iperstimolazione del recettore~NMDAM; una teoria serotoninergica derivata dall'osservazione che gli antagonisti serotoninergici sono antipsicotici e che l'LSD, un agonista~5-HT, fa venire i sintomi positivi.

I farmaci usati si dividono in tipici a atipici. I tipici quali i fenotiazidici come la \index{clorpromazina}clorpromazina e i butirrofenonici come l'\index{aloperidolo}aloperidolo (Serenase), agiscono bloccando i recettori dopaminergici diminuendo i sintomi positivi e, i più recenti, anche quelli negativi ma hanno effetti collaterali sul sistema extrapiramidale come distonie acute e tardive. 

Gli atipici non hanno effetti sulla via extrapiramidale in quando bloccano selettivamente la via mesolimbica (della gratificazione) ignorando la via nigrostriata e sono quindi usati principalmente se gli effetti collaterali dei tipici sono eccessivi. Tali farmaci bloccano anche i recettori $\alpha$--adrenergici e i~5-HT e sono la \index{clorapina}clorapina (una dibenzodiazepina) e l'\index{olanzapina}olanzapina.

Il difetto di tutti i farmaci anti--psicotici descritti è che impiegano settimane prima del loro effetto terapeutico e questo è un segno che vi deve essere un qualche altro effetto secondario ad agire come, ad esempio, l'aumento dei~\ce{D_2} a livello limbico.

\subsection{Farmaci antianemici}

L'anemia è una riduzione della massa eritrocitaria nel sangue misurabile come riduzione dei valori di Hb. Le cause principali sono carenza di ferro, talassemie e emoglobinopatie con caratteristiche microcitiche (MCV < 80 fl), emolitiche, aplastiche o emorragiche con caratteristiche normocitiche ( 100 < MCV < 80 fl) e deficit di \ce{B_12} o folati con caratteristiche macrocitiche (MCV > 100 fl).

La carenza di ferro si ha nel caso di amoraggie croniche, aumento del fabbisogno o diminuzione nell'assorbimento e si cura con sali ferrosi per os come il solfato ferroso o parenterale com il ferro destrano nel caso di intolleranza alla terapia per os. L'assorbimento segue le stesse vie del ferro alimentare. Le reazioni avverse vanno da disfunzioni intestinali quali diarrea, vomito e nausea fino alla gastrite necrotizzante nell'intossicazione acuta. Per via parenterale vi possono essere anche casi di reazione anafilattica.

Nel caso di sovradosaggi può essere utile l'impiego di chelanti del ferro come la \index{desferrioxamina}desferrioxamina.

Nel caso di anemia megaloblastica, dato che i sintomi del deficit di \ce{B_12} e di folati sono molto simili, va valutato dapprima l'eventuale deficit di \ce{B_12} eliminando dubbi sulla presenza di anemia perniciosa da deficit del fattore primario, malassorbimento primitivo, gravidanza e eliminazione dalla dieta della vitamina \ce{B_12}. Il deficit di tale vitamina, al contrario di quella dei folati, causa anche problemi neurologici che non si risolvono una volta risolto il deficit.

Le reazioni avverse alla vitamina~\ce{B_12} sono la trombosi, policitemia e insufficienza cardiaca fino allo shock anafilattico. Non si registrano grosse reazioni avverse all'uso dei folati.

Nel caso di anemie aplastiche, disordini midollari e insufficienza renale sono utili i fattori di crescita emopoietici come l'\index{eritropoietina}eritropoietina e la \index{darbepoetina $\alpha$}darbepoetina $\alpha$ con reazioni avverse un eventuale aumento dell'ematocrito e aumentata viscosità oltre che al prurito.

\subsection{Farmaci per il trattamento dell'obesità}

L'obesità è una malattia multifattoriale e poligenica in cui l'apporto calorico nel lungo periodo è superiore al consumo energetico causando un aumento del BMI, l'indice di massa corporea.

I principali fattori che entrano nella regolazione del cibo e del consumo energetico sono la leptina, la colecistochina (CCK), l'insulina, il sistema nervoso simpatico e fattori psico--socio--economici.

La leptina è sintetizzata dalle cellule adipose e il suo aumento dovrebbe portare ad un effetto anoressizzante ma nei pazienti obesi tale effetto è mancante per qualche forma di resistenza dovuta a degradazione, a difetto del trasportatore o inefficacia dei recettori.

La sintesi della leptina è regolata positivamente da glucocorticoidi, insulina. Una regolazione negativa è data da agonisti $\beta$--adrenergici.

La CCK agisce sul rilascio di bile, stimolando la secrezione di insulina e attiva la stimolazione vagale portando un effetto di sazietà.

L'insulina stimola la leptina ma nell'obeso, essendo insensibile ciò causa ipertensione.

Il sistema nervoso simpatico invece causa un effetto termogenico grazie alla fosforilazione ossidativa disaccoppiata nelle cellule brune e relativo aumento de consumi energetici.

L'obesità causa patologie secondarie quali il diabete mellito la cui terapia, l'insulina, causa un ulteriore aumento di assunzione di cibo, malattie cardiovascolari, tumori ormoni dipendenti, probelmi digestivi e respiratori e osteoartriti.

I farmaci usati sono la \index{sibutramina}sibutramina che inibisce la ricaptazione della serotonina e noradrenalina (IRSN) agendo negativamente sui siti regolanti l'appetito con aumento della sazietà, diminuzione del BMI, diminuzione di LDL e aumento di HDL. La sibutramina ha però controindicazioni quali un aumentato rischio cardiovascolare, costipazione e insonnia.

Un altro farmaco usato per trattare l'obesità è l'\index{orlistat}orlistat che blocca il sito delle lipasi gastriche e pancreatiche bloccando la degradazione dei grassi e l'assorbimento che vengono quindi eliminate dalle feci con steatorrea, crampi addominali e flatuenza.

Rimedi chirurgici sono il bypass e il bendaggio gastrico. Attività fisica e dieta controllata sono i primi approcci terapeutici imprescindibili anche se coadiuvati da eventuale terapia farmacologica.

\subsection{Le displidemie o iperlipidemie}

La displidemia indica un elevato livello di lipidi nel sangue. 

I lipidi presenti nel sangue arrivano da una via esogena e da una via endogena.

Dalla via esogena, dal cibo presente nel lume intestinale, i lipidi vengono internalizzati da un recettore chiamato NPC1L1 presente sull'orletto a spazola degli enterociti e qui esterificati e inglobati in chilomiconi che, attraverso il sangue, raggiungono muscolo, tessuto adiposo e fegato.

La via endogena prevede la sintesi nel fegato da parte, tra l'altro, di un enzima, l'HMG-CoA reduttasi che risulta catalizzare la tappa limitante della sintesi dei grassi.

I grassi vengono poi inglobate da lipoproteine a formare micelle classifficate sulla base della densità in HDL-C, LDL-C, VLDL.

Le displidemie possono essere primarie o secondarie per diabete mellito, alcolemia, insufficienza renale cronica o per effetto collaterale da farmaci.

I farmaci che agiscono sulla via endogena sono le statine (\index{simvastina}simvastina) che inibisce la HMG-CoA reduttasi, i fibrati (\index{benzofibrato}benzofibrato) che attivano un gruppo di geni che trascrivono per le lipasi, le apoA1 (quindi HDL) e apoA5 che a sua volta stimola la produzione di lipasi.

I farmaci che inibiscono l'assorbimento di colesterolo sono l'\index{ezetimide}ezetimide che blocca il recettore NPC1L1 e, un po' in disuso, le resine leganti gli acidi biliari che, voluminose e di cattivo gusto, sequestrano gli acidi biliari a livello del lume intestinale evitandone il riassorbimento ma causano diarrea per iperosmolarità del contenuto intestinale.

Da citare che le statine sono anche usate nella prevenzione dell'infarto del miocardio e nella prevenzione di placche aterosclerotiche in pazienti con LDL alto.

\subsection{Farmaci anti epilettici}

L'epilessia è una anomala scarica parossistica dei neuroni corticali dovuta, in alcuni casi, ad un deficit di inibizione~\ce{GABA_A}--mediato e ipereccitabilità glutammato--mediata.

Le crisi si classificano in parziali o convulsioni con locus encefalico specifico e generalizzate. Le generalizzate possono essere caratterizzate da assenza o piccolo male, tonico--cloniche o grande male e miocloniche. Le parziali possono evolvere in grande male. Esiste anche una categoria a se stante per gli spasmi infantili.

Le convulsioni vengono trattate con \index{carbamazepina}carbamazepina e \index{valproato}valproato oppure con \index{clonazepam}clonazepam e \index{fenitoina}fenitoina. 

Le crisi tonico--cloniche vengono trattate con \index{carbamazepina}carbamazepina o \index{valproato}valproato o \index{fenitoina}fenitoina.

Le assenze con \index{etosuccimide}etosuccimide o \index{valproato}valproato.

Le crisi miotoniche con \index{diazepam}diazepam.

Gli spasmi infantili con \index{corticotropina}corticotropina.

Nelle emergenze si usa \index{diazepam}diazepam o altra benzodiazepina insieme a la \index{fosfofenitoina}fosfofenitoina, una molecola simil--\index{fenitoina}fenitoina da usare in IM o IV.

La \index{carbamazepina}carbamazepina è un antidepressivo triciclico che ininbisce i canali~\ce{Na^+} con alta scarica di frequenza evitando cosi il blocco dei neuroni nello stato normale. Blocca anche i canali~\ce{Ca^2+}. Interagisce accellerando il metabolismo di \index{fenitoina}fenitoina e \index{warfarin}warfarin. Sconsigliato in pazienti sotto \index{MAOI}MAOI. 

La \index{fenitoina}fenitoina ha azione simile alla \index{carbamazepina}carbamazepina ma può causare le assenze per cui non va usata in questa patologia. Viene trasportata dall'albumina ma il \index{valproato}valproato ha maggiore affinità per cui aumentano la concentrazione di farmaco libero. Sono stati rilevati possibili effetti teratogeni.

Il \index{valproato}valproato aumenta i livelli di GABA inibendo due enzimi inattivanti questo neurotrasmettitore. Ha bassa tossicità e non ha effetti sedativi con scari effetti collaterali. Spazza la \index{fenitoina}fenitoina dalle proteine plasmatiche e inibisce il metabolismo di \index{fentobarbital}fentobarbital, usato come cura delle epilessie in età pediatrica, \index{fenitoina}fenitoina, \index{carbamazepina}carbamazepina.

L'\index{etosuccimide}etosuccimide inibisce i canali del~\ce{Ca^2+} a bassa soglia responsabili delle correnti nelle assenze ma può esacerbare crisi tonico--cloniche. Può portare inoltre a nausea, vertigini e reazioni di ipersensibilità.

\subsection{Calcio antagonisti}

I calcio antagonisti sono una classe di farmaci che, bloccando i canali del calcio, possono produrre effetti utili sulla muscolatura, soprattutto quella cardiaca grazie ad un ovvio effetto inotropo negativo che può essere usato per ridurre le richieste di ossigeno da parte del miocardio migliorando i sintomi anginosi.

Attualmente quasi tutti i farmaci \ce{Ca^2+}--antagonisti agiscono sui canali L del calcio presenti su cuore, muscolo scheletrico e liscio e neuroni.

La \index{nifedipina}nifedipina è una diidropiridina che agisce principalmente sulla muscolatura liscia arteriolare producendo una riduzione della pressione arteriosa. Non agendo sul circolo venoso risente in misura minore di effetti di ipotensione ortostatica presente, ad esempio, con i $\beta$--bloccanti. 

Scarso è l'effetto della \index{nifedipina}nifedipina sul nodo seno--atriale e sul nodo atrio--ventricolare dove invece il \index{diltiazem}diltiazem, una benzotiazepina, e il \index{verapamil}verapamil sono molto più efficaci e sono ampiamente utilizzati nel trattamento dell'angina.

Questi hanno anche un effetto anti--adrenergico che contribuisce a una generale vasodilatazione periferica anche se minore delle diidropiridine per cui sono da preferirsi nel caso di pazienti con ipotensione.

La tossicità di questa classe di farmaci è una diretta conseguenza della loro azione terapeurica con una diminuzione eccessiva di inotropismo che può portare a bradicardie, scompensi e arresti cardiaci.

La \index{nifedipina}nifedipina è efficace anche nel bloccare il parto prematuro anche se, per la sua maggiore tossicità, si preferisce l'uso dell'\index{atosiban}atosiban, un antagonista dell'ossitocina.

\subsection{Chinoloni}

I chinoloni sono farmaci ad azione battericida analoghi sintetici fluorati dell'\index{acido nalixidico}acido nalixidico che interferiscono con la sintesi di DNA batterico inibendo la topoisomerasi II, una DNA girasi che previene lo svolgimento di DNA superspiralizzato impedendo quindi trascrizione e replicazione.

Questa famiglia inibisce anche la topoisomerasi IV coinvolta con la separazione del DNA cromosomico.

Vengono classificati sulla base dello sviluppo temporale della molecola in una prima generazione comprendente lo stesso \index{acido nalixidico}acido nalixidico e i primi suoi derivati fluorati coma la \index{norfloxacina}norfloxacina che agisce sui gram- principalmente nelle infezioni urinarie.

La seconda generazione ha un più ampio spettro di azione come la \index{ciprofloxacina}ciprofloxacina con maggiore attività sui gram-, gonococco, molti gram+ e micobatteri. Questa generazione da però precocemente la comparsa di ceppi resistenti.

La terza generazione come la \index{levofloxacina}levofloxacina ha spettro di azione minore contro i grma- ma maggiore sui gram+, S. pneumonia e alcuni entero e staffilococchi meticillina resistenti.

Gli ultimi, di più recente introduzione coma la \index{moxifloxacina}moxifloxacina sono l'ultima linea di difesa contro ceppi resistenti, hanno il più ampio spettro di azione e hanno attività ance contro gli anerobi.

I meccanismi di resistenza presuppongono lo sviluppo di pompe di efflusso e cambi strutturali nelle porine al fine di ridurre la concentrazione di farmaco assorbito e una ridotta sensibilità agli enzimi bersaglio oltre che mutazioni del gene delle DNS girasi.

Gli effetti indesiderati sono disturbi gastrointestinali, eritemi, insonnia, tendiniti e rottura dei tendini, in particolare di quello di Achille. 

Vi possono poi essere sovrainfezioni di candida e streptococco.

Nelle interazioni, aumentano i livelli plasmatici delle \index{metilxanine}metilxantine.

\subsection{Farmaci procinetici}

Sono una famiglia di farmaci che agiscono a livello del sistema nervoso enterico con effetti su aumento della peristalsi e relativa diminuzione del tempo di svuotamento gastrico, aumento del tono dello sfintere gastroesofageo e conseguente prevenzione del reflusso gastroesofageo, effetto antiemetico.

Questi farmaci sono quindi utili per il trattamento di emesi e di alterazioni motorie del tratto  gastroesofageo come dispepsia funzionale, stasi gastrica e reflusso gastroesofageo oltre alla sindrome da intestino irritabile.

I farmaci di prima generazione, della famiglia dei buttirrofenoni, come il \index{metoclopramide}metoclopramide o \index{plasil|see {metoclopramide}}plasil e il \index{domperidone}domperidone sono antidopaminergici, bloccando il recettore \ce{D_2} della dopamina e agiscono principalmente sul tratto prossimale dove l'attivazione dei recettori dopaminergici inibisce la stimolazione colinergica della muscolatura liscia aumentando la peristalsi e la pressione dello sfintere gastroesofageo.

Bloccando i \ce{D_2} anche nel bulbo, nella zona del grilletto, produce importanti effetti antinausea e antiemetici.

Il \index{domperidone}domperidone, al contrario del \index{plasil|see {metoclopramide}}plasil non attraversa la barriera ematoencefalica ed ha quindi minori effetti centrali quali parkinsonismo, effetti extrapiramidali quali trisma e torcicollo e senso di spossatezza propri del \index{metoclopramide}metoclopramide.

Il \index{domperidone}domperidone favorisce la montata lattea in quanto è un suo effetto collaterale.

I farmaci di seconda generazione come il \index{metoclopramide}metoclopramide o l'\index{ondansetron}ondansetron e agiscono come antagonisti del recettore serotoninergico \ce{5-HT_3} che bloccato, favorisce l'azione procinetica. Agendo nel tratto prossimale, questa famiglia di farmaci ha azione antiemetica.

I farmaci di terza generazione della famiglia dei benzofurani e aminoguanidil-indoli, sono agonisti del recettore \ce{5-HT_4} che stimolato, ha azione procinetica sul tratto distale. Questi sono utili nelle sindomi da colon irritato e hanno come effetti collaterali emicrania, sonnolenza e vertigini.

\subsection{Dipendenza da alcol}

L'assunzione per lunghi tempi di grandi quantità di alcolici instaura una dipendenza di tipo fisico e psicologico.

L'etanolo è una piccola molecola idrosolubile che viene rapidamente assorbita nel tratto gastrointestinale. La sua distribuzione è rapida con livelli tissutali e nel sistema nervoso centrale simili a quelli ematici. Viene eliminato principalmente nel fegato ad opera di alcol-deidrogenasi che la metabolizzano in acetaldeide a poi l'aldeide deidrogenasi la converte in acetato. Una parte viene invece eliminata da enzimi microsomiali del citocromo P450 e in parte per via respiratoria e nelle urine.

L'alcol agisce deprimendo il sistema nervoso centrale aumentando la funzione recettoriale del \ce{GABA_A} e diminuendo la trasmissione eccitatoria tramite inibizione dei recettori NMDA e kainato del glutammato.

Inoltre l'alcol aumenta la sintesi di POMC con effetti gratificanti simil oppioidi.

Per dosi moderate ha effetti psichici di disinibizione, espansività, loquacità, aumento della libido ma anche sonnolenza e disturbi dell'attenzione mentre ha effetti neurovegetativi come innalzamento della soglia del dolore, freddo, riduzione della coordinazione motoria.

Per dosi elevate ha effetti prettamente vegetativi con profonda depressione del SNC, depressione respiratoria, coma etilico, collasso cardiocircolatorio, morte.

I danni derivanti invece da una somministrazione cronica sono principalmente a carico del fegato con dapprima steatosi epatica reversibile fino alla cirrosi, danni muscolari, cardiopatie, pancreatiti, anemia, aumentato incidenza di cancro del gastrointestinale, impotenza.

Da ricordare la sindrome di Wernike-Korsakoff con paralisi dei muscoli oculari, atassia, amnesia fino alla demenza causata da carenza di assorbimento della tiamina (vitamina B1).

Gravi danni sul feto nel caso di utilizzo durante la gravidanza con ritardo mentale, alterazioni del SNC causata da migrazioni neuronale aberrante.

L'interruzzione dell'assunzione cronica causa come detto un sindrome da astinenza fisica con sintomi che partono da tremore, tachicardia, irritabilità fino a disturbi percettivi e allucinazioni, per sfociare nei casi più gravi in convusioni tonico-cloniche e delirium tremens con pericolo anche per la vita se non gestito.

Il principale farmaco usato per dissuadere i pazienti dal bere alcol è il \index{disulfiram}disulfiran, un inibitore dell'aldeide deidrogenasi che causa quindi un accumulo di acetaldeide nel sangue con spiacevoli sintomi quali nausea, vomito, vampate di calore. Analogamente un antagonista oppioide come il \index{naloxone}naloxone o il \index{naltrexone}naltrexone, blocca le proprietà di rinforzo dell'alcol causate dalla sintesi di POMC. Sempre come dissuasore si può usare un inibitore del reuptake della serotonica quale la \index{fluoxetina}fluoxetina (\index{prozac|see {fluoxetina}}prozac. Antiemetici antagonisti del recettore \ce{5-HT_3} come \index{ondansetron}ondansetron possono essere efficaci nell'alcolismo ad insorgenza precoce.

La disintossicazione si attua alleviando i sintomi della sindrome da astinenza. L'ansia con \index{benzodiazepine}benzodiazepine come a lunga durata come il \index{diazepam}diazepam o a breve durata come il \index{lorazepam}lorazepam; le eventuali convulsioni con farmaci anticonvulsivanti quali la \index{vigabatrina}vigabratina; farmaci antipsicotici per i deliri e le allucinazioni e $\beta$--bloccanti adrenegici o agonisti $\alpha_2$--adrenergici quali la \index{clonidina}clonidina per il controllo della sintomatologia neurovegetativa. Integrazione di fluidi ed elettroliti è altresi fondamentale come lo il \index{$\gamma$--idrossibutirrato}$\gamma$--idrossibutirrato o \index{GHB|see {$\gamma$--idrossibutirrato}}GHB con il nome commerciale di \index{alcover|see {$\gamma$--idrossibutirrato}}alcover per lo svezzamento anche noto come metadone dell'alcol.

La forma acuta si tratta con la gestione delle funzioni vitali, l'eventuale intubazione e ventilazione meccanica, lavanda gastrica e emodialisi.

\subsection{Farmaci anti--\ce{H_1}}

L'istamina è un importante mediatore delle reazioni allergiche oltre che con funzioni di neurotrasmettitore e neuromodulatore.

L'isamina si trova principalmente sequestrata in forma legata in mastociti e basofili. Il principale meccanismo di rilascio dell'instamina in forma libera è immunologico ove queste cellule, sensibilizzate da anticorpi di tipo IgE rilasciano forti dosi di istamina quando esposti ad un opportuno antigene.

Una volta in circolo essa esercita azioni biologiche legandosi a 4 diversi recettori \ce{H_{1--4}} associati a proteine~G.

Gli effetti indesiderati sono esarcebazioni dei loro effetti fisiologici che vanno da arrossamenti cutanei e ponfi, ipotensione per vasodilatazione dei letti arteriolari e rilascio degli sfinteri pre--capillati, tachicardia, edema tra cui il temutissimo edema della lottide, broncocostrizione.

Gli antagonisti del recettore \ce{H_1} sono utilizzati da lungo tempo come anti--allergici (antistaminici), anti--nausea e nelle chinetosi.

Si classificano in farmaci di prima generazione con effetti sedativi che, per alcuni, con spiccato effetto sedativo, questo effetto collaterale è usato a scopo terapeutico, e di II generazione senza tali effetti grazie alla loro non completa diffusione nel sistema nervoso centrale.

Tutti gli anti--\ce{H_1} sono ammine rapidmente assorbibili per os e metabolizzate da O450 epatico con durata d'azione di poche ore per quelli di prima generazione e fino a un giorno per quelli di seconda.

La maggior parte degli anti--\ce{H_1} sono agonisti inversi mentre altri sono antagonisti.

Come detto, il principale effetto degli anti--\ce{H_1} è antistaminico ed è utile sia nella prevenzione che nel trattamenti dei sintomi di riniti allergiche e orticaie. L'edema scatenato dall'istamina non è invece influenzato dagli anti--\ce{H_1} in quanto sembra dipendere da peptidi della famiglia delle chinine.

Un'altra azione importante degli anti--\ce{H_1} è antinausea e antiemetica. La \index{scopolamina}scopolamina è l'agente più usati nelle chinetosi mentre la \index{doxilamina}doxilamina è usata nelle nausee da gravidanza anche se questo farmaco ha avuto una storia di caccia alle streghe di aumentata incidenza di malformazioni neonatali che però non furomo mai dimostrate anche se la cattiva pubblicità fece ritirare dal mercato il prodotto commerciale che lo conteneva come principio attivo.

Altri farmaci anti--\ce{H_1} come la \index{difenidramina}difenidramina sopprimono i sintomi collaterali acuti extrapiramidali di molti antipsicotici.

Gli effetti avversi degli anti--\ce{H_1} sono sedazione e azione anticolinergica. Nei bambini, al contrario, si può avere eccitazione e convulsioni.

Nei casi di sovradosaggio gli effetti sono simil--atropina e sono trattati nello stesso modo.

\subsection{Agenti chelanti}

GLi agenti chelanti sono farmaci che, legandosi ai metalli pesanti, rendono quest'ultimi non disponibili per possibili interazioni tossiche.

Essi contengono uno o più atomi di coordinazione che doneranno elettroni per formare legami covalenti chiamati legami dentati con cationi metallici.

Esempi di metalli e relativi agenti chelanti sono:

:Piombo
;\index{dimercaprolo}dimercaprolo e \index{EDTA}EDTA nella fase iniziale e \index{succimer}succimer per os come mantenimento

:Arsenico
;\index{dimercaprolo}dimercaprolo o \index{unitolo}unitolo in acuto, \index{succimer}succimer per os in cronico

:Mercurio
;\index{dimercaprolo}dimercaprolo + \index{unitolo} in acuto, \index{unitolo}unitolo + \index{succimer}succimer in cronico. Il \index{dimercaprolo}dimercaprolo non viene usato perchè sposta il mercurio metallico e organico dai siti al SNC quindi non va usato in questi tipi di intossicazione

Quindi il legamen  con agenti chelanti può aumentare l'escrezione del metallo ma può, come per il \index{dimercaprolo}dimercaprolo e il mercurio metallico, spostarlo da un sito ad un altro.

Un altro esempio è il Cadmio dove gli agenti chelanti non possono essere usati perchè redistribuiscono il Cadmio nel rene aumentando la sua nefrotossicità e annullando l'uso terapeutico.

Dato che c'è la possibilità che agenti chelanti possano ridurre la disponibilità di cationi esseziali come lo zinco, andrebbe considerata una integrazione di questi cationi con la dieta.

Vediamo nello specifico alcuni agenti chelanti più importanti.

Il \index{dimercaprolo}dimercaprolo è un liquido oleoso con odore di uova marce sviluppato in inghilterra durante la seconda guerra mondiale come antidoto per un'arma chimica a base di arsenico chiamata Lewisite. Per questo il \index{dimercaprolo}dimercaprolo è noto con il nome di \index{BAL|see {dimercaprolo}}BAL British Anti--Lewisite. Viene somministato con iniezioni IM molto dolorose. Le reazioni avverse sono tachicardia, vomito, febbre, trombocitopenia che porta ad ematomi nella sede di iniezione.

Il \index{succimer}succimer è un analogo idrosolubile del \index{BAL|see {dimercaprolo}}BAL con una formulazione per os. Le sue azioni avverse sono nausea, vomito, diarrea e eruzioni cutanee.

L'\index{EDTA}EDTA è un agente chelante dato come edetano calcio bisodico per evitare il dupaperamento del calcio. Ionico quindi somministrato per EV è nefrotossico se non vi è un opportuno volume urinario escreto.

Il \index{DTPA}DTPA dietil--EDTA è usato contro uranio, plutonio, americio e curio.


\newpage