\maketitle
	\begin{abstract}
	Questo articolo riassume con delle carte mnemoniche gli argomenti di farmacologia spiegati nel IV anno del corso di laurea in medicina e chirurgia a Chieti.
	L'uso di questo articolo non sostituisce la lettura e lo studio di un libro e degli appunti di farmacologia.

Per errori, omissioni o altre note, non esitate a contattarmi via e-mail.

Potete utilizzare direttamente il PDF compilato o ricrearlo compilando i sorgenti utilizzando il \LaTeX. 

Potete anche modificare, correggere e integrare il documento a patto di rilasciarlo con la stessa sua licenza.

Questo documento è rilasciato secondo la licenza Creative Commons CC-BY-NC-SA 2.0 IT (\url{https://creativecommons.org/licenses/by-nc-sa/2.0/it/})

Tu sei libero di:

Condividere — riprodurre, distribuire, comunicare al pubblico, esporre in pubblico, rappresentare, eseguire e recitare questo materiale con qualsiasi mezzo e formato 

Modificare — remixare, trasformare il materiale e basarti su di esso per le tue opere

Alle seguenti condizioni:

Attribuzione — Devi riconoscere una menzione di paternità adeguata, fornire un link alla licenza e indicare se sono state effettuate delle modifiche. Puoi fare ciò in qualsiasi maniera ragionevole possibile, ma non con modalità tali da suggerire che il licenziante avalli te o il tuo utilizzo del materiale.

NonCommerciale — Non puoi utilizzare il materiale per scopi commerciali.

StessaLicenza — Se trasformi il materiale o ti basi su di esso, devi distribuire i tuoi contributi con la stessa licenza del materiale originario.

Divieto di restrizioni aggiuntive — Non puoi applicare termini legali o misure tecnologiche che impongano ad altri soggetti dei vincoli giuridici su quanto la licenza consente loro di fare.
	
Documento originale e aggiornato su \url{https://github.com/EmilianoBruni/farmacologia_mnemonic_charts}
\end{abstract}

\newpage

\tableofcontents

\newpage\newpage