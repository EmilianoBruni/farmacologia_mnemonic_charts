\chapter{Farmacocinetica}

La curva dose/concentrazione di un farmaco si basa su una media del funzionamento del farmaco. Ma questa curva può variare di molto da individuo a individuo.

I due parametri principali che influenzano questa variabilità sono il \textbf{volume di distribuzione} e la \textbf{cleareance}.

\section{Volume di distribuzione}

\`E quel volume teorico che contiene quella determinata concentrazione di farmaco nel comparto in oggetto, spesso il flusso sanguigno

$$V =\frac{\text{quantità di farmaco nell'organismo}}{[F]}$$

ove $[F]$ può essere riferito a sangue, plasma, farmaco libero.

Come si vede dalla relazione $V$ è uno spazio virtuale in quanto si presuppone che la concentrazione misurata in $[F]$ sia omogenea su tutto il corpo.

Quindi farmaci con distribuzione prettamente ematica danno delle $V$ piccole e confrontabili con il valore reale del compartimento che, per il sangue, è $\simeq 3\ce{L}/70\ce{Kg}$.

Farmaci con distribuzione prettamente extravascolare avremo piccole concentrazioni nel sangue e quindi $V$ elevati. Ad esempio, per la digossina, $V \simeq 500\ce{L}/70\ce{Kg}$.

La $V$ è utile, ad esempio, nel calcolo dell'emivita come vedremo a breve.

\section{Clereance}

\`E la quantità di farmaco eliminata nel tempo in rapporto alla sua concentrazione nel comparto in oggetto (sangue, plasma, farmaco libero)

$$CL =\frac{\text{velocità di eliminazione}}{[F]}$$

La clearenace è additiva per cui se un farmaco ha eliminazione renale, epatica e respiratoria

$$ CL_{tot} = CL_{\text{rene}} + CL_{\text{epat.}} + CL_\text{resp.} = \frac{V_{\text{rene}}^{\text{elim}}}{[F]} + \frac{V_{\text{epat.}}^{\text{elim}}}{[F]} + \frac{V_{\text{resp}}^{\text{elim}}}{[F]}$$

Per la maggior parte dei farmaci la clereance è costante all'interno del range di concentrazioni della pratica terapeutica per cui la velocità di eliminazione dipende solo dalla concentrazione del farmaco

$$\ce{Vel.} = CL \cdot [F]$$

Quando ciò accade si parla di cinematica di ordine 1. In questo caso la clereance può essere calcolata misuando l'area sotto la

\section{Emivita}

L'emivita di un farmaco è definita come il tempo necessario a ridurre
il farmaco a \unitfrac{1}{2} della quantità di farmaco presente nell'organismo
allo steady-state.

Presupponendo che la quantità di farmaco nell'organismo abbia un andamento esponenziale decrescente con il tempo, si pu definire questo matematicamente come:

$$
Q(t)=\alpha e^{-\beta t}
$$

Per trovare i due parametri $\alpha$ e $\beta$ consideriamo che a $t=0$
$Q(0)=Q_{\text{TOT}}=\alpha$ e quindi l'equazione sopra si pro scrivere come
$$
Q(t)=Q_{\text{TOT}}e^{-\beta t}
$$
e d'altra parte se consideriamo la velocità di eliminazione del farmaco
al tempo $t$ si ha che\vspace{.5em}

$-\dfrac{\text{d}\,Q(t)}{\text{d}\,t}=v_{\text{elim}}(t)=-Q_{\text{TOT}}(-\beta)e^{-\beta t}$ \vspace{.5em}

Ma d'altra parte, per definizione
$$
\text{CL} = \dfrac{v_\text{ELIM}^\text{STEADY STATE}}{c^\text{STEADY STATE}} =\dfrac{v_\text{ELIM}(0)}{c(0)}
$$
e, a $t=0\Rightarrow v_{\text{elim}}(0)=\text{CL}\cdot c(0)=-Q_{\text{TOT}}(-\beta)$ da cui
$\beta=\dfrac{\text{CL\ensuremath{\cdot}}c(0)}{Q_{\text{TOT}}}$ ma

$$V_\text{DIST} = \dfrac{Q_{\text{TOT}}}{c(0)}$$ 

e quindi \vspace{.5em}

$\beta=\dfrac{\text{CL\ensuremath{\cdot}}\cancel{c(0)}}{V_{\text{DIST}}\cdot\cancel{c(0)}}\Rightarrow\beta=\dfrac{\text{CL}}{V_{\text{DIST}}}$ e quindi

$$
Q(t)=Q_{\text{TOT}}e^{-\frac{\text{CL}}{V_{\text{DIST}}}t}
$$


a $t=t_{\unitfrac{1}{2}}\Rightarrow Q(t_{\unitfrac{1}{2}})=\dfrac{1}{2}\cancel{Q_{\text{TOT}}}=\cancel{Q_{\text{TOT}}}e^{-\frac{\text{CL}}{V_{\text{DIST}}}t_{\unitfrac{1}{2}}}$ 

e passando ai logaritmi naturali

$\ln\dfrac{1}{2}=-\dfrac{\text{CL}}{V_{\text{DIST}}}t_{\unitfrac{1}{2}}\Rightarrow t_{\unitfrac{1}{2}}=\ln\dfrac{1}{2}\cdot\left(-\dfrac{V_{\text{DIST}}}{\text{CL}}\right)=\dfrac{\ln2\cdot V_{\text{DIST}}}{\text{CL}}$

e quindi

\[
t_{\unitfrac{1}{2}}\simeq0.7\cdot\dfrac{V_{\text{DIST}}}{\text{CL}}
\]

