\documentclass[12pt,paper=a4,twoside=false,parskip=half]{scrartcl}
% aggiungere draft alla classe per vedere gli overfull hbox

% /-- Packages loading --------------------------------------------------------\
\usepackage[italian]{babel}
\usepackage[utf8]{inputenc} 	% lettere accentate in documento UTF-8
\usepackage[T1]{fontenc} 		% doppi --
\usepackage{lmodern}
\usepackage{amsmath}
\usepackage{amsthm}
\usepackage{amsfonts}
\usepackage{units}
\usepackage{cancel}         
\usepackage{tikz}
\usepackage{tikz-qtree}
\usepackage{fancyhdr}           % Header e Footer 
\usepackage{chemfig}
\usepackage[version=3]{mhchem}
\usepackage{makeidx}			% Indice
\usepackage[colorlinks=true,allcolors=blue]{hyperref}
% \----------------------------------------------------------------------------/

% /-- title and authors -------------------------------------------------------\
\def\gtitle#1{\gdef\gtitle{#1}}
\def\gauthor#1{\gdef\gauthor{#1}}
\title{Appunti di Farmacologia}
\gtitle{Appunti di Farmacologia}
\author{Emiliano Bruni (info@ebruni.it)}
\gauthor{Emiliano Bruni (info@ebruni.it)}
\date{} %%If commented, the current date is used.

\pdfinfo {
/Title (\gtitle)
/Author (\gauthor)
/Subject (Appunti di Farmacologia)
/Keywords (farmacology, university, medicine, book, charts, mnemonic, flash, card, italian, Chieti, appunti)
/Copyright (copyrighted)
}
% \----------------------------------------------------------------------------/

\def\tikzbyncsa{
	\draw[fill=black] (2pt,2pt) rectangle (78pt,13pt);
	\begin{scope}
	\clip (0,0) rectangle (80pt,15pt);
	\fill[gray!67] (12pt,7.5pt) circle [x radius=19pt,y radius=14pt];
	\node[draw,
		circle,
		inner sep=2pt,
		outer sep=0ex,
		very thick, fill=white] at (15pt,7.5pt) {\small\bfseries\textsf{cc}};
	\end{scope}
	\draw[line width=1pt, white] (78pt,14pt)--(1pt,14pt)--(1pt,1pt) -- (78pt,1pt);
	\draw[line width=1pt] (0,0) rectangle (80pt,15pt);
	\node[white] at (55pt,7.5pt) {\scriptsize\bfseries\textsf{BY-NC-SA}};
}

% /-- Sezione/numeropag. nell'header e date e numero di revisione nel footer --\
\pagestyle{fancy}
\newcommand{\helv}{\fontfamily{phv}\fontseries{b}\fontsize{8}{10}\selectfont}
\fancyhf{}
\fancyhead[LO]{\helv \rightmark}
\fancyhead[LO]{\helv \leftmark}
\fancyhead[RO]{\helv \thepage}
\fancyfoot[ol]{\begin{tikzpicture}[scale=0.6, transform shape,baseline=2pt ]\tikzbyncsa\end{tikzpicture}\ \scriptsize\itshape Copyright \copyright\ 2016 \gauthor }
\fancyfoot[or]{\scriptsize\itshape Revisione del \today } % Date
% \----------------------------------------------------------------------------/

\usetikzlibrary{shapes,arrows,matrix,calc,automata,positioning,patterns}
\numberwithin{equation}{section}  % numeri delle equazioni del tipo x.y
\numberwithin{table}{section}     % numeri delle tabelle del tipo x.y
\numberwithin{figure}{section}    % numeri delle figure del tipo x.y


\makeindex							% generate the index


\begin{document}
	\tikzset{
	%Define standard arrow tip
	>=stealth',node distance=1cm, auto,font=\tiny,
	%Define style for boxes
	itm/.style={
		rectangle,
		rounded corners,
		draw=black, very thick,
		minimum width=5em,
		minimum height=1.5em,
		text centered,
		align=center,
		inner sep=3pt,
	},
	% Define arrow style
	ar/.style={
		<-,
		thick,
		shorten <=2pt,
		shorten >=2pt,},
	dummyar/.style={
		thick,
		shorten <=0pt,
		shorten >=2pt,},
	dummy/.style={
		minimum width=10ex,
		minimum height=0,
		inner sep=0,
		outer sep=0,
	},
	dummy0/.style={
		minimum width=0,
		minimum height=0,
		inner sep=0,
		outer sep=0,
	},
	count/.style={
		draw,
		circle,
		inner sep=.3ex,
		outer sep=.3ex,
		thick,
		anchor=south,fill=blue!30!white!10
	},
}
\def \lastitem {}
\def \dummy #1{\node[dummy,right=of #1] (#1_dummy) {};\gdef\lastitem{#1_dummy}}
\def \dummyz #1{\node[dummy0,right=of #1] (#1_dummy) {}}
\def \dummyzar #1{\node[dummy0,right=of #1] (#1_dummy) {} edge[ar] (#1.east);\gdef\lastitem{#1_dummy}}
\def \dummystart #1#2{\node[dummy0, below=of #2] (start#1) {};\gdef\lastitem{start#1}}
\def \dummyend #1{\node[dummy0,right=of #1] (#1_end) {} edge[ar] (#1.east);;\gdef\lastitem{#1_end}}
\def \itm #1#2{\gdef\lastitem{#1};\node[itm,fill=blue!40!white!40] (#1) {#2}}

\def \itmright #1#2#3#4{\node[itm, right=of #1] (#2) {#3} edge[ar] (#4.east);\gdef\lastitem{#2}}
%\def \itmabove #1#2#3#4{\node[itm, above=of #1] (#2) {#3} edge[ar] (#4.east);\gdef\lastitem{#2}}
\def \itmabove #1#2#3#4{
	\node[itm, above=of #1] (#2) {#3}; 
	\draw[thick,shorten <=2pt] (#4.east) -- ++(.3,0) [shorten >=2pt,shorten <=0pt,->] to[out=0,in=225] (#2.south west);
	\gdef\lastitem{#2}
}
\def \itmbelow #1#2#3#4{
	\node[itm, below=of #1] (#2) {#3};
	\draw[thick,shorten >=2pt,->] ($(#4.east) + (.3,0)$)  to[out=0,in=155] (#2.north west);
	\gdef\lastitem{#2};
}

\def \itmrightnoarrow #1#2#3{\node[itm, right=of #1] (#2) {#3}}

\def \itmrights #1#2#3{\itmright{#1}{#2}{#3}{#1}}
\def \itmaboves #1#2#3{
		\node[itm, above=of #1] (#2) {#3};
		\draw[ar]($(#2.south) + (.5cm,0)$) to[bend left] ($(#1.north)+ (.5cm,0)$);
		\gdef\lastitem{#2}
}
\def \itmbelows #1#2#3{
	\node[itm, below=of #1] (#2) {#3};
	\draw[ar]($(#2.north) + (-.5cm,0)$) to[bend left] ($(#1.south)+ (-.5cm,0)$);
	\gdef\lastitem{#2}
}

\def \itmsplit #1#2#3#4#5{\dummy{#1};
	\itmabove{#1_dummy}{#2}{#3}{#1};
	\itmbelow{#1_dummy}{#4}{#5}{#1}}

\def \itmmerge #1#2#3#4#5{\dummyz{#1_dummy}
			edge[dummyar] (#2.east)
			edge[dummyar] (#3.east);
	\node[itm, right=of #1_dummy_dummy] (#4) {#5}
			edge[ar,shorten >=0pt] (#1_dummy_dummy.north)}

\def \itmsplitthree #1#2#3#4#5#6#7{
	\itmrights{#1}{#4}{#5};
	\itmabove{#4}{#2}{#3}{#1};
	\itmbelow{#4}{#6}{#7}{#1};
}

\def \itmsplitfour #1#2#3#4#5#6#7#8#9{
	\node[itm,above right=0em and 3em of #1] (#4) {#5};
	\node[itm,above right=3em and 3em of #1] (#2) {#3};
	\node[itm,below right=0em and 3em of #1] (#6) {#7};
	\node[itm,below right=3em and 3em of #1] (#8) {#9};
	
	\draw[thick,shorten <=2pt] (#1.east) -- ++(.3,0) [shorten >=2pt,shorten <=0pt,->] to[out=0,in=225] (#2.south west);
	\draw[thick,shorten >=2pt,->] ($(#1.east) + (.3,0)$)  to[out=0,in=205] (#4.south west);
	\draw[thick,shorten >=2pt,->] ($(#1.east) + (.3,0)$)  to[out=0,in=155] (#6.north west);
	\draw[thick,shorten >=2pt,->] ($(#1.east) + (.3,0)$)  to[out=0,in=135] (#8.north west);
}

\def \armergetwo #1#2#3{ 
	\draw[thick,shorten <=2pt] (#1.east) to[out=0,in=180] ($(#3.west) + (-.3cm,0)$);
	\draw[thick,shorten <=2pt,shorten >=2pt,->] (#2.east) to[out=0,in=180] ($(#3.west) + (-.3cm,0)$) -- (#3.west);
}

\def \armergethree #1#2#3{
	\draw[thick,shorten <=2pt] (#1.east) to[out=0,in=180] ($(#3.west) + (-.3cm,0)$);
	\draw[thick,shorten <=2pt] (#2.east) to[out=0,in=180] ($(#3.west) + (-.3cm,0)$);
}

\def \itmcount #1#2{\node[count] at (#2.north) {#1};}

\newcounter{itmcvalue}
\def \itmcinit #1{\setcounter{itmcvalue}{#1}}
\def \itmcnoinc {\addtocounter{itmcvalue}{-1};}
\def \itmc #1{\itmcount{\arabic{itmcvalue}}{#1};\addtocounter{itmcvalue}{1};}

\def\itmr #1#2{\itmrights{\lastitem}{#1}{#2}}

	\maketitle
	\begin{abstract}
	Questo articolo riassume con delle carte mnemoniche gli argomenti di farmacologia spiegati nel IV anno del corso di laurea in medicina e chirurgia a Chieti.
	L'uso di questo articolo non sostituisce la lettura e lo studio di un libro e degli appunti di farmacologia.

Per errori, omissioni o altre note, non esitate a contattarmi via e-mail.

Potete utilizzare direttamente il PDF compilato o ricrearlo compilando i sorgenti utilizzando il \LaTeX. 

Potete anche modificare, correggere e integrare il documento a patto di rilasciarlo con la stessa sua licenza.

Questo documento è rilasciato secondo la licenza Creative Commons CC-BY-NC-SA 2.0 IT (\url{https://creativecommons.org/licenses/by-nc-sa/2.0/it/})

Tu sei libero di:

Condividere — riprodurre, distribuire, comunicare al pubblico, esporre in pubblico, rappresentare, eseguire e recitare questo materiale con qualsiasi mezzo e formato 

Modificare — remixare, trasformare il materiale e basarti su di esso per le tue opere

Alle seguenti condizioni:

Attribuzione — Devi riconoscere una menzione di paternità adeguata, fornire un link alla licenza e indicare se sono state effettuate delle modifiche. Puoi fare ciò in qualsiasi maniera ragionevole possibile, ma non con modalità tali da suggerire che il licenziante avalli te o il tuo utilizzo del materiale.

NonCommerciale — Non puoi utilizzare il materiale per scopi commerciali.

StessaLicenza — Se trasformi il materiale o ti basi su di esso, devi distribuire i tuoi contributi con la stessa licenza del materiale originario.

Divieto di restrizioni aggiuntive — Non puoi applicare termini legali o misure tecnologiche che impongano ad altri soggetti dei vincoli giuridici su quanto la licenza consente loro di fare.
	
Documento originale e aggiornato su \url{https://github.com/EmilianoBruni/farmacologia_mnemonic_charts}
\end{abstract}

\newpage

\tableofcontents

\newpage\newpage
	\chapter{Farmacocinetica}

La curva dose/concentrazione di un farmaco si basa su una media del funzionamento del farmaco. Ma questa curva può variare di molto da individuo a individuo.

I due parametri principali che influenzano questa variabilità sono il \textbf{volume di distribuzione} e la \textbf{cleareance}.

\section{Volume di distribuzione}

\`E quel volume teorico che contiene quella determinata concentrazione di farmaco nel comparto in oggetto, spesso il flusso sanguigno

$$V =\frac{\text{quantità di farmaco nell'organismo}}{[F]}$$

ove $[F]$ può essere riferito a sangue, plasma, farmaco libero.

Come si vede dalla relazione $V$ è uno spazio virtuale in quanto si presuppone che la concentrazione misurata in $[F]$ sia omogenea su tutto il corpo.

Quindi farmaci con distribuzione prettamente ematica danno delle $V$ piccole e confrontabili con il valore reale del compartimento che, per il sangue, è $\simeq 3\ce{L}/70\ce{Kg}$.

Farmaci con distribuzione prettamente extravascolare avremo piccole concentrazioni nel sangue e quindi $V$ elevati. Ad esempio, per la digossina, $V \simeq 500\ce{L}/70\ce{Kg}$.

La $V$ è utile, ad esempio, nel calcolo dell'emivita come vedremo a breve.

\section{Clereance}

\`E la quantità di farmaco eliminata nel tempo in rapporto alla sua concentrazione nel comparto in oggetto (sangue, plasma, farmaco libero)

$$CL =\frac{\text{velocità di eliminazione}}{[F]}$$

La clearenace è additiva per cui se un farmaco ha eliminazione renale, epatica e respiratoria

$$ CL_{tot} = CL_{\text{rene}} + CL_{\text{epat.}} + CL_\text{resp.} = \frac{V_{\text{rene}}^{\text{elim}}}{[F]} + \frac{V_{\text{epat.}}^{\text{elim}}}{[F]} + \frac{V_{\text{resp}}^{\text{elim}}}{[F]}$$

Per la maggior parte dei farmaci la clereance è costante all'interno del range di concentrazioni della pratica terapeutica per cui la velocità di eliminazione dipende solo dalla concentrazione del farmaco

$$\ce{Vel.} = CL \cdot [F]$$

Quando ciò accade si parla di cinematica di ordine 1. In questo caso la clereance può essere calcolata misurando l'area sotto la curva (AUC) delle concentrazioni ematiche nel tempo dopo somministrazione di una singola dose di farmaco come

$$\text{CL} = \frac{\text{dose}}{\text{AUC}}$$

\section{Eq. di Michaelis--Menten}

Ma non tutti i farmaci seguono un andamento lineare. Alcuni farmaci (\index{fenitoina}fenitoina, \index{etanolo}etanolo, \index{acido acetilsalicilico}acido acetilsalicilico) hanno un'eliminazione saturabile a cinematica non lineare.

Questi seguono l'equazione di Michaelis--Menten

$$v_{\text{elim}} = \frac{v_{\text{max}}\cdot [F]}{K_m + [F]}$$

dove $K_m$ è chiamata costante di Michaelis--Menten e rappresenta la concentrazione del farmaco che produce una velocità di eliminazione del 50\% della massima.

A concentrazioni levate la $v_{\text{elim}}$ non dipende più dalla concentrazione e diventa costante con una cinetica di ordine 0. 

Per questi farmaci non si può parlare di clereance ne usare l'AUC per descrivere la loro eliminazion.

\section{Emivita}

L'emivita di un farmaco è definita come il tempo necessario a ridurre
il farmaco a \unitfrac{1}{2} della quantità di farmaco presente nell'organismo
allo steady-state.

Supponendo un modello monocorpartimentale, la variazione della concentrazione nel tempo è una funzione lineare della concentrazione stessa

$$\frac{d\,q}{d\,t} = -kq$$

che ha come soluzione una esponenziale decrescente con il tempo:

$$
q(t)=q_0 e^{-kt}
$$

con $q_0$ concentrazione iniziale. Per calcolare il parametro $k$ valutiamo la AUC di questo andamento

$\text{AUC} =\displaystyle\int_0^\infty q_0e^{-kt}\,dt = \frac{q_0}{k}$ ma d'altra parte $\text{AUC} = \displaystyle\frac{\text{dose}}{\text{CL}}$ ed essendo 

$\text{dose} = Vq_0$ si ha che $\displaystyle\frac{Vq_0}{\text{CL}} = \frac{q_0}{k} \Rightarrow k = \frac{\text{CL}}{V}$ e quindi

$$q(t) = q_0\,e^{-\frac{\text{CL}}{V}t}$$

e ponendo $q(t_{\unitfrac{1}{2}}) = \displaystyle\frac{q_0}{2}$ si ha che

$$t_{\unitfrac{1}{2}}=\ln\,2\cdot\dfrac{V}{\text{CL}}\simeq0.7\cdot\dfrac{V}{\text{CL}}$$

Identica curva, ma al contrario, per l'accumulo durante una somministrazione continua a velocità costante.

Così, dopo $t_{\unitfrac 12}$ avrò il 5\% di $q_0$ e dopo circa 4 emivite avrò più del 90\% del $q_0$.

Analogamente nell'eliminazione, dopo 4 emivite, avrò meno del 10\% del farmaco iniziale nel corpo.

Nel caso di somministrazioni multiple, se l'intervallo tra le dosi è minore di 4 emivite avrò un accumulo di farmaco e questo accumulo è inveramente proporzionale alla frazione di farmaco eliminata

$$\text{frazione di accumulo} = \frac{1}{\text{frazione di farmaco eliminato}} = \frac{1}{1-\text{frazione residua}}$$

cosi, ad esempio, un farmaco somministrato ogni emivita ha una frazione di accumulo pari a $\frac{1}{0.5} = 2$.

\section{Biodisponibilità}

\`E la frazione di farmaco non modificato che raggiunge la circolazione sistemica

\begin{tikzpicture}
	\Tree
	[.{vie di somministrazione}
		[.{endovenosa (IV)}
			{bio: 100\%}
		]
		[.{intramuscolare (IM)}
			{bio: $75\sim \leq 100$}
		]
		[.{sottocutanea (SC)}
			{bio: $75\sim \leq 100$}
		]
		[.{orale (PO)}
			{bio: $5\sim < 100$}
		]
		[.{rettale (PR)}
			{bio: $30\sim < 100$}
		]
		[.{inalatoria}
			{bio: $5\sim < 100$}
		]
		[.{transdermica}
			{bio: $80\sim \leq 100$}
		]															
	]
\end{tikzpicture}

\section{Effetto di primo passaggio}

A seguito dell'assorbimento per os il farmaco attraversa il fegato prima di andare nella circolazione sistemica e qui può perdere una frazione $\text{ER}$ di quanto assorbito $f$ con un valore di biodisponibilità $F$ pari a 

$$F=f(1-\text{ER})$$
dove $\text{ER} = \displaystyle\frac{\text{CL}_{\text{fegato}}}{\Phi^{\text{fegato}}_{\text{ematico}}}$ con $\Phi\simeq 90$L/h/70Kg.

Ad esempio, la morfina è quasi tutta assorbita nell'intestino per cui $f=1$. Tuttavia la $\text{CL}_{\text{fegato}}^{\text{morfina}} = 60$L/h/70Kg cosi che $\text{ER}=0.67$ da cui la biodisponibilità orale della morfina è $\simeq 0.33$.

\section{Dose di mantenimento}

I farmaci vengono somministrati per mantenere quanto più possibile uno stato stazionario ad un livello di concentrazione target $[F]_\text{target}$ andando a compensare l'eliminazione del farmaco. A regime quindi

$$v_\text{somm} = v_\text{elim} = \text{CL}\cdot[F]_\text{target}$$

e, se la via di somministrazione non è quella ideale ma, ad esempio, quella orale

$$v_\text{somm}^\text{orale} = \frac{v_\text{somm}}{[F]_\text{orale}}$$

Se invece di una somministrazione continua ho dosi ripetute, la dose di mantenimento è tale che 

$$\text{dose}_\text{mant} = v_\text{somm} \cdot \text{intervallo tra le dosi}$$

Se l'emivita del farmaco è lunga e quindi lo stato stazionario lo raggiungerei dopo molto tempo si può somministrare una dose di carico per raggiungere brevemente la concentrazione target come

$$\text{dose}_\text{carico} = V_\text{DIST} \cdot [F]_\text{target}$$




	\newpage\part{Flash Cards}

\section{Farmaci del SNC e del SNP}

\begin{tikzpicture}
	\tikzset{level distance=90pt,frontier/.style={distance from root=370pt}} 
	\Tree
	[.SNC
		[.Periferico
			[.Sensitivo ]
			[.Autonomo
		 		[ .{Simpatico\\ (toraco--addominale)} {gangli pre e para--vertebrali} ]
				[ .{Parasimpatico \\(nervi crani e sacrale)} {nell'intima degli organi} ]
			]
			[.Gastroenterico {plessi mioenterici (Auerbach) e\\ sottomucosi (Meissner)}
			]
		]
		[.Centrale ]
	]
\end{tikzpicture}

\begin{tikzpicture}
	\tikzset{level 1/.style={level distance=150pt}}
	\Tree
	[.{Neurotrasmettitori\\ SNP}
		 [.\node(acetilcolina){acetilcolina}; {recettori colinergici} ]
		 [.noradrenalina {recettori adrenergici} ]
		 [.\node(serotonina){serotonina\\5-HT 5-idrossitriptamina}; {recettori serotoninergici} ]
		 [.{monossido d'azoto (NO)} ] 
		 [.purine ]
	]
	\begin{scope}[yshift=-9em]
	\Tree
	[.\node(snc){Neurotrasmettitori\\ SNC};
		[.dopamina {recettori dopaminergici} ]
		Glutammato
		GABA
	]
	\end{scope}
	\draw[drawarrow] (snc.east) to[in=180,out=40] (acetilcolina.west)
		(snc.east) to[in=180,out=0] (serotonina.west);
\end{tikzpicture}

\subsection{Acetilcolina}

\begin{tikzpicture}
	\tikzset{level 1/.style={level distance=150pt}}
	\Tree
	[.localizzazione {tutte le fibre pregangliali sia para che orto\\ nicotiniche} {parasimpatiche post gangliali (quasi tutte).\\ muscariniche} {ghiandole sudoripare (simpatico)\\ muscariniche} {giunzione neuromuscolare\\ nicotiniche} ]
\end{tikzpicture}

\begin{tikzpicture}
	\tikzset{level 2/.style={level distance=130pt}, level 3/.style={level distance=120pt}}
	\Tree
	[.Sintesi
		[.colina \edge node[smallfont,yshift=5pt,xshift=5.4em]{entra nel neurone} node[smallfont,yshift=-5pt,xshift=5.4em]{tappa limitante};
			[.{acetilCOA + colina} \edge node[smallfont,yshift=-5pt,xshift=5em]{acetiltrasferasi} node[smallfont,yshift=5pt,xshift=5em]{colina}; acetilcolina ]
		]
	]
\end{tikzpicture}

\begin{tikzpicture}
	\tikzset{level 2/.style={level distance=130pt}}
	\Tree
	[.Degradazione
		[.acetilcolina \edge node[smallfont, yshift=-5pt,xshift=5.5em]{acetilcolinesterasi}; {acetato + colina} ]
	]
\end{tikzpicture}

\begin{tikzpicture}
	\Tree
	[.Liberazione 
		[.{Ca${}^{2+}$ + VAMP/SNAPS}
			[.{fusione vescicole con\\ membrana neuronale} esocitosi
			]
		]
	]
\end{tikzpicture}

\begin{tikzpicture}
	\tikzstyle{cwhite}=[circle,shadedraw=yellow];
	\shade[ball color=yellow] node (ach) {\small Ach} circle[radius=.45];
	\draw (ach) -- +(.7,.7) +(.7,.7) arc [start angle=225, end angle=270, radius=6pt]  (.7,.7) arc [start angle=225, end angle=180, radius=6pt];
	\draw (2,2) arc [start angle=45, end angle=0, radius=3cm];
	\draw (2,2) arc [start angle=45, end angle=80, radius=3cm];
	\draw (2,2) -- (1.3,1.3);
	\filldraw (1.3,1.3) circle[radius=3pt];
	\node at (1,.4) {\tiny VAMP};
	\node at (1.9,1.3) {\tiny SNAP};
	\node at (2.1,2.7) {\tiny Canale Ca${}^{2+}$};
\end{tikzpicture}

\begin{tikzpicture}
	\Tree
	[.Effetti
		[.{$\uparrow$permeabilit\`a ai cationi\\ Na${}^+$, K${}^+$, Ca${}^{2+}$}
			[.depolarizzazione
				[.{fibre post sinaptiche} PdA ]
				[.{fibre muscolari} {generazione potenziale\\ di placca}
				]
			]
		]
	]
\end{tikzpicture}

\begin{tikzpicture}
	\tikzset{level 1/.style={level distance=80pt},level 2/.style={level distance=140pt},level 3/.style={level distance=140pt}}
	\Tree
	[.{Recettore colinergico}
		[.{muscarinico\\ (metabotropo)}
			[.{M${}_1$ eccitatorio $\uparrow\text{IP}_3, \uparrow\text{DAG},\uparrow\text{Ca}^{2+}$} {SNC, simpatico post--gangliare,\\ cellule parietali dello stomaco}
			]
			[.{M${}_2$ inibitorio $\downarrow$cAMP } {cuore, muscolo liscio}
			]
			[.{M${}_3$ eccitatorio $\uparrow\text{IP}_3, \uparrow\text{DAG},\uparrow\text{Ca}^{2+}$} {ghiandole esocrine, muscolo liscio,\\ endotelio dei vasi}
			]
			[.{M${}_4$ come $\text{M}_2$} {SNC}
			]
			[.{M${}_5$  come $\text{M}_1$} {endotelio vasale, cervello, SNC}
			]
		]
		[.{nicotinico\\ (ionotropo)}
			[.{N${}_\text{N}$ gangliare} {para e ortosimpatico gangliare} ]
			[.{N${}_\text{M}$ muscolare} {giunzione\\ neuromuscolare} ]
		]
	]
\end{tikzpicture}

\subsubsection{Agonisti colinergici}

\begin{tikzpicture}
	\tikzset{
		level distance=80pt,
		level 1/.style={level distance=70pt},
		level 2/.style={level distance=70pt},
		frontier/.style={distance from root=400pt} 
	}
	\Tree
	[.\node(main){Tipo}; 
		[.diretti
			[.{attivano recettori} 
				[.{esteri della colina} 
					\node(ach){acetilcolina };
					metacolina
					carbacolo
					\node[farmaco](beta){betanecolo};
				]
				[.alcaloidi
					[.muscarinici
						muscarina
						\node[farmaco]{pilocarpina};
					]
					[.nitotinici \node[farmaco]{nicotina}; ]				
				]
			]
		]
		[.indiretti
			[.\node(AchEI){inibitori AchE}; 
				\node[farmaco](neo){neostigmina};
				\node[farmaco]{edrifonio}; 
				[.{organofosfati} \node[farmaco]{ecotiopato}; ]
				\node(somar){somar (gas nervino)};  
			]
		]
	]	
	\node[right=5pt of somar] (somary) {};
	\node[above=6em of somary] (neox) {};
	\draw[drawarrow] (neox) -- (somary) node[midway,above,sloped] {\tiny $\uparrow$durata azione};
	\node[right=5pt of ach] (achn) {};
	\node[right=5pt of beta] (betan) {};
	\draw[drawarrow] (achn) -- (betan) node[midway,above,sloped] {\tiny $\uparrow$resist. idrolisi e quindi durata azione};
\end{tikzpicture}

\begin{tikzpicture}
	\Tree
	[.{Inibitori AchE}
		[.{alcool+gruppo N quaternario}
			[.\node[farmaco]{edrofonio}; 
				[.{legame idrogeno\\ o ionico con AchE} {idrolisi in minuti}
				]
			]
		]
		[.carbammati 
			[.\node[farmaco]{neostigmina\\ fisostigmina}; 
				[.{legame covalente con AchE} {idrolisi in ore}
				]
			]
		]
		[.organofosfati
			[.somar
				[.\node(fAchE){fosforilazione AchE}; \node(idro){idrolisi in giorni};
				]
			]
			\node[farmaco](ecotiopato){ecotipato};
		]
	]		
	\draw[drawarrow] (ecotiopato) to[out=0,in=180] (fAchE);
	\node[chartnode,below=1em of fAchE] (invec) {invecchiamento\\rottura legame O-P\\ con raffozamento\\ legame con AchE}; 
	\node[chartnode,below=1em of idro] (pral) {pralidossina pu\`o\\ scindere la\\ fosforilazione};
	\draw[drawarrow] (pral)--(fAchE) node[midway,above,sloped] {\tiny qui si};
	\draw[drawarrow] (pral)--(invec) node[midway,above,sloped] {\tiny qui no};
\end{tikzpicture}

\begin{tikzpicture}
	\tikzset{level distance=120pt, level 2/.style={level distance=150pt}}
	\Tree
	[.\node(sub){Effetti}; 
		[.occhio 
			{contrazione muscolo sfintere dell'iride (miosi)\\ contrazione muscolo ciliare (accomodamento da vicino)}
		]
		[.cuore
			{$\downarrow$frequenza (cronotopo-), $\downarrow$forza (inotropo-),\\ $\downarrow$vel. conduzione (dromotropo-), $\uparrow$periodo refrattario, NAV}
		]
		[.vasi
			{dilatazione a basse dosi o contrazione a alte dosi}
		]
		[.polmone
			 {broncocostrizione, $\uparrow$secrezione}
		]
		[.intestino
			{$\uparrow$motilit\`a, $\downarrow$ muscolatura sfinteri, $\uparrow$ secrezioni} 
		]
		[.vescica
			 {contrazione destrusore, rilascio trigono}
		]
		[.{ghiandole sudoripare\\ lacrimali\\ salivali} 
			{$\uparrow$secrezioni}
		]
		[.{giunzione neuromuscolare\\ (indiretti)}
			{basse concentrazioni: $\uparrow$forza contrazione utile \\ se intossicazioni da curaro o miastenia grave\\ alte concentrazioni: fibrillazione fibre muscolari}
		]
	]
\end{tikzpicture}

\begin{tikzpicture}
	\tikzset{level distance=130pt, level 2/.style={level distance=160pt},frontier/.style={distance from root=300pt}}
	\Tree
	[.{usi clinici\\ agonisti colinergici}
		[.occhio {glaucoma con diminuzione della pressione\\ nelle emergenze ad angolo chiuso,\\ agonista muscarinico (pilocarpina)\\ + inibitore colinesterasi (ecotiopato)\\ 
		nel glaucoma cronico ora si usano i $\beta$-bloccanti.} ]
		[.{gastrointestinale\\ urinari} {Tutti i casi di depressione dell'attivit\`a senza ostruzione\\ neostigmina/betanecolo nelle ritenzioni urinarie e ileo\\
		pilocarpina usata per $\uparrow$secrezioni salivari in xerostomia da\\
		sindrome di Sjogren.} ]
		{intossicazione da\\ farmaci antimuscarinici\\ (intossicazione da atropina)\\ fisostigmina}
		[.cuore {tachiaritmia parossistica sopraventricolare\\ edrofonio(in disuso)\\ ora si usa l'adenosina.} ]
		[.{giunzione neuromuscolare} 
			[.{miastenia grave\\ edofonio/neostigmina} ]
			[.{post anestesia\\ ??? Leggo nelle correzioni\\ NO negli inibitori AchE ma boh!! } ]
		]
	]
\end{tikzpicture}

\subsubsection{Antagonisti colinergici}

\begin{tikzpicture}
	\Tree
	[.{antagonisti\\ colinergici}
		[.antimuscarinici \node[farmaco]{atropina}; \node[farmaco]{scopolamina};
		]
		[.{ganglioplegico\\ antinicotinico} \node[farmaco]{tetraetilammonio (TEA)}; \node[farmaco]{esametanio (C6)};
		]
		[.{rigeneratori dell'AchE} \node[farmaco]{pralidossima\footnotemark};
		]
	]
\end{tikzpicture}

\footnotetext{vedi inibitori dell'AchE}

\begin{tikzpicture}
	\tikzset{level 2/.style={level distance=160pt}}
	\Tree
	[.{anti-muscarinici\\ effetti}
		[.SNC {effetto stimolante (-atropina +scopolamina) $\downarrow$ tremore parkinson}
		]
		[.occhio {$\uparrow$attività simpatica $\Rightarrow$ midriasi\\ (belladonna $\equiv$ occhi dilatati)}
		]
		[.cuore {tachicardia, blocco vagale, $\downarrow$PR per $\uparrow$dromotropo}
		]
		[.vasi incerta ]
		[.{apparato respiratorio} {broncodilatazione, $\downarrow$secrezioni\\ (ma meglio i $\beta$-adrenergici)}
		]
		[.{gastrointestinale} {$\downarrow$secrezioni salivali, minori su tutto il resto} ]
		[.{gh. sudoripare} {soppressione termoregolazione\\ (febbre da atropina)} ]
	]
\end{tikzpicture}

\begin{tikzpicture}
	\tikzset{level 1/.style={level distance=130pt}}
	\Tree
	[.{anti-muscarinici\\ usi clinici} 
		{malattia di Parkinson} 
		{chinetosi} 
		{occhio} 
		{apparato respiratorio} 
		{apparato cardiovascolare} 
		{apparato gastrointestinale} 
		{apparato urinario}
	]
\end{tikzpicture}

\begin{tikzpicture}
	\Tree
	[.{rigeneratori\\ dell'AchE}
	]
\end{tikzpicture}

\subsection{Noradrenalina}

\begin{tikzpicture}
	\Tree
	[.noradrenalina 
		[.{simpatiche postgangliari} 
			[.escluso
				{muscolatura vasi renali ($\text D_1$)}
				{ghiandole sudoripare (Ach)}
			]
		]		
	]
\end{tikzpicture}

\begin{tikzpicture}
	\begin{scope}
	\tikzset{level distance=90pt,level 3/.style={level distance=150pt},
	level 2/.style={level distance=130pt},level 4/.style={level distance=60pt}}
	\Tree 
	[.Sintesi 
		[.Tirosina  \edge node[smallfont,yshift=-5pt,xshift=5.5em]{tirosin--idrossilasi} node[smallfont,yshift=5pt,xshift=5.5em]{tappa limitante}; 
			[.{L-Dopa} \edge node[smallfont,yshift=-5pt,xshift=6em]{DOPA decarbossilasi};
				[.\node[farmaco]{dopamina}; \node[dummyc]{};]
			]
		]
	]
	\end{scope}
	\begin{scope}[yshift=-3em,xshift=1em]
	\tikzset{level distance=80pt}
	\Tree
	[.\node[dummyc]{}; 
		[.\node[farmaco]{noradrenalina}; \node[farmaco]{adrenalina};]
	]	
	\end{scope}				
	
\end{tikzpicture}

\begin{tikzpicture}
	\tikzset{level distance=130pt}
	\Tree 
	[.Degradazione
		[.{MAO (mono-ammino ossidasi)\\ in fegato e cellule ?????} ]
		[.{COMT (catecolo O-metiltransferasi)\\ nei neuroni} ]
	]
\end{tikzpicture}

\begin{tikzpicture}
	\tikzset{level 2/.style={level distance=150pt}}
	\Tree	
	[.{Recettore adrenergico\\ (metabotropo \\ a proteine G)} 
		[.{$\alpha$}
			[ .{$\alpha_1\,\text{G}_{\text{q}} \uparrow\text{IP}_3, \uparrow\text{Ca}^{2+}$ (postsinaptiche muscolo liscio)} ]
			[ .{$\alpha_2\,\text{G}_{\text{i}} \downarrow\text{cAMP}$ (presinaptiche muscolo liscio)} ]
		]
		[.{$\beta$}
			[.{$\beta_1\,\text{G}_{\text{s}} \uparrow\text{cAMP}$ (postsinaptiche cuore, adipociti,\\ iuxaglomerulare, epitelio corpi ciliari)} ]
			[.{$\beta_2\,\text{G}_{\text{s}} \uparrow\text{cAMP}$ (postsinaptiche muscolo liscio e cuore)} ]
			[.{$\beta_3\,\text{G}_{\text{s}} \uparrow\text{cAMP}$ (postsinaptiche cuore, adipociti, vescica)} ]		
		]
	]
\end{tikzpicture}

\begin{tabular}{|c|c|c|c|}
\hline 
\textbf{Organo} & \textbf{Tipo} & \textbf{Recettore} & \textbf{Azione} \\ 
\hline\hline 
M. radiale & simpatico & $\alpha_1$ & costrizione \\ 
\hline 
M. circolare & parasimpatico & M${}_3$ & costrizione pupilla \\ 
\hline 
M. ciliare & simpatico & $\beta$ & rilasciamento \\ 
\hline 
M. ciliare & parasimpatico & M${}_2$ & contrazione \\ 
\hline 
Nodo SA & simpatico & $\beta_1\beta_2$ & accellerazione \\ 
\hline 
Nodo SA & parasimpatico & M${}_2$ & rallentamento \\ 
\hline 
Forza contrazione & simpatico & $\beta_1\beta_2$ & aumento \\ 
\hline 
Forza contrazione & parasimpatico & M${}_2$ & diminuzione \\ 
\hline 
vasi muscolari & simpatico & $\beta$ & rilasciamento \\ 
\hline 
muscolo gastrointestinale & simpatico & $\alpha_2\beta_2$ & rilasciamento \\ 
\hline 
muscolo gastrointestinale & parasimpatico & M${}_3$ & contrazione \\ 
\hline 
sfinteri gastrointestinali & simpatico & $\alpha_1$ & contrazione \\ 
\hline 
sfinteri gastrointestinali & parasimpatico & M${}_3$ & rilasciamento \\ 
\hline 
\end{tabular} 

\subsection{Serotonina}

Serotonina o 5-HT o 5-idrossitriptamina

\begin{tikzpicture}
	\tikzset{level 3/.style={level distance=130pt}}
	\Tree
 	[.Recettore
 		[.5-HT1
 			[.SNC {GpCR,$\downarrow$cAMP, inibizione presinaptica} ]
 		]
 		[.5-HT2
 			[.{muscolo liscio\\ piastrine} {$\uparrow$IP3, DAG, GpCR} ]
 		]
 		[.5-HT3
 			[.{SNP (nocicettori\\ neuroni enterici)} {canale ionico stimolatore} ]
 		]
 		[.5-HT4
 			[.{SNC,vescica\\ cuore} {$\uparrow$cAMP,GpCR,eccitazione} ]
 		]
 	]
\end{tikzpicture}

\begin{tikzpicture}
	\tikzset{level distance=80pt,level 3/.style={level distance=130pt},
	level 2/.style={level distance=130pt}}
	\Tree 
	[.Sintesi 
		[.Triptofano  \edge node[smallfont,yshift=5pt,xshift=5.5em]{triptofano} node[smallfont,yshift=-5pt,xshift=5.5em]{idrossilasi} ; 
			[.{5-idrossitriptofano} \edge node[smallfont,yshift=5pt,xshift=5.5em]{amminoacido} node[smallfont,yshift=-5pt,xshift=5.5em]{decarbossilasi}; 5-HT 
			]
		]
	]	
\end{tikzpicture}

\begin{tikzpicture}
	\tikzset{level distance=130pt}
	\Tree 
	[.Degradazione {MAO (mono-ammino ossidasi)}
	]
\end{tikzpicture}

\begin{tikzpicture}
	\tikzset{level distance=160pt}
	\Tree
	[.Effetti {piastrine: aggregazione} {terminazioni nervose: dolore} {SNC: eccitatorio 5-HT4,\\ inibitorio 5-HT1} {vasi: costrizione} {gastroenterico: attivazione secrezione\\ e peristalsi} ]
\end{tikzpicture}

\subsection{Neurotrasmettitori purinici}

\begin{tikzpicture}
	\tikzset{level 2/.style={level distance=130pt}}
 	\Tree
 	[.{Neurotrasmettitori\\ purinici} 
 		[.ATP {Aumento della permeabilit\`a di membrana} ]
 		[.Adenosina {vasodilatatore tranne che nel rene} {inibizione dell'aggreg. piastrinica} {blocco della conduzione AV}
 		]
 	]
\end{tikzpicture}

\subsection{Monossido d'azoto (NO)}

\begin{tikzpicture}
	\Tree
	[.Tipi
		[.iNOS {prodotto dai macrofagi tramite IF$\gamma$}
		]
		[.eNOS {endotelio e piastrine}
		]
		[.nNOS neuroni
		]
	]
\end{tikzpicture}

\begin{tikzpicture}
	\tikzset{level distance=160pt}
	\Tree
	[.Causa {vasodilatazione} {inibizione dell'aggregazione piastrinica} {plasticit\`a sinaptica} {difesa da cellule neoplastiche, batteri, parassiti} ]
\end{tikzpicture}

Per via inalatoria $\downarrow$shunt, $\downarrow$broncocostrizione, $\downarrow$ipertensione polmonare e quindi utile anche nella cura dell'asma.

Utile nel trattamento delle malattie neurovegetative e shock settico dove aumenta e nell'ateorscelosi e ipercolesterolemia dove diminuisce. 

\subsection{Dopamina}

\begin{tikzpicture}
	\tikzset{level distance=130pt}
	\Tree
	[.{Recettore dopaminergico}
		[.{$\text D_1$, $\text D_5$, eccitatorio, $\uparrow$cAMP} {cervello, muscolatura vasi rene} 
		]
		[.{$\text D_2$, inibitorio, apertura canali $\text K^+$} {cervello, muscolatura liscia} 
		]
		[.{$\text D_3$, inibitorio, apertura canali $\text K^+$} {cervello} 
		]
		[.{$\text D_4$, inibitorio, apertura canali $\text K^+$} {cervello, sistema cardio vascolare} 
		]
	]
\end{tikzpicture}

\newpage
	\section{Farmaci del sistema cardiovascolare e renale}

\subsection{Farmaci anti--ipertensivi}

\begin{tikzpicture}
	\Tree
	[.Anti-ipertensivi diuretici simpaticolitici vasodilatatori ]
\end{tikzpicture}

\begin{tikzpicture}
	\Tree
	[.{Diuretici\\ (capitolo ah hoc)}
		[.{Diuretici dell'ansa} \node[farmaco]{\index{furosemide}}; ]
		[.{Inibitori del simporto\\ \ce{Na+}-\ce{Cl-}} \node[farmaco]{tiazidici}; ]
		[. {Risparmiatori di \ce{K+}} \node[farmaco]{\index{spironolattone}}; ]
	]
\end{tikzpicture}

\begin{tikzpicture}
	\tikzset{level 3/.style={level distance=120pt}}
	\Tree
	[.Simpaticolitici
		[.{SNC}
			[.\node[farmaco]{\index{$\alpha$-metildopa}}; {Inibitore dopa-carbossilasi\\ emergenza ipertensiva \\ Da sedazione, tossicit\`a epatica\\ coombs positivo} ]
			[.\node[farmaco]{\index{clonidina}}; {Agonista $\alpha_2$. $\downarrow$noradrenalina\\ Usato in gravidanza \\ Da sonnolenza, depressione\\ $\downarrow$libido, secchezza fauci } ]
		]		
		[.{$\beta$--bloccanti}
			[.\node[farmaco]{\index{propranololo}}; {Usato in ipertensione, scompenso cardiaco, \\ aritmie, glaucoma. Produce $\downarrow$GC e renina. \\ Da affaticamento,$\downarrow$umore, insomnia, $\uparrow$glicemia, \\ alterazione assetto lipidico (i non ASI). \\ Interruzione improvvisa $\uparrow$infarto.} ]
		]		
		[.{$\alpha$--agonisti} \node[farmaco]{\index{doxazosina}}; ]
		[.{Misti $\alpha$/$\beta$}
			[.\node[farmaco]{\index{labetalolo}}; {ipertensione da feocromocitoma.\\ Da prurito intenso, $\downarrow$eiaculazione} ]
		]
	 ]
\end{tikzpicture}

\begin{tikzpicture}
	\tikzset{level distance=80pt, level 4/.style={level distance=100pt}}
	\Tree
	[.{Vasodilatatori}
		[.{diretti}
			[.{prevalentemente\\ arteriosi}
				[.{Inibitori IP3} \node[farmaco]{\index{idralazina}\\ (non pi\`u usato)}; ]
				[.{\ce{Ca^{2+}} antagonisti}  \node[farmaco]{\index{nifedipina}\footnotemark\\ (anche verapamil\\ e diltiazem\\ ma su cuore)}; ]
			]
			[.{arterovenosi} 
				[.{rilascio \ce{NO}} \node[farmaco]{\index{nitroprussiato}\footnotemark\\ nitroglicerina}; ]
			]
		]
		[.{indiretti}
			[.{ACE inibitori}
				[.\node[farmaco]{\index{captopril}\\ \index{enalapril}\\ \index{fosinopril}}; {Dilata arteriole e grandi vene. \\$\downarrow$pre/post carico. \\ Non inficia riflesso barocettivo\\ ne secrezione di aldosterone. \\ $\uparrow$bradichinina da tosse secca\\ e edema angioneurotico.} ]
			]
			[.{Antagonisti AT--1}
				[.{sartani} {Uso in ipertensione, ACC, \\ nefropatia diabetica\\ NO in gravidanza} ]
			]
		]
	]
\end{tikzpicture}

\footnotetext{Vedere farmaci angina}

\footnotetext{Vedere farmaci angina}

\newpage

\subsection{Farmaci nell'angina e infarto cardiaco}

\begin{tikzpicture}
	\Tree
	[.{angina\\ infarto} vasodilatatori simpaticomimetici ]
\end{tikzpicture}

\begin{tikzpicture}
	\tikzset{level distance=90pt, level 3/.style={level distance=130pt}}
	\Tree
	[.{Vasodilatatori}
		[.Nitrati
			[.\node[farmaco]{\index{Isosorbide mononitrato}}; {Duranta d'azione pi\`u lunga} ]
			[.\node[farmaco]{\index{Nitroglicerina}}; {Rilascio \ce{NO}, $\uparrow$cGMP, relax muscolatura lis.\\ Via sublinguale, transdermica, rapido assorbimento\\ grazie alla solubilit\`a lipidica}  ]	
		]
		[.{\ce{Ca^{2+}}  antagonisti}
			[.\node[farmaco]{\index{verapamil}\\ (\index{diidropiridine})}; {$\downarrow$conduzione NSA. $\downarrow$ RVP} ]
			[.\node[farmaco]{\index{diltiazem}}; {$\downarrow$conduzione NSA. $\downarrow$ RVP} ]
			[.\node[farmaco]{\index{nifedipina}}; {$\updownarrow$conduzione NSA.  Possibile tachicardia riflessa\\ minori effetti cardiaci} ]
		]
	]
\end{tikzpicture}

\begin{tikzpicture}
	\tikzset{level distance=90pt, level 3/.style={level distance=130pt}}
	\Tree
	[.{Simpaticolitici}
		[.{$\beta$--bloccanti}
			[.\node[farmaco]{\index{propranololo}\footnotemark}; {$\downarrow$GC, $\downarrow$PA, $\downarrow$consumo \ce{O2} micardico} ]
		]
	]
\end{tikzpicture}

\footnotetext{vedi farmaci anti-ipertensivi}

\begin{tikzpicture}
	\node[chartnode,anchor=west] at(0,0)(mlck){MLCK} node[chartnode,xshift=125pt] (mlckstar){MLCK${}^*$};
	\draw[drawarrow](mlck)[yshift=10pt]--node[smallfont,yshift=6pt,midway](Ca){\ce{Ca^{2+}}}
				node[chartnode,yshift=50pt,midway](CCa){Canali \ce{Ca^{2+}}}
				node[smallfont,yshift=-15pt,midway](camp){cAMP}
				node[chartnode,yshift=-60pt,midway](atp){ATP}(mlckstar);
	\draw[drawarrow] (mlckstar) [yshift=-10pt] -- (mlck);
	\draw[drawarrow] (CCa)-- node[midway](CCCa){} node[smallfont,xshift=5em](bloc){bloccanti canali} (Ca);
	\draw[drawarrow] (bloc)-- node[midway, below]{$\ominus$} (CCCa);
	\draw[drawarrow] (atp)-- node[midway](catp){} node[smallfont,xshift=5em](beta){$\beta$-bloccanti} (camp);
	\draw[drawarrow] (beta)-- node[midway, above]{$\oplus$} (catp);
	
	\node[chartnode,below right=1em and 2em of mlckstar](mlc){MLC};
	\node[chartnode,right=50pt of mlc](mlcstar){MLC${}^*$} node[right=3pt of mlcstar](+){+}
		node[chartnode, right=3pt of +](actina){actina};
	\node[chartnode,above=20pt of actina](contrazione){contrazione};
	\node[chartnode,below=20pt of mlc](relax){relax};
	\draw[drawarrow] (mlc) [yshift=10pt]-- node[midway](a){}(mlcstar);
	\draw[drawarrow] (mlcstar) [yshift=-10pt]-- 
		node[smallfont,midway,yshift=-6pt](cgmp){cGMP} 
		node[chartnode,yshift=-57pt,midway](gtp){GTP}
		(mlc);
	\draw[drawarrow] (actina) -- (contrazione);
	\draw[drawarrow] (mlc) -- (relax);
	\draw[drawarrow] (mlckstar) -| (a);
	\draw[drawarrow] (gtp)-- node[midway](cgtp){} 
		node[smallfont,xshift=10pt](gcstar){GC${}^*$}
		node[chartnode,xshift=100pt](gc){Guanil ciclasi} 
		(cgmp);
	\draw[drawarrow] (gc)-- node[midway](cgc){} node[midway, chartnode,yshift=-50pt](no){NO} (gcstar);
	\draw[drawarrow] (no)--node[midway, right]{$\oplus$}(cgc);
	
	\node[smallfont,text width=12em,anchor=west] at(0,-4) {* $\equiv$ elemento attivato\\
	MLCK $\equiv$ Miosina Catena Leggera chinasi\\ MLC $\equiv$ Miosina Catena Leggera};
\end{tikzpicture}

\begin{tikzpicture}
	\Tree
	[.Angina 
		[.{ischemia cardiaca transitoria\\ senza danno al miocardio}
			{stabile}
			{instabile}
			{di prinzmetal}
			{silente}
			{cronica}
		]
	]		
\end{tikzpicture}

\begin{tikzpicture}
	\Tree
	[.Terapia
		comportamentale
		chirurgica
		[.farmacologica
			[.{vasodilatatori\\ \ce{NO} e \ce{Ca^{2+}} antagonisti}
				{per aumentare il flusso}
			]
			[.antiaggreganti {per evitare i trombi}
			]
			[.fibrinolitici {per distruggere i\\ trombi preesistenti}
			]
			[.{$\beta$--bloccanti} {per ridurre il fabbisogno energetico}
			]
			[.oppioidi {per ridurre il dolore}
			]
		]
	]
\end{tikzpicture}

\begin{tikzpicture}
	\Tree
	[.terapia
		[.stabile
			{nitrati organici}
			{$\beta$--bloccanti}
			{statina}
			{aspirina}
		]
		[.instabile
			{nitrati}
			{aspirina}
			{eparina}
		]
		[.variante
			{nitrati organici}
			{\ce{Ca^{2+}} antagonisti}
		]
	]	
\end{tikzpicture}

\subsubsection{Nitrati organici}

\begin{tikzpicture}
	\Tree
	[.{nitrati organici}
		{nitroglicerina}
		{isosorbide mononitrato}
	]
\end{tikzpicture}

\begin{tikzpicture}
	\Tree
	[.effetti
		[.{diminuzione della richiesta\\ di O${}_2$}
			{$\downarrow$ritorno venoso}
			{$\downarrow$volume intracardiaco}
			{$\downarrow$pressione arteriosa}
		]
		[.{scompasa spasmo arterioso} {vasodilatazione arterie\\ coronariche}
		]
	]
\end{tikzpicture}

\begin{tikzpicture}
	\Tree
	[.{effetti collaterali}
		{tachicardia riflessa}
		{aumento riflesso contrattile}
		{riduzione del tempo di perfusione\\ diastolica indotta da tachicardia}
	]
\end{tikzpicture}

\subsubsection{Calcio antagonisti}

\begin{tikzpicture}
	\Tree
	[.tipo
		[.L
			[.{Corrente lunga}
				[.Verapamil
					{cuore}
					{muscolo scheletrico\\ e liscio}
					{neuroni}
					{ossa}
				]
			]
		]
		[.T
			[.{Corrente breve}
				[.Flunarizina
					{cuore}
					{neuroni}
				]
			]
		]
		[.N
			[.{Corrente breve} {neuroni} 
			]
		]
		[.P
			[.{Corrente lunga} {neuroni} 
			]
		]
		[.{Q/R}
			[.{Segnapassi} {neuroni} 
			]
		]
	]
\end{tikzpicture}

\begin{tikzpicture}
	\tikzset{level 2/.style={level distance=120pt}}
	\tikzset{frontier/.style={distance from root=350pt}}
	\Tree
	[.effetti
		[.{muscolo liscio}
			[.{arteriole + sensibili delle venule.\\ Quindi meno effetto di ipotensione ortostatica}
				\node[farmaco]{\index{nifedipina}};
			]
		]
		[.{miocardio}
			\node[farmaco]{\index{varapamil}\\ \index{diltiazem}};
		]
		[.{muscolo scheletrico}
			[.{nessun effetto\\ il \ce{Ca^{2+}} \`e intracellulare} ]
		]
	]
\end{tikzpicture}

\subsubsection{$\beta$--bloccanti}

\begin{tikzpicture}
	\Tree
	[.effetti
		{$\downarrow$frequenza cardiaca}
		{$\downarrow$pressione arteriosa}
		{$\downarrow$contrattilit\`a}
	]
\end{tikzpicture}

\begin{tikzpicture}
	\Tree
	[.{effetti indesiderati}
		{$\uparrow$volume telediastolico}
		{$\uparrow$tempo di eiezione}
		{insomnia}
		{sonni spiacevoli}
		{senso di affaticamento}
		{disfunzione erettile}
	]
\end{tikzpicture}

\begin{tikzpicture}
	\Tree
	[.controindicazioni
		asma
		{affezioni broncospastiche}
		{grave bradicardia}
		{blocco atriventricolare}
		{insufficienza ventricolare sinistra}
	]
\end{tikzpicture}

\subsection{Insufficienza cardiaca}

\begin{tikzpicture}
	\Tree
	[.{I.C.}
		[.{gittata insufficiente\\ a fornire \ce{O2}\\ a organismo}
			[.{i. sistolica}
				{$\downarrow$contrattilità}
				{$\downarrow$fraz. di eiezione}
			]
			[.{i. diastolica}
				{rigidità}
				{perdità di rilasciamento}
			]
		]
	]
\end{tikzpicture}

\begin{tikzpicture}
	\Tree
	[.{scopo del\\ trattamento}
		[.{fase stabile\\(cronica)}
			{$\downarrow$sintomi}
			{rallentare progressione}
		]
		[.{fase scompensata\\(acuta)}
			{ricondurre il paziente\\ alla fare stabile}
		]
	]
\end{tikzpicture}

\begin{tikzpicture}
	\tikzset{level 2/.style={level distance=150pt}}
	\Tree
	[.terapia
		[.{fase cronica}
			{antagonisti aldosterone}
			{ACE inibitori}
			{sartani}
			{$\beta$--bloccanti}
			{digitalici}
			\node(diuretici){diuretici};
			\node(vasodilatatori){vasodilatatori};
		]
		[.\node(acuta){fase acuta};
			{$\beta$--agonisti}
		]
	]
	\draw[drawarrow] (acuta) to[out=0,in=180] (diuretici);
	\draw[drawarrow] (acuta) to[out=0,in=180] (vasodilatatori);
\end{tikzpicture}

\begin{tikzpicture}
	\tikzset{level distance=80pt}
	\Tree
	[.\node(git){$\downarrow$gittata cardiaca};
		[.{$\downarrow$P.A.}
			[.{attivazione barocettori}
				[.\node(simpatico){$\uparrow$simpatico};
					[.{inotropo+} {rimodellamento}
					]
					[.{cronotopo+}
					]
				]
			]
		]
		[.{$\downarrow$flusso renale}
			[.{$\uparrow$renina}
				[.{$\uparrow$angII}
					\node(pre){$\uparrow$ pre--carico};
					[.\node(post){$\uparrow$post--carico};
						\node(fraz){$\downarrow$fraz. eiezione};
					]
				]
			]
		]
	]
	\draw[drawarrow] (simpatico) to[out=0,in=180] (post);
	\draw[drawarrow] (simpatico) to[out=0,in=180] (pre);
	\node[below=1em of fraz](a){};
	\draw[drawarrow] (fraz) --  (a.north) -| (git);
\end{tikzpicture}

\begin{tikzpicture}
	\tikzset{level distance=120pt}
	\Tree
	[.{rimodellamento causato da\\ ipertrofia per riattivazione\\ fattori di crescita}
		[.concentrico
			{da sovraccarico pressorio per \upa post--carico}
		]
		[.eccentrico
			{da sovraccarico volume per \upa pre--carico}
		]
		[.compensato {se raggio della cavità ventricolare,\\ massa ventricolo e volume cavità\\ sono rispettati}
		]
		[.scompensato 
			[.{se tali rapporti non sono rispettati}
				{evolve in\\ scompenso cardiaco}
			]
		]
	]
\end{tikzpicture}

\begin{tikzpicture}
	\tikzset{level distance=80pt}
	\Tree
	[.{funzionalità\\ cardiaca}
		[.pre--carico
			[.{pressione riempimento\\ ventricolo sx}
				[.{$\uparrow$I.C.}
					[.vasodilatatori	
						{nitrati organici}
					]						
				]
			]
		]
		[.post--carico
			[.{resistenze vasc.\\ sistemiche e\\ impedenza aortica}
				[.{$\uparrow$I.C.}
					[.{farmaci $\downarrow$tono\\ arteriolare}	
						{???}
					]						
				]
			]
		]
		[.contrattilità
			[.{$\downarrow$I.C.}
				[.{farmaci $\uparrow$inotropismo}	
					{???}
				]						
			]
		]
		[.frequenza 
			{$\uparrow$I.C. per compensazione}
		]
	]
\end{tikzpicture}

\begin{tikzpicture}
	\Tree
	[.farmaci
		[.\node[farmaco]{\index{digitale}/\index{digossina|see{digitale}}\\ (inotropo+)};
			[.{inibizione \ce{Na+}/\ce{K+} ATPasi}
				[.{\upa\ce{Ca^2+} per\\ blocco NCX}
					{inotropo+}
				]
				[.{\dwa condutt. \ce{K+}}
					[.{\dwa durata PdA da cui\\ \upa PR, depressione\\ a cucchiaio ST}
					]
				]
			]
		]
		[.\node[farmaco]{\index{dobutamina}\\ (agonista $\beta_1$ selett.)};
			{\upa GC}
			{\dwa pre--carico}
		]
		[.\node[farmaco]{\index{furosemide}\\ (diuretico)};
			{\dwa P.A.}
			{\dwa pre--carico}
		]
		[.\node[farmaco]{\index{captopril}\\ \index{elanapril} (ACE inibitore)};
			{\dwa post--carico}
		]
		[.\node[farmaco]{\index{losartan} (antagonista AT-1)};
			{\dwa post--carico}
		]
		[.\node[farmaco]{\index{carvedilolo}\\ \index{metoprololo} ($\beta$--bloccanti)};
			{cronotopo-}
			{\dwa rimodellamento per\\ inibizione catecolamine}
		]
	]
\end{tikzpicture}

\begin{tikzpicture}
	\Tree
	[.{digitale +}
		[.\ce{K+}
			[.iper {\dwa effetti digitale}
			]
			[.ipo {\upa effetti digitale}
			]
		]
		[.\ce{Ca^2+}
			[.iper {\upa effetti digitale}
			]
			[.ipo {\dwa effetti digitale}
			]
		]
		[.\ce{Mn}
			[.iper {\dwa effetti digitale}
			]
			[.ipo {\upa effetti digitale}
			]
		]
	]
\end{tikzpicture}

\begin{tikzpicture}
	\tikzset{level 2/.style={level distance=150pt}}
	\Tree
	[.{effetti avversi}
		[.\node[farmaco]{\index{digitale}/digossina\\ (a dosi elevate)};
			{\upa aritmie, tachicardia, extrasistole\\ torsione di punta, FV}
		]
	]
\end{tikzpicture}

\subsection{Aritmie Cardiache}

\begin{tikzpicture}
	\Tree
	[.{ritmo cardiaco}
		[.{nodo seno--atriale\\(NSA)}
			[.{nod atrio--ventricolare\\(NAV)}
				[.{Fasci di His}
					\node[dummyc]{};
				]
			]
		]
	]
	\begin{scope}[yshift=-3em,xshift=1em]
	\Tree
	[.\node[dummyc]{}; 
		[.{Fibre del Purkinkje}
			[.{apice}
				[.{endocardio}
					[.{base epicardica}
					]
				]
			]
		]
	]
	\end{scope}
\end{tikzpicture}

\begin{tikzpicture}
	\tikzset{level 3/.style={level distance=130pt}}
	\Tree
	[.{fasi miocardio}
		[.{0: ascesa\\($\sim$ -65mV)}
			{corrente \ce{Na+} in ingresso}
		]
		[.{1: ripolarizzazione rapida\\($\sim$ -35mV)}
			[.{chiusura canali \ce{Na+}}
				{apertura canali \ce{K+} e \ce{Cl-} in ingresso}
			]
		]
		[.{2: playeau\\($\sim$ ??mV)}
			[.{correte \ce{Ca^2+} lenta in ingresso}
				{apertura canali \ce{Ca^2+}\\ su reticolo endplasmatico\\ che rilascia ulteriore \ce{Ca^2+}\\ che si lega aa troponina}
			]
		]
		[.{3: ripolarizzazione \\($\sim$ -20mV)}
			[.{corrente \ce{K+} in uscita\\ ($I_{\rm K_R}+I_{\rm K_S}$)}
			]
		]
		[.{4: diastole \\($\sim$ -??mV)}
			[.{la pompa \ce{Na+}/\ce{K+}\\ ripristina le condizioni iniziali}
			]
		]
	]
\end{tikzpicture}

Periodo refrattario tra fase 0 e ripristino del canale \ce{Na+} niattivati utile a consentire il propagarsi di un nuovo PdA.

\begin{tikzpicture}
	\tikzset{level 2/.style={level distance=130pt}}
	\Tree
	[.{fasi nodi}
		[.{4: depolarizzazione spontanea} 
			{apertura canali funny del \ce{Na+}}
		]
		[.{0: depolarizzazione} 
			{apertura canali \ce{Ca^2+]} in uscita}
		]
		[.{1: ripolarizzazione} 
			{apertura canali \ce{K+} e \ce{Cl-} in ingresso}
		]
		[.{2: plateau} 
			{assente}
		]
		[.{3:} 
			{assente}
		]
	]
\end{tikzpicture}

\begin{tikzpicture}
	\Tree
	[.cause
		[.{alterazioni nella\\ generazione dell'impulso}
			[.{fase 3}
				{post deplezione precoce\\(EAD)}
			]
			[.{fase 4}
				{post deplezione precoce\\(DAD)}
			]
		]
		[.{alterazioni nella\\ conduzione dell'impulso}
			[.{percorso allungato}
				{cuore dilatato}
			]
			[.{ridotta velocità\\ di conduzione}
				blocchi
				ischemie
				iperkaliemia
			]
			[.{periodo refrattario\\ accorciato}
				{risposta a farmaci}
				{stimolazione elettrica ripetitiva}
			]
		]
		[.{entrambe}
		]
	]
\end{tikzpicture}

\begin{tikzpicture}
	\Tree
	[.soluzioni
		[.{\dwa attività PMK extopici}
			{blocco canali \ce{Na+}}
			{blocco canali \ce{Ca^2+}}
			{blocco simpatico\\(cronotopo-)}
		]
		[.{disattivazione circuiti\\ di rientro}
			{prolungamento del\\ periodo refrattario}
		]
	]
\end{tikzpicture}

\begin{tikzpicture}
	\tikzset{level distance=80pt}
	\Tree
	[.{classi farmaci\\ aritmici}
		[.{Classe I}
			[.{blocco canali \ce{Na+}\\ uso dipendente\footnotemark}
				[.{\dwa pendenza corrente\\ fase 4}
					[.IA
						{dissociazione\\ intermedia}
					]
					[.IB
						{dissociazione\\ rapida}	
					]
					[.IC
						{dissociazione\\ lenta}	
					]
				]
			]
		]
		[.{Classe II}
			[.{$\beta$--bloccanti} 
				{blocco \ce{Na+},\ce{K+}}
			]
		]
		[.{Classe III}
			{blocco canale \ce{K+}.\\ Sono anche lievemente IA}
		]
		[.{Classe IV}
			{\ce{Ca^2+} antagonisti}
		]
		[.{non Vaughan Williams\\ V}
			{iperpolarizzazione per\\ attivazione canali $\rm I_{K_L}$}
		]
	]
\end{tikzpicture}

\footnotetext{Ossia agiscono soprattutto sui canali in uso ossia aperti o refrattari. Questi sono maggiormente in questi stati nei tessuti aritmici e quindi si ha un maggiore effetto proprio su quei tessuti che stanno causando il problema rispetto a quelle che funzionano normalmente.}

\begin{tikzpicture}
	\Tree
	[.farmaci
		[.IA
			[.\node[farmaco]{\index{procainamide}\\ \index{amiodarone}};
				{rallentamento della\\ ripolarizzazione fase 3}
				{prolungamento PdA}
				{aumento periodo refrattario}
			]
		]
		[.IB
			[.\node[farmaco]{\index{lidocaina}};
				{\dwa PdA}
				{\upa periodo refrattario}			
			]
		]
		[.IC
			[.\node[farmaco]{\index{flecaimide}};
				{\dwa PdA lieve}
			]
		]
		[.II
			[.\node[farmaco]{\index{propanololo}};
				{\dwa PdA lieve}
				{\upa periodo refrattario AV}
			]
		]
		[.III
			[.\node[farmaco]{\index{amiodarone}\\ \index{sotalolo}};
				{come IA}
			]
		]
		[.IV
			[.\node[farmaco]{\index{verapamil}};
				{riduzione fase plateau\\ con rallent. conduzione}
				{\upa periodo refrattario}
			]
		]
		[.V
			[.\node[farmaco]{\index{adenosina}};
				{rallentamento conduzione\\ a livello AV}
			]
		]
	]
\end{tikzpicture}

\begin{tikzpicture}
	\Tree
	[.{modifiche ECG}
		[.IA
			{\upa QRS, \upa QT}
		]
		[.IB
			{-}
		]
		[.IC
			{\upa QRS}
		]
		[.II
			{\dwa QT,cronotopo-}
		]
		[.III
			{\upa QRS, \upa QT}
		]
		[.IV
			{\upa PR}
		]
		[.V
			{?}
		]
	]
\end{tikzpicture}

\begin{tikzpicture}
	\tikzset{level 2/.style={level distance=140pt}}
	\Tree
	[.{effetti\\ collaterali}
		[.IA
			{\upa gastroent., vagolitici, atropina simili,\\ inotropo-, reazioni autoimmuni}
		]
		[.IB
			\node(s){stato confuzionale, vasocostrizione};
		]
		[.IC
			{BAV o blocco di branca, bradicardica}
		]
		[.II
			{\dwa GC, inotropo-}
		]
		[.III
			{BAV o blocco di branca, bradicardica, pro--aritmici,\\ torsione di punta}
		]
		[.IV
			{ipotensione, bradicardia, interazione con digossina\\ perchè verapamil spazza la digossina e quindi\\\upa conc. plasmatica libera}
		]
		[.V
			{ipotensione}
		]
	]
	\node[right=1em of s](par){$\left.\rule{0pt}{46pt}\right\}$};
	\begin{scope}[xshift=27.5em,yshift=4.5em]
		\tikzset{level distance=70pt}
		\Tree
		[.\node[dummyc]{};
			{???}
			aritmogeno
			{iper\ce{K+} \upa cardiotossicità}
		]
	\end{scope}
\end{tikzpicture}

\begin{tikzpicture}
	\tikzset{level distance=140pt}
	\Tree
	[.{usi clinici}
		[.\node{flutter};
			\node[farmaco](digossina){\index{digossina|see{digitale}} (???)};
		]
		[.\node(tachiaritmie){tachiaritmie};
		]
		[.\node(fibsott){fibrillazione artiale\\ sottov. parossistica};
			\node(ia){IA};
		]
		[.\node(tac){tachiaritmia sopraventr.\\ parossistica};
			\node[farmaco](sotalolo){\index{sotalolo} (III)};
			\node[farmaco](adenosina){\index{adenosina} (V)};
		]
		[.\node{aritmie post infarto};	
			\node(II){II};
			\node(ib){IB};
		]
		[.\node{aritmie da rientro};
			\node(ic){IC};
		]
		[.\node{tutte le aritmie};
			\node[farmaco](amiodarone){\index{amiodarone} (III)};
		]
	]
	\draw[drawarrow] (fibsott) to[out=0,in=180] (II);
	\draw[drawarrow] (fibsott) to[out=0,in=180] (digossina);
	\draw[drawarrow] (tac) to[out=0,in=180] (ia);
	\draw[drawarrow] (tachiaritmie) to[out=0,in=180] (ia);
\end{tikzpicture}

\subsection{Diuretici}

\subsubsection{Tubulo prossimale}

\begin{tikzpicture}
	\tikzset{level 3/.style={level distance=120pt}}
	\Tree
	[.{Tubulo prossimale}
		[.{\circleout\ce{HCO3-},\circlein\ce{H+},\circleout\ce{Cl-}\\
			\circleout\ce{NaCl} (distale), \circleout\ce{H2O}
			}
			[.{\pumplr{\ce{H+}}{\ce{Na+}} ???}
				[.{nel tubulo\\ \ce{HCO3- + H+ -> H2CO3 -> H2O + CO2}}
					\node[dummyc]{};
				]
			]
		]
	]
	\begin{scope}[yshift=-3em,xshift=1em]
		\tikzset{level 1/.style={level distance=80pt}}
		\tikzset{level 2/.style={level distance=120pt}}
		\tikzset{level 3/.style={level distance=140pt}}
		\Tree
		[.\node[dummyc]{};
			[.{\ce{H2O+CO2 ->}nella cellula}
				[.\ce{H2O+CO2 -> H2CO3 -> HCO3- + H+}
					[.{\apumprr{\ce{HCO3-}}{\ce{Na+}} + \pumpnak}
					]
				]
			]
		]
	\end{scope}
\end{tikzpicture}

Nella parte terminale del tubulo gli \ce{H+} pompati fuori non trovano quasi più \ce{HCO3-} da convertire per cui $\downarrow\ce{pH}$ dell'urina che fa attivare le \pumplr{\ce{Cl-}}{\ce{base-}} che \circleout\ce{NaCl}.

\begin{tikzpicture}
	\Tree
	[.farmaci
		[.{inibitori del\\ anidrasi cambonica}
			[.{impediscono \circleout\ce{NaHCO3}}
				[.{ma $\uparrow$\ce{NaCl}\\ nel restante nefrone}
					{$\downarrow$azione dopo qualche gg}
				]
			]
		]
		[.{diuretici osmotici\\ non assorbibili}
			[.{$\uparrow$osmolarità urine}
				{$\downarrow$\circleout\ce{H2O}\\ per osmosi}
			]
		]
	]
\end{tikzpicture}

\begin{tikzpicture}
	\Tree
	[.usi
		[.\node[farmaco]{\index{acetazolamide}\\ inibitore AC};
			{glaucoma $\downarrow$umore acqueo}
			{alcanizzazione urine}
			{alcalosi metabolica}
			[.{malattia da alta quota}
				{$\downarrow$liquido cefalorachidiano}
				{$\downarrow$edema polmonare}
			]
		]
		[.\node[farmaco]{\index{mannitolo}\\ diuretico osmotico};
			[.{per os}
				{diarrea}
			]
			[.{per IV}
				[.diuresi
					{$\downarrow$pressione intracranica}
					{$\uparrow$escrezione renale di tossine}
				]
			]
		]
	]
\end{tikzpicture}

\begin{tikzpicture}
	\Tree
	[.tossicità
		[.{inibitori AC}
			{acidosi metabolica ipercloremica}
			{calcoli renali}
			{perdita di \ce{K+} causa\\ $\uparrow$\ce{Na+} nel tubulo}
		]
		[.{osmotici}
			{disidratazione}
			{iper\ce{K+}}
			{ipernatriuremia}
		]
	]
\end{tikzpicture}

\subsubsection{Ansa di Henle (tratto discendente)}

\begin{tikzpicture}
	\Tree
	[.{Ansa di Henle (tratto discendente)}
		{\circleout\ce{H2O}}
	]
\end{tikzpicture}

\subsubsection{Ansa di Henle (tratto ascendente)}

\begin{tikzpicture}
	\tikzset{level 2/.style={level distance=180pt}}
	\tikzset{level 3/.style={level distance=140pt}}
	\Tree
	[.{Ansa di Henle\\ (tratto ascendente)}
		[.{\circleout\ce{NACl}, \circleout\ce{Mg^2+}, \circleout\ce{Ca^2+}}
			[.{\pump3{->}{\ce{Na+}}{->}{\ce{K+}}{->}{\ce{2Cl-}} NKCC2, \pumpnak,
			\apumprr{\ce{K+}}{\ce{Cl-}}} \node[dummyc]{};
			]
		]
	]
	\begin{scope}[yshift=-3em,xshift=1em]
		\tikzset{level 2/.style={level distance=100pt}}
		\tikzset{level 3/.style={level distance=100pt}}
		\Tree
		[.\node[dummyc]{};
			[.{accumulo di \ce{K+}\\ con escrezione nel tubulo}
				[.{urina sviluppa pot+}
					{passaggio di ioni \ce{Mg^2+},\ce{Ca^2+}\\ via paracellulare}
				]
			]
		]
	\end{scope}
\end{tikzpicture}

\begin{tikzpicture}
	\Tree
	[.farmaci
		[.{diuretici dell'ansa}
			[.{bloccano NKCC2}
				[.{$\uparrow$ \circleout\ce{NACl}, \circleout\ce{Mg^2+}, \circleout\ce{Ca^2+}}
				]
			]
		]
	]
\end{tikzpicture}

\begin{tikzpicture}
	\Tree
	[.usi
		\node[farmaco](s){\index{furosemide}};
		\node[farmaco]{\index{acido etacrinico}};
	]
	\node[below right=-2.5em and 1em of s](par){$\left.\rule{0pt}{32pt}\right\}$};
	\begin{scope}[xshift=13em,yshift=0em]
		\tikzset{level distance=70pt}
		\Tree
		[.\node[dummyc]{};
			{edema polmonare acuto}
			{edema}
			{ipercalcemia acuta}
			{iperkaliemia}
			{insuff. renale acuta}
			{overdose di anioni\\(bromuri, fluoruri, ioduri)}
		]
	\end{scope}
\end{tikzpicture}

\begin{tikzpicture}
	\tikzset{level distance=150pt}
	\Tree
	[.tossicità
		{alcalosi metab. ipokaliemica}
		{iperuricemia causata dal riassorbimento\\ dell'acido urico per\\ ipovolemia nel tubulo}
	]
\end{tikzpicture}

\subsubsection{Tubulo contorto distale}

\begin{tikzpicture}
	\Tree
	[.{tubulo contorto\\ distale}
		[.{\circleout\ce{NaCl}\circleout\ce{Ca^2+}}
			{\pumprr{\ce{Na+}}{\ce{Cl-}} NCC, \pumpnak}
		]
	]
\end{tikzpicture}

Non c'è qui l'ingresso del \ce{K+} quindi non c'è il riassorbimento del \ce{Mg^2+}. C'è invece il riassorbimento del \ce{Ca^2+} in quanto c'è un canale dedicato e regolato dall'ormone PTH. 

Il \ce{Ca^2+} finisce poi nel flusso sanguigno tramite due canali {\tiny\apumplr{\ce{Na+}}{\ce{Ca^2+}}} e {\tiny\apumplr{\ce{H+}}{\ce{Ca^2+}}}

\begin{tikzpicture}
	\Tree
	[.farmaci
		[.tiazidici
			[.{blocco NCC}
				{$\uparrow$escr. \ce{NaCl},\\ $\uparrow$ riass. \ce{Ca^2+}}
			]
		]
	]
\end{tikzpicture}

\begin{tikzpicture}
	\Tree
	[.usi
		[.\node[farmaco](a){\index{cloratiazide}};
			{per parenterale}
		]
		[.\node[farmaco](b){\index{idrocloratiazide}};
			\node(os){per os};
		]
		\node[farmaco](c){\index{metodazone}};
	]
	\draw (a) to[out=0, in=180] (os);
	\draw (c) to[out=0, in=180] (os);
	\node[right=9em of b](par){$\left.\rule{0pt}{42pt}\right\}$};
	\begin{scope}[xshift=21em,yshift=0em]
		\tikzset{level distance=70pt}
		\Tree
		[.\node[dummyc]{};
			{ipertensione}
			{scompenso cardiaco}
			{diabete insipido nefrogico}
		]
	\end{scope}
\end{tikzpicture}

\begin{tikzpicture}
	\Tree
	[.tossicità
		{iperuricemia}
		{iperkaliemia}
		{iperlipidemia}
		{ipernatriemia}
		{reazioni allergiche}
	]
\end{tikzpicture}

\subsubsection{Tubulo collettore}

\begin{tikzpicture}
	\Tree
	[.{tubulo collettore}
		[.{cellule principali}
			{\circleout\ce{Na+},\circleout\ce{H2O},\circlein\ce{K+}}
		]
		[.{cellule intercalate}
			{\circleout\ce{HCO^-3},\circleout\ce{H2O},\circlein\ce{H+}}
		]
	]
\end{tikzpicture}

Il sodio viene riassorbito dal tubulo, il potassio vie escreto e la pompa sodio--potassio tenta di mantenere l'equilibrio. Più \ce{Na+} viene assorbito e più \ce{K+} viene escreto. Tutto questo regolato dall'aldosterone.

Ecco il motivo per cui i diuretici depauperano il corpo di potassio.

In questo stesso settore, l'ADH regola l'espressione di acquaporine di tipo 2 e $\uparrow$ADH causa $\uparrow$acq2 e quindi $\uparrow\circleout\ce{H2O}$

\begin{tikzpicture}
	\Tree
	[.farmaci
		[.{diuretici risparmiatori\\ di \ce{K+} sui recettori\\ dei mineralcorticoidi}
			[.{antagonisti dell'aldosterone}
				{$\downarrow$\circleout\ce{Na+}, $\downarrow$\circlein\ce{K+}}
			]
		]
		[.vaptani
			{antagonisti ADH}
		]
	]
\end{tikzpicture}

\begin{tikzpicture}
	\Tree
	[.usi
		[.\node[farmaco]{\index{spironolattone}};
			[.{iperaldosterone}
				{primatio da sindrome di Conn}
				{secondario da scompenso cardiaco}
			]
		]
		[.vaptani
			{sindrome da $\uparrow$secrezione ADH}
		]
	]
\end{tikzpicture}

\begin{tikzpicture}
	\Tree
	[.tossicità
		[.\node[farmaco]{\index{spironolattone}};
			{iperkaliemia}
			{ginecomastia}
			{acidosi metabolica iperclorica}
		]
		[.vaptani
			{diabete insipido}
			{insufficienza renale}
		]
	]
\end{tikzpicture}

\newpage

	\section{Farmaci del sistema respiratorio}

\subsection{Asma}

\begin{tikzpicture}
	\tikzset{level 2/.style={level distance=130pt}}
	\tikzset{level 3/.style={level distance=130pt}}
	\Tree
	[.Asma
		[.{malattia infiammatoria\\ delle vie aeree}
			{infiammazione}
			[.{ostruzione bronchiale}
				{solitamente reversibile}
				{in alcuni casi irreversibile}
			]
			{iperreattività agli allergeni}
		]
	]	
\end{tikzpicture}

\begin{tikzpicture}
	\Tree
	[.sintomi
		{sibili respiratori}
		{dispnea}
		tosse
		{costrizione torace}
	]
\end{tikzpicture}

\begin{tikzpicture}
	\tikzset{level distance=80pt}
	\Tree
	[.fisiopatogenesi
		[.{ingesso allergene}
			[.{APC presentano\\ antigeni ai \ce{T_H_2}}
				[.{\ce{T_H_2} stimolano B\\ a produrre IgE}
					[.{IgE si legano\\ agli allergeni}
						\node[dummyc]{};
					]
				]
			]
		]
	]
	\begin{scope}[yshift=-6em]
	\Tree
	[.\node[dummyc]{};
		[.{Mastociti legano IgE}
			{fase immediata}
			{fase tardiva}
		]
	]
	\end{scope}
\end{tikzpicture}

\begin{tikzpicture}
	\tikzset{level 1/.style={level distance=75pt}}
	\tikzset{level 2/.style={level distance=90pt}}
	\tikzset{level 3/.style={level distance=90pt}}
	\tikzset{level 4/.style={level distance=80pt}}
	\Tree
	[.{fase immediata}
		[.{mastociti\\ liberano}
			[.istamina
				[.{liberazione \ce{Ca^2+} nel REL}
					{broncospasmo}
				]
			]
			[.{leucotreni/citochina}
				[.IL5
					[.{attivazione eosinofili}
						{danno tissutale}
						{edema}
						{congestione}
					]
					[.{fase tardiva}
					]
				]
				[.IL4
					{stimolo a produrre IgE}
				]
			]
			[.{fattori di crescita}
			]
		]
	]
\end{tikzpicture}

\begin{tikzpicture}
	\tikzset{level distance=140pt}
	\Tree
	[.{fase tardiva}
		{inspessimento della parete\\ con restringimento del lume}
		{flogosi}
		rimodellamento
		{$\uparrow$produzione muco}
	]
\end{tikzpicture}

tutto ciò causa iperresponsività bronchiale futura.

\begin{tikzpicture}
	\tikzset{level distance=140pt}
	\Tree
	[.{farmaci}
		[.{broncodilatatori\\ (a breve durata d'azione)}
		]
		[.{glucocorticosteroidi\\ (in aerosol)}
		]
		[.{broncodilatatori\\(a lunga durata d'azione)}
		]
		[.{metilxantine o\\ antagonisti dei leucotreni}
		]
		[.{corticosteroidi orali}
		]
		[.{anticorpi monoclonali anti--IgE}
		]
	]
	\begin{scope}[xshift=18em]
		\draw[drawarrow] (0,2) -> (0,-2);
		\node[text width=8em] at (2,0){step operativi via via che la malattia diventa più grave};
	\end{scope}
\end{tikzpicture}



\newpage
	\section{Farmaci dell'emostasi}

\begin{tikzpicture}
	\tikzset{frontier/.style={distance from root=300pt}} 
	\Tree 
		[ .Emostasi 
			[ .anticoagulanti 
				[ .iniettabili 
					\node[farmaco]{eparina};
				]
				[ .{inibitori della trombina} 
					\node[farmaco]{lepirudina\\ argatroban}; 
				]
				[ .orali \node[farmaco]{warfarin};  ]
			]
			[ .antiaggreganti 
				[ .FANS ]
				[ .{inibitori \\ della fosfodiesterasi} 
					\node[farmaco]{dipiridamolo\\ cilostazolo };
				]
				[ .{antagonisti recettori ADP} 
					\node[farmaco]{ticlopidina\\ clopidogrel }; 
				]
				[ .{inibitori recettore\\ Gp IIb/IIIA} 
					\node[farmaco]{abciximab\\ tirofiban\\ eptifibatide}; 				
				]
			]
			[ .trombolitici 
				[.{Attivatori tissutali\\ del plasminogeno (tPA)}
					\node[farmaco]{urochinasi\\ streptochinasi }; 
				]
			]
		]
			
\end{tikzpicture}

	\chapter{Farmaci epatici}

\section{Citocromo P450}

\ce{RH + O_2 +2H^+ + 2e^- ->[monoossigenasi] ROH + H2O}

\begin{tikzpicture}
	\tikzset{level distance=130pt}
	\Tree
	[.\node(inibitori){inibitori};
		[.CYP1A2
			\node[farmaco]{\index{ciprofloxacina}ciprofloxacina};
			\node[farmaco]{\index{tacrina}tacrina};
		]
		[.CYP2C9
			\node[farmaco]{\index{fluconazolo}fluconazolo};
		]
		[.{CYP3A4\\(principale nella\\ detossificazione da farmaci)}
			\node[farmaco]{\index{eritromicina}eritromicina};
			\node[farmaco]{\index{claritromicina}claritromicina};
			\node[farmaco]{\index{ketocomazolo}ketocomazolo};
			{succo di pompelmo}
		]
	]
	\begin{scope}[yshift=-10em]
		\Tree
		[.\node(metabolismo){metabolismo};
			\node[farmaco]{\index{corticosteroidi}corticosteroidi};
			\node[farmaco]{\index{warfarin}warfarin};
		]
	\end{scope}
	\begin{scope}[yshift=-20em]
		\Tree
		[.\node(attivatori){attivatori};
			barbiturici
			\node[farmaco]{\index{carbamazepina}carbamazepina};
		]
	\end{scope}
	\node[chartnode](rallenta) at (0,-5em) {rallenta};
	\node[chartnode](accellera) at (0em,-15em) {accellera};
	\draw[drawarrow] (attivatori) -> (accellera);
	\draw[drawarrow] (accellera) -> (metabolismo);
	\draw[drawarrow] (inibitori) -> (rallenta);
	\draw[drawarrow] (rallenta) -> (metabolismo);
\end{tikzpicture}

\newpage
	\printindex
\end{document}
