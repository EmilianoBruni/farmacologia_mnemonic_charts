\documentclass[12pt,paper=a4,twoside=false,parskip=half]{scrartcl}
% aggiungere draft alla classe per vedere gli overfull hbox

% /-- Packages loading --------------------------------------------------------\
\usepackage[italian]{babel}
\usepackage[utf8]{inputenc} 	% lettere accentate in documento UTF-8
\usepackage[T1]{fontenc} 		% doppi --
\usepackage{lmodern}
\usepackage{amsmath}
\usepackage{amsthm}
\usepackage{amsfonts}
\usepackage{units}
\usepackage{cancel}         
\usepackage{tikz}
\usepackage{tikz-qtree}
\usepackage{fancyhdr}           % Header e Footer 
\usepackage{chemfig}
\usepackage[version=3]{mhchem}
\usepackage{makeidx}			% Indice
\usepackage[colorlinks=true,allcolors=blue]{hyperref}
% \----------------------------------------------------------------------------/

% /-- title and authors -------------------------------------------------------\
\def\gtitle#1{\gdef\gtitle{#1}}
\def\gauthor#1{\gdef\gauthor{#1}}
\title{Appunti di Farmacologia}
\gtitle{Appunti di Farmacologia}
\author{Emiliano Bruni (info@ebruni.it)}
\gauthor{Emiliano Bruni (info@ebruni.it)}
\date{} %%If commented, the current date is used.

\pdfinfo {
/Title (\gtitle)
/Author (\gauthor)
/Subject (Appunti di Farmacologia)
/Keywords (farmacology, university, medicine, book, charts, mnemonic, flash, card, italian, Chieti, appunti)
/Copyright (copyrighted)
}
% \----------------------------------------------------------------------------/

\def\tikzbyncsa{
	\draw[fill=black] (2pt,2pt) rectangle (78pt,13pt);
	\begin{scope}
	\clip (0,0) rectangle (80pt,15pt);
	\fill[gray!67] (12pt,7.5pt) circle [x radius=19pt,y radius=14pt];
	\node[draw,
		circle,
		inner sep=2pt,
		outer sep=0ex,
		very thick, fill=white] at (15pt,7.5pt) {\small\bfseries\textsf{cc}};
	\end{scope}
	\draw[line width=1pt, white] (78pt,14pt)--(1pt,14pt)--(1pt,1pt) -- (78pt,1pt);
	\draw[line width=1pt] (0,0) rectangle (80pt,15pt);
	\node[white] at (55pt,7.5pt) {\scriptsize\bfseries\textsf{BY-NC-SA}};
}

% /-- Sezione/numeropag. nell'header e date e numero di revisione nel footer --\
\pagestyle{fancy}
\newcommand{\helv}{\fontfamily{phv}\fontseries{b}\fontsize{8}{10}\selectfont}
\fancyhf{}
\fancyhead[LO]{\helv \rightmark}
\fancyhead[LO]{\helv \leftmark}
\fancyhead[RO]{\helv \thepage}
\fancyfoot[ol]{\begin{tikzpicture}[scale=0.6, transform shape,baseline=2pt ]\tikzbyncsa\end{tikzpicture}\ \scriptsize\itshape Copyright \copyright\ 2016 \gauthor }
\fancyfoot[or]{\scriptsize\itshape Revisione del \today } % Date
% \----------------------------------------------------------------------------/

\usetikzlibrary{shapes,arrows,matrix,calc,automata,positioning,patterns}
\numberwithin{equation}{section}  % numeri delle equazioni del tipo x.y
\numberwithin{table}{section}     % numeri delle tabelle del tipo x.y
\numberwithin{figure}{section}    % numeri delle figure del tipo x.y


\makeindex							% generate the index


\begin{document}
	\tikzset{
	%Define standard arrow tip
	>=stealth',node distance=1cm, auto,font=\tiny,
	%Define style for boxes
	itm/.style={
		rectangle,
		rounded corners,
		draw=black, very thick,
		minimum width=5em,
		minimum height=1.5em,
		text centered,
		align=center,
		inner sep=3pt,
	},
	% Define arrow style
	ar/.style={
		<-,
		thick,
		shorten <=2pt,
		shorten >=2pt,},
	dummyar/.style={
		thick,
		shorten <=0pt,
		shorten >=2pt,},
	dummy/.style={
		minimum width=10ex,
		minimum height=0,
		inner sep=0,
		outer sep=0,
	},
	dummy0/.style={
		minimum width=0,
		minimum height=0,
		inner sep=0,
		outer sep=0,
	},
	count/.style={
		draw,
		circle,
		inner sep=.3ex,
		outer sep=.3ex,
		thick,
		anchor=south,fill=blue!30!white!10
	},
}
\def \lastitem {}
\def \dummy #1{\node[dummy,right=of #1] (#1_dummy) {};\gdef\lastitem{#1_dummy}}
\def \dummyz #1{\node[dummy0,right=of #1] (#1_dummy) {}}
\def \dummyzar #1{\node[dummy0,right=of #1] (#1_dummy) {} edge[ar] (#1.east);\gdef\lastitem{#1_dummy}}
\def \dummystart #1#2{\node[dummy0, below=of #2] (start#1) {};\gdef\lastitem{start#1}}
\def \dummyend #1{\node[dummy0,right=of #1] (#1_end) {} edge[ar] (#1.east);;\gdef\lastitem{#1_end}}
\def \itm #1#2{\gdef\lastitem{#1};\node[itm,fill=blue!40!white!40] (#1) {#2}}

\def \itmright #1#2#3#4{\node[itm, right=of #1] (#2) {#3} edge[ar] (#4.east);\gdef\lastitem{#2}}
%\def \itmabove #1#2#3#4{\node[itm, above=of #1] (#2) {#3} edge[ar] (#4.east);\gdef\lastitem{#2}}
\def \itmabove #1#2#3#4{
	\node[itm, above=1em of #1] (#2) {#3}; 
	\draw[thick,shorten <=2pt] (#4.east) -- ++(.3,0) [shorten >=2pt,shorten <=0pt,->] to[out=0,in=225] (#2.south west);
	\gdef\lastitem{#2}
}
\def \itmbelow #1#2#3#4{
	\node[itm, below=1em of #1] (#2) {#3};
	\draw[thick,shorten >=2pt,->] ($(#4.east) + (.3,0)$)  to[out=0,in=155] (#2.north west);
	\gdef\lastitem{#2};
}

\def \itmrightnoarrow #1#2#3{\node[itm, right=of #1] (#2) {#3}}

\def \itmrights #1#2#3{\itmright{#1}{#2}{#3}{#1}}
\def \itmaboves #1#2#3{
		\node[itm, above=of #1] (#2) {#3};
		\draw[ar]($(#2.south) + (.5cm,0)$) to[bend left] ($(#1.north)+ (.5cm,0)$);
		\gdef\lastitem{#2}
}
\def \itmbelows #1#2#3{
	\node[itm, below=of #1] (#2) {#3};
	\draw[ar]($(#2.north) + (-.5cm,0)$) to[bend left] ($(#1.south)+ (-.5cm,0)$);
	\gdef\lastitem{#2}
}

\def \itmsplit #1#2#3#4#5{\dummy{#1};
	\itmabove{#1_dummy}{#2}{#3}{#1};
	\itmbelow{#1_dummy}{#4}{#5}{#1}}

\def \itmmerge #1#2#3#4#5{\dummyz{#1_dummy}
			edge[dummyar] (#2.east)
			edge[dummyar] (#3.east);
	\node[itm, right=of #1_dummy_dummy] (#4) {#5}
			edge[ar,shorten >=0pt] (#1_dummy_dummy.north)}

\def \itmsplitthree #1#2#3#4#5#6#7{
	\itmrights{#1}{#4}{#5};
	\itmabove{#4}{#2}{#3}{#1};
	\itmbelow{#4}{#6}{#7}{#1};
}

\def \itmsplitfour #1#2#3#4#5#6#7#8#9{
	\node[itm,above right=0em and 3em of #1] (#4) {#5};
	\node[itm,above right=3em and 3em of #1] (#2) {#3};
	\node[itm,below right=0em and 3em of #1] (#6) {#7};
	\node[itm,below right=3em and 3em of #1] (#8) {#9};
	
	\draw[thick,shorten <=2pt] (#1.east) -- ++(.3,0) [shorten >=2pt,shorten <=0pt,->] to[out=0,in=225] (#2.south west);
	\draw[thick,shorten >=2pt,->] ($(#1.east) + (.3,0)$)  to[out=0,in=205] (#4.south west);
	\draw[thick,shorten >=2pt,->] ($(#1.east) + (.3,0)$)  to[out=0,in=155] (#6.north west);
	\draw[thick,shorten >=2pt,->] ($(#1.east) + (.3,0)$)  to[out=0,in=135] (#8.north west);
}

\def \armergetwo #1#2#3{ 
	\draw[thick,shorten <=2pt] (#1.east) to[out=0,in=180] ($(#3.west) + (-.3cm,0)$);
	\draw[thick,shorten <=2pt,shorten >=2pt,->] (#2.east) to[out=0,in=180] ($(#3.west) + (-.3cm,0)$) -- (#3.west);
}

\def \armergethree #1#2#3{
	\draw[thick,shorten <=2pt] (#1.east) to[out=0,in=180] ($(#3.west) + (-.3cm,0)$);
	\draw[thick,shorten <=2pt] (#2.east) to[out=0,in=180] ($(#3.west) + (-.3cm,0)$);
}

\def \itmcount #1#2{\node[count] at (#2.north) {#1};}

\newcounter{itmcvalue}
\def \itmcinit #1{\setcounter{itmcvalue}{#1}}
\def \itmcnoinc {\addtocounter{itmcvalue}{-1};}
\def \itmc #1{\itmcount{\arabic{itmcvalue}}{#1};\addtocounter{itmcvalue}{1};}

\def\itmr #1#2{\itmrights{\lastitem}{#1}{#2}}

% Set for tkiz-qtree
\tikzset{
	grow'=right,level distance=100pt,
	frontier/.style={distance from root=800pt},
	every tree node/.style={
		rectangle,
		rounded corners,
		draw=black, very thick,
		minimum width=5em,
		minimum height=1.5em,
		text centered,
		align=center,
		inner sep=3pt,
		font=\ttfamily\normalsize\tiny
	},
	every level 0 node/.style={
		top color=white, bottom color=blue!30
	},
	edge from parent/.append style={
		draw,->, thick,
		shorten <=0pt,
		shorten >=2pt,edge from parent path={
			(\tikzparentnode) to[out=0,in=180] (\tikzchildnode)
		}
	},
		farmaco/.style={
		top color=white, bottom color=green!30
	},
}

	\maketitle
	\begin{abstract}
	Questo articolo riassume con delle carte mnemoniche gli argomenti di farmacologia spiegati nel IV anno del corso di laurea in medicina e chirurgia a Chieti.
	L'uso di questo articolo non sostituisce la lettura e lo studio di un libro e degli appunti di farmacologia.
	Per errori, omissioni o altre note, non esitate a contattarmi via e-mail.
\end{abstract}

\newpage

\tableofcontents

\newpage\newpage
	\chapter{Farmacocinetica}

\section{Emivita}

L'emivita di un farmaco è definita come il tempo necessario a ridurre
il farmaco a \unitfrac{1}{2} della quantità di farmaco presente nell'organismo
allo steady-state.

Presupponendo che la quantità di farmaco nell'organismo abbia un andamento esponenziale decrescente con il tempo, si pu definire questo matematicamente come:

$$
Q(t)=\alpha e^{-\beta t}
$$

Per trovare i due parametri $\alpha$ e $\beta$ consideriamo che a $t=0$
$Q(0)=Q_{\text{TOT}}=\alpha$ e quindi l'equazione sopra si pro scrivere come
$$
Q(t)=Q_{\text{TOT}}e^{-\beta t}
$$
e d'altra parte se consideriamo la velocità di eliminazione del farmaco
al tempo $t$ si ha che\vspace{.5em}

$-\dfrac{\text{d}\,Q(t)}{\text{d}\,t}=v_{\text{elim}}(t)=-Q_{\text{TOT}}(-\beta)e^{-\beta t}$ \vspace{.5em}

Ma d'altra parte, per definizione
$$
\text{CL} = \dfrac{v_\text{ELIM}^\text{STEADY STATE}}{c^\text{STEADY STATE}} =\dfrac{v_\text{ELIM}(0)}{c(0)}
$$
e, a $t=0\Rightarrow v_{\text{elim}}(0)=\text{CL}\cdot c(0)=-Q_{\text{TOT}}(-\beta)$ da cui
$\beta=\dfrac{\text{CL\ensuremath{\cdot}}c(0)}{Q_{\text{TOT}}}$ ma

$$V_\text{DIST} = \dfrac{Q_{\text{TOT}}}{c(0)}$$ 

e quindi \vspace{.5em}

$\beta=\dfrac{\text{CL\ensuremath{\cdot}}\cancel{c(0)}}{V_{\text{DIST}}\cdot\cancel{c(0)}}\Rightarrow\beta=\dfrac{\text{CL}}{V_{\text{DIST}}}$ e quindi

$$
Q(t)=Q_{\text{TOT}}e^{-\frac{\text{CL}}{V_{\text{DIST}}}t}
$$


a $t=t_{\unitfrac{1}{2}}\Rightarrow Q(t_{\unitfrac{1}{2}})=\dfrac{1}{2}\cancel{Q_{\text{TOT}}}=\cancel{Q_{\text{TOT}}}e^{-\frac{\text{CL}}{V_{\text{DIST}}}t_{\unitfrac{1}{2}}}$ 

e passando ai logaritmi naturali

$\ln\dfrac{1}{2}=-\dfrac{\text{CL}}{V_{\text{DIST}}}t_{\unitfrac{1}{2}}\Rightarrow t_{\unitfrac{1}{2}}=\ln\dfrac{1}{2}\cdot\left(-\dfrac{V_{\text{DIST}}}{\text{CL}}\right)=\dfrac{\ln2\cdot V_{\text{DIST}}}{\text{CL}}$

e quindi

\[
t_{\unitfrac{1}{2}}\simeq0.7\cdot\dfrac{V_{\text{DIST}}}{\text{CL}}
\]


	\newpage\part{Flash Cards}

\section{Farmaci del SNC e del SNP}

\begin{tikzpicture}
	\tikzset{level distance=90pt,frontier/.style={distance from root=370pt}} 
	\Tree
	[.SNC
		[.Periferico
			[.Sensitivo ]
			[.Autonomo
		 		[ .{Simpatico\\ (toraco--addominale)} {gangli pre e para--vertebrali} ]
				[ .{Parasimpatico \\(nervi crani e sacrale)} {nell'intima degli organi} ]
			]
			[.Gastroenterico {plessi mioenterici (Auerbach) e\\ sottomucosi (Meissner)}
			]
		]
		[.Centrale ]
	]
\end{tikzpicture}

\begin{tikzpicture}
	\tikzset{level 1/.style={level distance=150pt}}
	\Tree
	[.{Neurotrasmettitori\\ SNP}
		 [.\node(acetilcolina){acetilcolina}; {recettori colinergici} ]
		 [.noradrenalina {recettori adrenergici} ]
		 [.\node(serotonina){serotonina\\5-HT 5-idrossitriptamina}; {recettori serotoninergici} ]
		 [.{monossido d'azoto (NO)} ] 
		 [.purine ]
	]
	\begin{scope}[yshift=-11em]
	\Tree
	[.\node(snc){Neurotrasmettitori\\ SNC};
		[.dopamina {recettori dopaminergici} ]
		[.{aminoacidi eccitatori}
			L--glutammato
			aspatato
			omocisteinato
		]
		GABA
	]
	\end{scope}
	\draw[drawarrow] (snc.east) to[in=180,out=40] (acetilcolina.west)
		(snc.east) to[in=180,out=0] (serotonina.west);
\end{tikzpicture}

\begin{tikzpicture}
	\Tree
	[.{Siti di azione\\ farmaci del SNC}
		[.{fibra pre--sinapica}
			{sintesi}
			{immagazzinamento}
			{rilascio}
			{ricaptazione}
			{metabolismo}
		]
		[.{fessura sinaptica}
			{recettore}
			{degradazione}
		]
		[.{fibra post--sinapica}
			{conducibilità ionica\\ del canale}
			{segnalazione retrograda}
		]
	]
\end{tikzpicture}

\subsection{Acetilcolina}

\begin{tikzpicture}
	\tikzset{level 1/.style={level distance=150pt}}
	\Tree
	[.\node(loc){localizzazione}; {tutte le fibre pregangliali sia para che orto\\ nicotiniche} {parasimpatiche post gangliali (quasi tutte).\\ muscariniche} {ghiandole sudoripare (simpatico)\\ muscariniche} {giunzione neuromuscolare\\ nicotiniche} 
		[.SNC
			proencefalo
			mesencefalo
			{tronco celebrale}
			cervelletto
			{interneuroni corpo striato}
		]
	]
\end{tikzpicture}

\begin{tikzpicture}
	\tikzset{level 2/.style={level distance=130pt}, level 3/.style={level distance=120pt}}
	\Tree
	[.Sintesi
		[.colina \edge node[smallfont,yshift=5pt,xshift=5.4em]{entra nel neurone} node[smallfont,yshift=-5pt,xshift=5.4em]{tappa limitante};
			[.{acetilCOA + colina} \edge node[smallfont,yshift=-5pt,xshift=5em]{acetiltrasferasi} node[smallfont,yshift=5pt,xshift=5em]{colina}; acetilcolina ]
		]
	]
\end{tikzpicture}

\begin{tikzpicture}
	\tikzset{level 2/.style={level distance=130pt}}
	\Tree
	[.Degradazione
		[.acetilcolina \edge node[smallfont, yshift=-5pt,xshift=5.5em]{acetilcolinesterasi}; {acetato + colina} ]
	]
\end{tikzpicture}

\begin{tikzpicture}
	\Tree
	[.Liberazione 
		[.{Ca${}^{2+}$ + VAMP/SNAPS}
			[.{fusione vescicole con\\ membrana neuronale} esocitosi
			]
		]
	]
\end{tikzpicture}

\begin{tikzpicture}
	\tikzstyle{cwhite}=[circle,shadedraw=yellow];
	\shade[ball color=yellow] node (ach) {\small Ach} circle[radius=.45];
	\draw (ach) -- +(.7,.7) node(vamp){} arc [start angle=225, end angle=270, radius=6pt]  (.7,.7) arc [start angle=225, end angle=180, radius=6pt];
	\draw (2,2) arc [start angle=45, end angle=0, radius=3cm] node[midway,above,sloped]{\tiny spazio sinaptico} ;
	\draw (2,2) 
		arc [start angle=45, end angle=60, radius=3cm] node(a){} 
		node[above right=2pt and 2pt of a] {\tiny Canale Ca${}^{2+}$}
		(a) arc [start angle=140, end angle=120, radius=1cm]
		(a) arc [start angle=140, end angle=160, radius=1cm];
	\path (2,2) arc [start angle=45, end angle=65, radius=3cm] node(b){};
	\draw (b) arc [start angle=-50, end angle=-30, radius=1cm]
		(b) arc [start angle=-50, end angle=-70, radius=1cm]
		(b) arc [start angle=65, end angle=80, radius=3cm];
	\draw (2,2) -- (1.3,1.3);
	\filldraw (1.3,1.3) circle[radius=3pt] node(snap){};
	\draw[->, shorten <=0pt,shorten >=2pt] (vamp)--(snap) node[midway,above,circle,draw,,yshift=3pt,xshift=-8pt]{\tiny 2};
	\draw[->, shorten <=0pt,shorten >=2pt] (ach) to[out=-20,in=225] node[midway,below,circle,draw,yshift=-5pt]{\tiny 3}  (4,1) ;
	\draw[->, shorten <=2pt,shorten >=6pt](3,3.5) to[out=205,in=90] node[near end,above,circle,draw,yshift=15pt]{\tiny 1}  node[very near end,above,yshift=15pt,xshift=-2pt]{\tiny Ca${}^{2+}$} (ach) ;
	\node at (1,.4) {\tiny VAMP};
	\node at (1.9,1.3) {\tiny SNAP};
\end{tikzpicture}

\begin{tikzpicture}
	\tikzset{level distance=80pt}
	\Tree
	[.Effetti
		[.SNP
			[.{$\uparrow$permeabilit\`a ai cationi\\ Na${}^+$, K${}^+$, Ca${}^{2+}$}
				[.depolarizzazione
					[.{fibre post gangliari} PdA ]
					[.{fibre muscolari} {generazione potenziale\\ di placca}
					]
				]
			]
		]
		[.SNC
			[.{$\uparrow$Ach}
				{veglia, apprendimento,\\ memoria}
			]
		]
	]
\end{tikzpicture}

\begin{tikzpicture}
	\tikzset{level 1/.style={level distance=80pt},level 2/.style={level distance=140pt},level 3/.style={level distance=140pt}}
	\Tree
	[.{Recettore colinergico}
		[.{muscarinico\\ (metabotropo)}
			[.{M${}_1$ eccitatorio $\uparrow\text{IP}_3, \uparrow\text{DAG},\uparrow\text{Ca}^{2+}$} {SNC, simpatico post--gangliare,\\ cellule parietali dello stomaco}
			]
			[.{M${}_2$ inibitorio $\downarrow$cAMP } {cuore, endotelio dei vasi}
			]
			[.{M${}_3$ eccitatorio $\uparrow\text{IP}_3, \uparrow\text{DAG},\uparrow\text{Ca}^{2+}$} {ghiandole esocrine, muscolo liscio,\\ endotelio dei vasi}
			]
			[.{M${}_4$ come $\text{M}_2$} {SNC}
			]
			[.{M${}_5$  come $\text{M}_1$} {endotelio vasale, cervello,\\ SNC (facilita rilascio glutammato e dopamina)}
			]
		]
		[.{nicotinico\\ (ionotropo)}
			[.{N${}_\text{N}$ gangliare} {para e ortosimpatico gangliare} ]
			[.{N${}_\text{M}$ muscolare} {giunzione\\ neuromuscolare} ]
		]
	]
\end{tikzpicture}

\subsubsection{Agonisti colinergici}

\begin{tikzpicture}
	\begin{scope}
	\tikzset{
		level distance=80pt,
		level 1/.style={level distance=70pt},
		level 2/.style={level distance=70pt},
		frontier/.style={distance from root=400pt} 
	}
	\Tree
	[.\node(main){Tipo}; 
		[.diretti
			[.{attivano recettori} 
				[.{esteri della colina} 
					\node(ach){acetilcolina };
					metacolina
					carbacolo
					\node[farmaco](beta){betanecolo};
				]
				[.alcaloidi
					[.muscarinici
						muscarina
						\node[farmaco]{pilocarpina};
					]
					[.nitotinici \node[farmaco]{nicotina}; ]				
				]
			]
		]
		[.indiretti
			[.\node(AchEI){inibitori AchE}; 
			]
		]
	]	
	\node[right=5pt of ach] (achn) {};
	\node[right=5pt of beta] (betan) {};
	\draw[drawarrow] (achn) -- (betan) node[midway,above,sloped] {\tiny $\uparrow$resist. idrolisi e quindi durata azione};
\end{scope}
\begin{scope}[yshift=-14em]
	\Tree
	[.\node(AchEIa){inibitori AchE};
		[.{alcool+gruppo N quaternario}
			[.\node[farmaco]{edrofonio}; 
				[.{legame idrogeno\\ o ionico con AchE} {idrolisi in minuti}
				]
			]
		]
		[.carbammati 
			[.\node[farmaco]{neostigmina\\ fisostigmina\footnotemark}; 
				[.{legame covalente con AchE} {idrolisi in ore}
				]
			]
		]
		[.organofosfati
			[.\node[farmaco](ecotiopato){ecotipato\footnotemark};
				[.\node(fAchE){fosforilazione AchE}; \node(idro){idrolisi in giorni};
				]
			]
			\node(somar){somar\\ (gas nervino)};
		]
	]		
	\draw[drawarrow] (somar) to[out=0,in=180] (fAchE);
	\node[chartnode,below=1em of fAchE] (invec) {invecchiamento\\rottura legame O-P\\ con raffozamento\\ legame con AchE}; 
	\node[chartnode,below=1em of idro] (pral) {pralidossina pu\`o\\ scindere la\\ fosforilazione};
	\draw[drawarrow] (pral)--(fAchE) node[midway,above,sloped] {\tiny qui si};
	\draw[drawarrow] (pral)--(invec) node[midway,above,sloped] {\tiny qui no};
	\node[below right=8pt and 5pt of somar] (somary) {};
	\node[above=9em of somary] (neox) {};
	\draw[drawarrow] (neox) -- (somary) node[midway,above,sloped] {\tiny $\uparrow$durata azione};
	\draw[drawarrow] (AchEI.south) to[out=-90,in=90] (AchEIa);
\end{scope}
\end{tikzpicture}

\footnotetext{Presente nella fava del Calabar}
\footnotetext{Unico degli organofosfati perch\`e altamente polare e pu\`o essere preparato come soluzione acquosa. Era utilizzato per il glaucoma, ora in disuso.}

\begin{tikzpicture}
	\tikzset{level distance=120pt, level 2/.style={level distance=150pt}}
	\Tree
	[.\node(sub){Effetti}; 
		[.{SNC ($\text M_1$)} {tremore, ipotermina, $\uparrow$capacità cognittive}	
		]
		[.{occhio ($\text M_3$)}
			{costrizione muscolo sfintere dell'iride (miosi)\\ contrazione muscolo ciliare (accomodamento da vicino)}
		]
		[.{cuore ($\text M_2$)}
			{$\downarrow$frequenza (cronotopo-), $\downarrow$forza (inotropo-),\\ $\downarrow$vel. conduzione (dromotropo-), $\uparrow$periodo refrattario, NAV}
		]
		[.{vasi ($\text M_2$)}
			{vasodilatazione a basse dosi\\ vasocontrazione a alte dosi}
		]
		[.{polmone ($\text M_3$)}
			 {broncocostrizione, $\uparrow$secrezione}
		]
		[.{intestino ($\text M_3$)}
			{$\uparrow$motilit\`a, $\downarrow$ muscolatura sfinteri, $\uparrow$ secrezioni} 
		]
		[.vescica
			 {contrazione destrusore ($\text M_3$), rilascio trigono ($\text M_2$)}
		]
		[.{ghiandole esocrine ($\text M_3$)}
			{$\uparrow$secrezioni}
		]
		[.{giunzione neuromuscolare\\ (indiretti)}
			{basse concentrazioni: $\uparrow$forza contrazione utile \\ se intossicazioni da curaro o miastenia grave\\ alte concentrazioni: fibrillazione fibre muscolari}
		]
	]
\end{tikzpicture}

\begin{tikzpicture}
	\tikzset{frontier/.style={distance from root=350pt}, level 2/.style={level distance=130pt}}
	\Tree
	[.{usi clinici}
		[.\node[farmaco](pilocarpina){pilocarpina};
			[.{xerostomia da\\ sindrome di Sjogren} {$\uparrow$secrezioni salivali}
			]
		]
		[.\node[farmaco](ecotiopato){ecotiopato};
			[.\node(glaucoma){glaucoma};
				{nelle emergenze ad angolo chiuso,\\ agonista muscarinico + inibitore colinesterasi. \\	nel glaucoma cronico ora si usano i $\beta$-bloccanti}
			]
		]
		[.\node[farmaco](betanecolo){betanecolo};
				\node(ritenzione){ritenzioni urinarie e ileo\\ depressione dell'attivit\`a senza ostruzione};
		]
		[.\node[farmaco](neostigmina){neostigmina};
			[.\node(miastenia){miastenia grave};
				{cura e mezzo diagnostico}
			]
		]
		[.\node[farmaco](edofonio){edofonio};
			{tachiaritmia parossistica sopraventricolare\\ in disuso, ora si usa l'adenosina.}
			\node(post){post anestesia per revertire i\\ neurobloccanti muscolari (vedi)};
		]
		[.\node[farmaco](fisostigmina){fisostigmina};
			{intossicazione da farmaci antimuscarinici\\ (intossicazione da atropina)}
		]
	]
	\draw[drawarrow](pilocarpina) to[out=0, in=180] (glaucoma);
	\draw[drawarrow](neostigmina) to[out=0, in=180] (ritenzione);
	\draw[drawarrow](neostigmina) to[out=0, in=180] (post);
	\draw[drawarrow](betanecolo) to[out=0, in=180] (ritenzione);
	\draw[drawarrow](edofonio) to[out=0, in=180] (miastenia);
\end{tikzpicture}

\subsubsection{Antagonisti colinergici}

\begin{tikzpicture}
	\Tree
	[.{antagonisti\\ colinergici}
		[.antimuscarinici 
			[.\node[farmaco]{atropina}; {deriva da Belladonna e\\ Datura Stramonium}
			]
			\node[farmaco]{scopolamina};
		]
		[.antinicotinici 
			[.{bloccanti gangli\\ ganglioplegici} \node[farmaco]{trimetafano}; 
			\node[farmaco]{tossina botulina};
			]
			[.{bloccanti neuromuscolari}
			]
		]
		[.{rigeneratori dell'AchE} \node[farmaco]{pralidossima\footnotemark};
		]
	]
\end{tikzpicture}

\footnotetext{vedi inibitori dell'AchE}

\begin{tikzpicture}
	\Tree
	[.Assorbimento
		[.\node(am3){Ammine III${}^\circ$};
			Transcutaneo
		]
		[.{Ammine IV${}^\circ$}
			\node(intestino){intestino};
			\node(occhio){occhio};
		]
	]
	\draw[drawarrow]
		(am3) to[out=0, in=180] (intestino)
		(am3) to[out=0, in=180] (occhio);
\end{tikzpicture}

\textsc{Antimuscarinici}

\begin{tikzpicture}
	\tikzset{level distance=150pt}
	\Tree
	[.{effetti}
		[.{SNC ($\text M_1$)} {effetto stimolante (-atropina +scopolamina) $\downarrow$tremore Parkinson\\ Parkinson \`e causato da un $\uparrow$ attivit\`a colinergica}
		]
		[.{occhio ($\text M_3$)} {$\uparrow$attivit\`a simpatica $\Rightarrow$ midriasi\\ (belladonna $\equiv$ occhi dilatati\\ incapacit\`a di adattamento\\ visione da vicino)}
		]
		[.{cuore ($\text M_2$)} {tachicardia, blocco vagale, $\downarrow$PR per $\uparrow$dromotropo}
		]
		[.{vasi ($\text M_2/\text M_3$)} incerta ]
		[.{apparato respiratorio ($\text M_3$)} {broncodilatazione, $\downarrow$secrezioni\\ (ma meglio i $\beta$-adrenergici)}
		]
		[.{gastrointestinale ($\text M_3$)} {$\downarrow$secrezioni salivali, minori su tutto il resto} ]
		[.{gh. sudoripare ($\text M_3$)} {soppressione termoregolazione\\ (febbre da atropina)} ]
	]
\end{tikzpicture}

\begin{tikzpicture}
	%\tikzset{level 1/.style={level distance=130pt}}
	\Tree
	[.{usi clinici} 
		[.\node[farmaco](atropina){atropina};
			{malattia di Parkinson}
			{esame oftalmico}
			{pre-operatorio\\ antilaringospasmo}
			{sincope vagale\\ da dolore infarto}
			{ulcera peptidica\\ in disuso}
		]
		[.\node[farmaco]{scopolamina};
			{chinetosi\\ mal di mare}
		]
		[.\node[farmaco]{inatroprio???};
			\node(bpco){BPCO};
		]
		[.\node[farmaco]{oxibutina};
			{tenismo urinario}
		]
		[.\node[farmaco]{pralidossina};
			[.{iperfunzione colinergica}
				{da organofosfati, gas nervino\\ o intossicazione funghi}
			]
		]
	]
\end{tikzpicture}

\begin{tikzpicture}
	\Tree
	[.{Effetti avversi} {febbre da atropina} tachicardia {vasodilatazione con\\ esantema da atropina\\ testa, collo, arti, tronco} ]
\end{tikzpicture}

\textsc{Ganglioplegici}

\begin{tikzpicture}
	\tikzset{level 2/.style={level distance=150pt}}
	\Tree
	[.effetti
		[.SNC {sedazione, tremore}
		]
		[.occhio {perdita accomodamento, effetto su pupilla incerto\\ per innervazione para e orto del m. sfintere}
		]
		[.cardiocirc {$\downarrow$pressione con ipotensione ortostatica marcata}
		]
		[.gasto {$\downarrow$motilit\`a}
		]
		[.urinario {ritardo nella minzione, problemi erezione e eiaculazione}
		]
	]
\end{tikzpicture}

\begin{tikzpicture}
	\tikzset{level 2/.style={level distance=150pt}}
	\Tree
	[.{usi clinici}
		[.\node[farmaco]{trimetafano};
			{ipotensivo nelle anestesie}
		]
		[.\node[farmaco]{tossina botulinica};
			{iniezione intravescicale\\ contro incontinenza}
		]
	]
\end{tikzpicture}

\textsc{Bloccanti neuromuscolari}

\begin{tikzpicture}
	\tikzset{level 3/.style={level distance=150pt}}
	\Tree
	[.{influenza sui muscoli}
		[.{bloccanti neuromuscolari}
			{per indurre paralisi\\ durante gli interventi}
		]
		[.spasmolitici
			[.{per ridurre dolore\\ in varie situazioni}
				{vedi cap. 27 sull'argomento\\ o poi se trattato\\(benzodiazepine, clonidina, antiepilettici, \\ dantrolene (anche contro ipertermia maligna),\\ tossina botulinica)}
			]
		]
	]
\end{tikzpicture}

\begin{tikzpicture}
	\Tree
	[.{bloccanti\\ neuromuscolari}
		[.{non depolarizzanti}
			[.{impediscono l'accesso dell'Ach\\ sul recettore $\text M_M$}
				{tubocuranina\\ (curaro)}
				\node[farmaco]{rocuronio};
			]
		]
		[.depolarizzanti
			[.{eccesso di Ach o simile}
				\node[farmaco]{succinilcolina\\(2 Ach legate tra loro)};
			]
		]
	]
\end{tikzpicture}

\begin{tikzpicture}
	\tikzset{level 2/.style={level distance=130pt}}
	\Tree
	[.Farmacocinetica
		[.assunzione EV ]
		[.metabolismo epatico ]
		[.degradazione \edge node[smallfont,yshift=-5pt,xshift=5em]{AchE (plasma)} node[smallfont,yshift=5pt,xshift=5em]{BuchE (fegato)}; {acido succinico + colina}
		]
	]		
\end{tikzpicture}

Una mutazione del gene che codifica la pseudocolinesterasi plasmatica rende alcuni pazienti pi\`u sensibili a metabolizzare la succinilcolina.

Il n. di dibucaina \`e un parametro per definire tali anomalie e dipende dal fatto che la dibucaina inibisce la pseudoAchE normale per l'80\% mentre l'inibizione \`e solo del 20\% in quella modificata.

\begin{tikzpicture}
	\tikzset{frontier/.style={distance from root=300pt}}
	\Tree
	[.funzionamento
		[.{non depolarizzanti} {stimolo tetanico} ]
		[.depolarizzanti
			[.{fase 1: depolarizzazione} fascicolazione ]
			[.{fase 2: desensibilizzazione} {stimolo tetanico} ]
		]
	]
\end{tikzpicture}

\begin{tikzpicture}
	\Tree
	[.{sequenza tetanica}
		[.{da muscoli piccoli a grandi}
			{m. occhio}
			{m. facciali}
			{m. arti}
			{faringe}
			\node(diaframma){diaframma};
		]
	]
	\node[below right=10pt and 10pt of diaframma] (da) {};
	\node[above=12em of da] (db) {};
	\draw[drawarrow] (da) -- (db) node[midway,above,sloped] {\tiny recupero sequ. inversa};
\end{tikzpicture}

\begin{tikzpicture}
		\tikzset{level distance=150pt}
	\Tree
	[.{effetti avversi}
		iperkalinemia
		{dolore muscolare post operatorio\\ (depolarizzanti)}
		{rilascio istamina e quindi ipotensione}
	]
\end{tikzpicture}

\subsection{Noradrenalina}

\begin{tikzpicture}
	\Tree
	[.noradrenalina 
		[.SNP
			[.{simpatiche postgangliari} 
				[.escluso
					{muscolatura vasi renali ($\text D_1$)}
					{ghiandole sudoripare (Ach)}
				]
			]		
		]
		[.SNC
			[.{$\alpha$ eccitatori/inibitori\\ recettoti $\beta$ inibitori}
				[.{$\uparrow$stato veglia,\\ $\alpha_2$ causano ipotensione}
					ponte
					{locus ceruleus}
					{midollo spinale}
				]
			]
		]
	]
\end{tikzpicture}

\begin{tikzpicture}
	\begin{scope}
	\tikzset{level distance=90pt,level 3/.style={level distance=150pt},
	level 2/.style={level distance=130pt},level 4/.style={level distance=60pt}}
	\Tree 
	[.Sintesi 
		[.Tirosina  \edge node[smallfont,yshift=-5pt,xshift=5.5em]{tirosin--idrossilasi} node[smallfont,yshift=5pt,xshift=5.5em]{tappa limitante}; 
			[.{L-Dopa} \edge node[smallfont,yshift=-5pt,xshift=6em]{DOPA decarbossilasi};
				[.\node[farmaco]{dopamina}; \node[dummyc]{};]
			]
		]
	]
	\end{scope}
	\begin{scope}[yshift=-3em,xshift=1em]
	\tikzset{level distance=80pt}
	\Tree
	[.\node[dummyc]{}; 
		[.\node[farmaco]{noradrenalina}; \node[farmaco]{adrenalina};]
	]	
	\end{scope}				
	
\end{tikzpicture}

\begin{tikzpicture}
	\tikzset{level distance=130pt}
	\Tree 
	[.Degradazione
		[.{MAO (mono-ammino ossidasi)\\ in fegato e cellule ?????} ]
		[.{COMT (catecolo O-metiltransferasi)\\ nei neuroni} ]
	]
\end{tikzpicture}

\begin{tikzpicture}
	\tikzset{level 2/.style={level distance=150pt}}
	\Tree	
	[.{Recettore adrenergico\\ (metabotropo \\ a proteine G)} 
		[.{$\alpha$}
			[ .{$\alpha_1\,\text{G}_{\text{q}} \uparrow\text{IP}_3, \uparrow\text{Ca}^{2+}$ (postsinaptiche muscolo liscio)} ]
			[ .{$\alpha_2\,\text{G}_{\text{i}} \downarrow\text{cAMP}$ (presinaptiche muscolo liscio)} ]
		]
		[.{$\beta$}
			[.{$\beta_1\,\text{G}_{\text{s}} \uparrow\text{cAMP}$ (postsinaptiche cuore, adipociti,\\ iuxaglomerulare, epitelio corpi ciliari)} ]
			[.{$\beta_2\,\text{G}_{\text{s}} \uparrow\text{cAMP}$ (postsinaptiche muscolo liscio e\\ cuore dove qualche volta sono \ce{G_i} inibitorie)} ]
			[.{$\beta_3\,\text{G}_{\text{s}} \uparrow\text{cAMP}$ (postsinaptiche cuore, adipociti, vescica)} ]		
		]
	]
\end{tikzpicture}

\begin{tabular}{|c|c|c|c|}
\hline 
\textbf{Organo} & \textbf{Tipo} & \textbf{Recettore} & \textbf{Azione} \\ 
\hline\hline 
M. radiale & simpatico & $\alpha_1$ & costrizione \\ 
\hline 
M. circolare & parasimpatico & M${}_3$ & costrizione pupilla \\ 
\hline 
M. ciliare & simpatico & $\beta$ & rilasciamento \\ 
\hline 
M. ciliare & parasimpatico & M${}_2$ & contrazione \\ 
\hline 
Nodo SA & simpatico & $\beta_1\beta_2$ & accellerazione \\ 
\hline 
Nodo SA & parasimpatico & M${}_2$ & rallentamento \\ 
\hline 
Forza contrazione & simpatico & $\beta_1\beta_2$ & aumento \\ 
\hline 
Forza contrazione & parasimpatico & M${}_2$ & diminuzione \\ 
\hline 
vasi muscolari & simpatico & $\beta$ & rilasciamento \\ 
\hline 
muscolo gastrointestinale & simpatico & $\alpha_2\beta_2$ & rilasciamento \\ 
\hline 
muscolo gastrointestinale & parasimpatico & M${}_3$ & contrazione \\ 
\hline 
sfinteri gastrointestinali & simpatico & $\alpha_1$ & contrazione \\ 
\hline 
sfinteri gastrointestinali & parasimpatico & M${}_3$ & rilasciamento \\ 
\hline 
\end{tabular} 

\subsubsection{Simpaticomimetici}

\begin{tikzpicture}
	\tikzset{frontier/.style={distance from root=350pt},level 2/.style={level distance=150pt}}
	\Tree
	[.simpaticomimetici
		[.diretta 
			[.{interazione con i recettori} 
				\node[farmaco]{adrenalina\\
				felilefrina\\
				clonidina\\
				dobutamina\\
				salbutamolo};
			]
		]
		[.indiretta 
			[.{rilascio catecolamine immagazzinate} ]
			[.{riduzione clearance della noradrenalina} ]
			[.{inibizione della ricaptazione\\ della noradrenalina} ]
			[.{inibizione del catabolismo\\ enzimatico via blocco MAO e COMT} ]
		]
		[.miste
			\node[farmaco]{efedrina};
		]
	]
\end{tikzpicture}

\begin{tikzpicture}
	\tikzset{frontier/.style={distance from root=350pt}}
	\Tree
	[.effetti
		[.diretti
			[.{dipendono da} 
		 		{vie somministrazione} 
		 		{selettivit\`a per i sottotipi recettoriali}
		 		{espressione dei sottotipi nei tessuti}
		 	]
		]
		[.indiretti {effetto proporzionale all'attivazione del simpatico}
		]
	]
\end{tikzpicture}

\begin{tikzpicture}
	\Tree
	[.{Eliminazione da fessura}
		[.ricaptazione {NET (90\%)} ]
		[.metabolismo {COMT (10\%)} ]
	]
\end{tikzpicture}

\begin{tikzpicture}
	\Tree
	[.{chimica dei\\ simpaticomimetici}
		[.catecolamine
			{adrenalina\\
			noradrenalina\\
			dopamina}
		]
		[.{non catecolamine}
			{fenilefrina\\ efedrina\\ anfetamina}
		]
	]
\end{tikzpicture}

\begin{tikzpicture}
	\setatomsep{2em}
	\schemestart
	\chemname{\chemfig[][scale=0.8]{[:30]*6(-=-=(-OH)-(-OH)=-)}}{Catecolo}\quad
	\chemname{\chemfig[][scale=0.8]{[:30]*6(-=(-CH([6]-OH)-CH_2-NH_2)-=(-OH)-(-OH)=-)}}{Noradrenalina}
	\chemname{\chemfig[][scale=0.8]{[:30]*6(-=(-CH([6]-OH)-CH_2-NH-CH_3)-=(-OH)-(-OH)=-)}}{Adrenalina}
	\schemestop
\end{tikzpicture}

\begin{tikzpicture}
	\setatomsep{2em}
	\schemestart
	\chemname{\chemfig[][scale=0.8]{[:30]*6(-=(-CH([6]-OH)-CH_2-NH-CH_3)-=(-OH)-=-)}}{Fenilefrina}\quad
	\chemname{\chemfig[][scale=0.8]{[:30]*6(-=(-CH([6]-OH)-CH([6]-CH_3)-NH-CH_3)-=-=-)}}{Efedrina}\quad
	\chemname{\chemfig[][scale=0.8]{[:30]*6(-=(-CH_2-CH([6]-CH_3)-NH2)-=-=-)}}{Anfetamina}
	\schemestop
\end{tikzpicture}

Le catecolamine sono degradate da COMT a livello intestinale e epatico per cui l'assorbimento per os \`e praticamente nulla.

L'assenza di uno o di ambedue i gruppi \chemfig{-[,.5]OH} ne aumenta la disponibilit\`a per os.

La metilazione sul primo carbonio a sx del gruppo ammino, comporta un'azione mista dei farmaci come nell'efedrina e l'anfetamina che hanno azione diretta e indiretta e quindi dipendono anche dalla presenza del neurotrasmettitore.

\begin{tikzpicture}
	\tikzset{level 3/.style={level distance=130pt}}
	\Tree
	[.{effetti}
		[.{occhio}
			[.$\alpha_1$ {contrazione muscolo pupillare\\ con dilatazione della pupilla} ]
		]
		[.{cuore}
			[.$\alpha_1$ {inotropo+, porta a ipertrofia} ]
			[.$\beta_1$ {effetto cronotropo+ e inotropo+} ]
		]
		[.{sistema circolatorio}
			[.$\alpha_1$ {contrazione vasi} ]
			[.$\alpha_2$ {aggregazione piastrinica} ]
			[.$\beta_2$ {rilasciamento vasi} ]
		]			
		[.{muscolo liscio}
			[.{$\alpha_1,\alpha_2$ endogena o IV} contrazione ]
			[.{$\alpha_2$ per os} {riduzione del tono simpatico per accumulo SNC} ]
			[.{$\beta_2$} {rilasciamento della muscolatura} ] 
		]
		[.{respiratorio}
			[.{$\beta_2$} {rilasciamento della muscolatura\\ liscia bronchiale$\Rightarrow$broncodilatazione} ] 
		]
		[.{gastointestinale} 
			[.{$\alpha, \beta$} {rilasciamento muscolatura} ]
		]
		[.{rene}
			[.$\beta_1$ {rilascio di renina} ]
		]
		[.{metabolismo}
			[.{$\beta_2$} {promozione della iperglicemia} ]
			[.{$\beta_3$} {lipolisi e quindi iperlipidemia} ]
		]
		[.{sistema immunitario}
			[.{$\beta$} {$\uparrow$lifociti, $\uparrow$killing, $\uparrow$citochine} ]
		]
	]
\end{tikzpicture}

\begin{tikzpicture}
	\tikzset{level 2/.style={level distance=130pt},level 3/.style={level distance=130pt}}
	\Tree
	[.{usi clinici\\(diretti)}
		[.\node[farmaco]{adrenalina\\($\alpha,\beta$)};
			[.vasocostrizione
				{prolungamento anestetici locali}
				{shock anafilattico}
			]
			[.{stimolazione cardiaca}
				{arresto cardiaco}
			]
		]
		[.\node[farmaco]{fenilefrina\\($\alpha_1$)};
			{$\downarrow$prurito}
			{esame retina (midriasi)}
			\node(congestione){$\downarrow$congestione mucose};
		]
		[.\node[farmaco](oxi){oximetazolina\\($\alpha_2$)};
		]
		[.\node[farmaco]{clonidina\\($\alpha_2$)};
			{ipertensione nelle gestanti}
			{$\downarrow$vampate calore in menopausa}
			{disintossicazione da droghe}
			{stimolazione $\alpha_2$ vagale con vasocostrizione\footnotemark}
		]
		[.\node[farmaco]{metildopa\\($\alpha_2$)};
			{emergenze ipertensive\footnotemark}
		]
		[.\node[farmaco]{dobutamina\\($\beta_1$)};
			[.{$\uparrow$gittata cardiaca ma non tachicardia}
				{shock cardiogeno}
				{test farmacologico da sforzo\\ quando non si pu\`o usare\\ la cyclette}
			]
		]
		[.\node[farmaco]{salbutamolo\\($\beta_2$)};
			{asma come broncodilatatore}
			{inibizione parti prematuri\\ per relax muscolatura uterina}
		]
	]
	\draw[drawarrow](oxi) to[out=0,in=180] (congestione);
\end{tikzpicture}

\footnotetext{Per cui pu\`o dare anche un aumento della pressione e per questo non si usa nelle emergenge da ipertensione}
\footnotetext{\`E anche un inibitore della DOPA decarbossilasi per cui $\downarrow$dopamina.}

\begin{tikzpicture}
	\tikzset{level 3/.style={level distance=150pt}}
	\Tree
	[.{usi clinici\\(indiretti)}
		[.\node[farmaco]{anfetamine};
			[.{rilascio noradrenalina\\ e dopamina} {stimolante SNC}
			]
		]
		[.\node[farmaco]{tiramina};
			[.{simile a noradrenalina} {$\uparrow$pressione in pazienti trattati\\ con inibitori delle MAO\\ che dovrebbero degradarle.\\ Prodotti dal metabolismo\\ della tirosina. Non assumere\\ cibi come formaggi,...\\ in terapia da inibitori MAO}
			]
		]
		[.\node[farmaco]{cocaina};
			[.{blocco ricaptazione noradrenalina\\ e dopamina}
				{anestetico locale}
				{nel SNC il blocco della ricaptazione\\ della dopamina provoca piacere}
			]
		]
	]
\end{tikzpicture}

\begin{tikzpicture}
	\Tree
	[.{Effetti avversi}
		[.{Accentuazione dell'effetto\\ farmacologico}
			{vasocostrizione eccessiva}
			{aritmie}
			{infarto miocardio}
			{edema polmonare}
			{emorragie polmonari}
		]
	]
\end{tikzpicture}

\subsubsection{Inibitori dei recettori adrenergici}

\begin{tikzpicture}
	\Tree
	[.tipo
		[.{$\alpha$-bloccanti} 
		]
		[.{$\beta$-bloccanti}
		]
	]
\end{tikzpicture}

\textsc{$\alpha$-bloccanti}

\begin{tikzpicture}
	\tikzset{level distance=150pt}
	\Tree
	[.effetti
		[.{riduzione delle resistenze periferiche\\ e tachicardia riflessa\\ ($\alpha$)} 
			{$\downarrow$pressione} 
			{ipotensione posturale}
			{inversione dell'adrenalina\footnotemark}
		]
		[.{rilasciamento muscoli vescica, prostata\\($\alpha_1$)}
			{ritenzione urinaria\\ da iperplasia prosatic benigna}
		]
	]	
\end{tikzpicture}

\footnotetext{Attiva sia gli $\alpha$ che i $\beta_2$. Se si immette un $\alpha$-bloccante questo neutralizzer\`a l'effetto vasocostrittore dell'adrenalina lasciando la sola attivazione dei $\beta_2$ che quindi causer\`a una vasodilatazione da cui un azione inversa a quella usuale dell'adrenalina}

\begin{tikzpicture}
	\tikzset{level distance=80pt}
	\tikzset{level 3/.style={level distance=100pt}}
	\tikzset{level 4/.style={level distance=120pt}}
	\Tree
	[.{usi clinici}
		[.{non selettivi}
			[.\node[farmaco]{fenossibenzamina\\ fentolamina};
				[.{riduzione pressione soprattutto\\ in presenza di elevato tono simpatico\\ (passaggio da prono a ortopostura)} {feocromocitoma\\ pre intervento}
				]
			]
		]
		[.{$\alpha_1$}
			[.\node[farmaco]{prazosina};
				{antiipertensivo}
				{ritenzione urinaria\\ da iperplasia prosatic benigna}
			]
			[.\node[farmaco]{labetalolo}; {blocca principalmente i $\beta$.\\ Descritto in seguito}
			]
		]
		[.{$\alpha_2$} {solo sperimentali}
		]
	]
\end{tikzpicture}

\textsc{$\beta$-bloccanti}

\begin{tikzpicture}
	\tikzset{level 2/.style={level distance=120pt}}
	\tikzset{level 3/.style={level distance=140pt}}
	\Tree
	[.effetti
		[.cuore
			[.{inibizione sistema renina-angiotensina} 
				{$\downarrow$pressione arteriosa} 
			]
			[.{inibizione recettori adrenergici cardiaci}
				{effetto inotropo-, cronotopo-}
				{allungamento PR}
			]
		]
		[.polmone
			[.{blocco $\beta_2$ nella\\ muscolatura liscia bronchiale}
				{BPCO (soprattutto se associate\\ a cardiopatie ischemiche)}
			]
		]
		[.occhio		
			[.{riduzione della produzione\\ di umor acqueo} {glaucoma}
			]
		]
		[.{blocco dei canali Na${}^+$} {azione anestetica locale}
		]
	]	
\end{tikzpicture}

\begin{tikzpicture}
	\tikzset{level 1/.style={level distance=120pt}}
	\tikzset{level 2/.style={level distance=150pt}}
	\Tree
	[.{usi clinici}
		[.\node[farmaco]{propanololo};
			{angina pectoris}
			{infarto del miocardio}
			{aritmie}
			{insufficienza cardiaca}
			{ipertiroidismo}
			{tempesta tiroidea}
		]
		[.\node[farmaco]{metoprololo\\ atenololo};
			{emicrania}
			{tremore muscolare}
			{stati d'ansia}
		]
		[.\node[farmaco]{labetalolo\\ carvedilolo\\(sono sia $\alpha$ che $\beta$};
			{feocromocitoma}
			{feoc. + scompenso + ipertensione}
		]
		[.\node[farmaco]{timololo}; 
			{glaucoma}
		]
	]
\end{tikzpicture}


\subsection{Dopamina}

Ricorda anche la dopamina \`e una catecolamina quindi anche i recettori dopaminergici sono recettori adrenergici

\begin{tikzpicture}
	\Tree
	[.SNC
		[.{corpo striato}
			[.{via nigrostriale}
				[.{dalla sostanza nera\\ al corpo striato}
					{via deficitaria nel parkinson}
				]
			]
		]
		[.{sistema limbico}
			[.{via mesolimbica}
				{da mesencefalo\\ a nucleo accumbens}
			]
		]
		[.{amigdala e corteccia\\ prefrontale}
			{via della gratificazione}
		]
		[.{ipotalamo} 
			[.{via tuberoinfundibolare}
				{da ipotalamo ventrale\\ a ipofisi}
			]
		]
	]
\end{tikzpicture}

\begin{tikzpicture}
	\tikzset{level distance=130pt}
	\Tree
	[.{Recettore dopaminergico}
		[.{$\text D_1$, $\text D_5$, eccitatorio, $\uparrow$cAMP} {cervello (striato e ipotalamo),\\ muscolatura vasi rene} 
		]
		[.{$\text D_2$, inibitorio, apertura canali $\text K^+$} { muscolatura liscia} 
		]
		[.{$\text D_3$, inibitorio, apertura canali $\text K^+$} {cervello (sistema limbico)} 
		]
		[.{$\text D_4$, inibitorio, apertura canali $\text K^+$} {cervello, sistema cardio vascolare} 
		]
	]
\end{tikzpicture}

\begin{tikzpicture}
	\Tree
	[.{$\uparrow$dopamina}
		{effetto stereotipato}
		{schizzofrenia}
		{$\uparrow$vomito e nausea}
		{$\uparrow$GH}
	]
\end{tikzpicture}

\subsection{Serotonina (5-idrossitriptamina)}

\begin{tikzpicture}
	\Tree
	[.SNC
		ponte
		{midollo allungato}
		{nuclei del rafe}
	]
\end{tikzpicture}

\begin{tikzpicture}
	\tikzset{level 3/.style={level distance=130pt}}
	\Tree
 	[.Recettore
 		[.\ce{5-HT_1}
 			[.SNC {GpCR,$\downarrow$cAMP, inibizione presinaptica} ]
 		]
 		[.\ce{5-HT_2}
 			[.{muscolo liscio\\ piastrine, corteccia e\\ sistema limbico} {$\uparrow$IP3, DAG, GpCR} ]
 		]
 		[.\ce{5-HT_3}
 			[.{SNP (nocicettori\\ neuroni enterici)} {canale ionico stimolatore} ]
 		]
 		[.\ce{5-HT_4}
 			[.{SNC,vescica\\ cuore, striato\\ e cervelletto} {$\uparrow$cAMP,GpCR,eccitazione} ]
 		]
 		[.\ce{5-HT_5}
 			{ippocampo}
 		]
 		[.\ce{5-HT_7}
 			[.{regolazione termica\\ ed endocrina}
 				corteccia
 				ippocampo
 				talamo
 			]
 		]
 	]
\end{tikzpicture}

\begin{tikzpicture}
	\tikzset{level distance=80pt,level 3/.style={level distance=130pt},
	level 2/.style={level distance=130pt}}
	\Tree 
	[.Sintesi 
		[.Triptofano  \edge node[smallfont,yshift=5pt,xshift=5.5em]{triptofano} node[smallfont,yshift=-5pt,xshift=5.5em]{idrossilasi} ; 
			[.{5-idrossitriptofano} \edge node[smallfont,yshift=5pt,xshift=5.5em]{amminoacido} node[smallfont,yshift=-5pt,xshift=5.5em]{decarbossilasi}; 5-HT 
			]
		]
	]	
\end{tikzpicture}

\begin{tikzpicture}
	\tikzset{level distance=130pt}
	\Tree 
	[.Degradazione {MAO (mono-ammino ossidasi)}
	]
\end{tikzpicture}

\begin{tikzpicture}
	\tikzset{level distance=160pt}
	\Tree
	[.Effetti {piastrine: aggregazione} {terminazioni nervose: dolore} {SNC: eccitatorio 5-HT4,\\ inibitorio 5-HT1} {vasi: costrizione} {gastroenterico: attivazione secrezione\\ e peristalsi} ]
\end{tikzpicture}

\subsection{Neurotrasmettitori purinici}

\begin{tikzpicture}
	\tikzset{level 2/.style={level distance=130pt}}
 	\Tree
 	[.{Neurotrasmettitori\\ purinici} 
 		[.ATP {Aumento della permeabilit\`a di membrana} ]
 		[.Adenosina {vasodilatatore tranne che nel rene} {inibizione dell'aggreg. piastrinica} {blocco della conduzione AV}
 		]
 	]
\end{tikzpicture}

\subsection{Monossido d'azoto (NO)}

\begin{tikzpicture}
	\Tree
	[.Tipi
		[.iNOS {prodotto dai macrofagi tramite IF$\gamma$}
		]
		[.eNOS {endotelio e piastrine}
		]
		[.nNOS 
			[.neuroni
				cervello
				ippocampo
			]	
		]
	]
\end{tikzpicture}

\begin{tikzpicture}
	\Tree
	[.{$\uparrow$\ce{nNOS}}
		{??? di $\uparrow$\ce{Ca^2+}}
		{favorisce eventi ischemici}
		{neurogenerazione}
		{parkinson}
		{demenza senile}
	]
\end{tikzpicture}

\begin{tikzpicture}
	\tikzset{level distance=160pt}
	\Tree
	[.Causa {vasodilatazione} {inibizione dell'aggregazione piastrinica} {plasticit\`a sinaptica} {difesa da cellule neoplastiche, batteri, parassiti} ]
\end{tikzpicture}

Per via inalatoria $\downarrow$shunt, $\downarrow$broncocostrizione, $\downarrow$ipertensione polmonare e quindi utile anche nella cura dell'asma.

Utile nel trattamento delle malattie neurovegetative e shock settico dove aumenta e nell'ateorscelosi e ipercolesterolemia dove diminuisce.

\subsection{L-glutammato}

Neurotrasmettitore ubiquitario eccitatorio del SNC

\begin{tikzpicture}
	\Tree
	[.sintesi
		[.glucosio
			[.{ciclo di Krebs}
				{glutammato}
			]
		]
		[.neurone \edge[<->] node {};
			[.glutammato \edge[<->] node {};
				[.{cellule della glia\\(astrocita)} \edge[<->] node {};
					glutammina
				]
			]
		]
	]
\end{tikzpicture}

\begin{tikzpicture}
	\Tree
	[.rilascio
		[.vescicolare \edge node[smallfont,above,xshift=+4em] {\ce{Ca^2+}dip.};
			esocitosi
		]
	]
\end{tikzpicture}

\begin{tikzpicture}
	\Tree
	[.recettori
		[.ionotropi
			[.ANPA
				[.ubiquitari	
					{trasmissione sinaptica veloce}
				]
			]
			[.{KA (kainato)}
				[.{ippocampo, cervelletto\\ e midollo spinale}
					{trasmissione sinaptica veloce}
				]
			]
			[.NMDA
				[.ubiquitari
					{plasticità sinaptica}
				]
			]
		]
		[.metabotropi
			[.{\ce{G_q} ($\uparrow$\ce{IP_3},$\uparrow$DAG,$\uparrow$\ce{Ca^2+}}
				{modulazione a\\ medio/lungo termine}
			]
		]
	]
\end{tikzpicture}

\begin{tikzpicture}
	\Tree
	[.PdA
		[.{bassa frequenza}
			[.AMPA
				attivazione
			]
			[.NMDA
				{bloccato da \ce{Mg^2+}}
			]
		]
		[.{alta frequenza}
			[.AMPA
				attivazione
			]
			[.NMDA
				[.{espulsione \ce{Mg^2+}} \edge node[smallfont,above,xshift=3.6em] {+glicina};
					[.\node(ca){$\uparrow$\ce{Ca^2+}\\ Long Term Potentiation\\ plasticità sinaptica};
					]
				]
			]
			[.\ce{G_q}
				\node(a){attivazione};
			]
		]
	]
	\draw[drawarrow] (a) to[out=0,in=180] (ca);
\end{tikzpicture}

\begin{tikzpicture}
	\Tree
	[.utilizzo
		[.{inefficacia causa\\ ubiquità del recettore}
			{lesioni traumatiche}
			epilessia
			Alzheimer
		]
	]
\end{tikzpicture}

\begin{tikzpicture}
	\Tree
	[.farmaci
		[.agonisti
			[.????
				[.{$\uparrow$AMPA}
					{$\uparrow$memoria}
					{$\uparrow$capacità cognittive}
				]
			]
		]
		[.antagonisti
			[.{recettore NMDA}
				[.\node[farmaco]{memantina};
					{malattie neurovegetative}
				]
				[.\node[farmaco]{ketamina};
					{anestetico dissociativo}
				]
			]
			[.{recettore glicina in NMDA}
				\node[farmaco]{acido chinuretico};
			]
			[.AMPA
				{causano depressione generale\\ del SNC e crisi respiratorie}
			]
			[.metabotropo
				{usi futuri}
			]
		]
	]
\end{tikzpicture}


\subsection{GABA (Acido $\gamma$--amminobutirrico)}

\begin{tikzpicture}
	\Tree
	[.GABA
		[.{inibitorio}
			[.{ubiquitario della\\ sostanza grigia}
				[.{nigrostriato}
				]
			]
		]
	]
\end{tikzpicture}

\begin{tikzpicture}
	\Tree
	[.Sintesi
		[.glutammato
			GABA
		]
	]
\end{tikzpicture}

\begin{tikzpicture}
	\Tree
	[.degradazione
		[.GABA
			[.{astrociti}
				{acido succinico}
			]
		]
	]
\end{tikzpicture}

Enzima GABA--transaminasi (o GABA amminotransferasi). Utile informazione relativamente al valproato (farmaco antiepilettico, vedi).

\begin{tikzpicture}
	\tikzset{level 3/.style={level distance=150pt}}
	\Tree
	[.recettori
		[.{\ce{GABA_a}}
			[.{accoppiato a canale \ce{Cl-}}
				{presinaptico: inibizione lenta}
				{postsinaptico: inibizione veloce}
			]
		]
		[.{\ce{GABA_b}}
			[.{Accoppiato a \ce{G_i}}
				{$\downarrow$adenilato ciclasi $\Rightarrow\uparrow$ingresso \ce{K+}}
			]
		]
	]
\end{tikzpicture}

\begin{tikzpicture}
	\Tree
	[.farmaci
		[.{\ce{GABA_a}}
			[.agonisti
				benzodiazepine
				barbiturici
				anestetici
			]
		]
		[.{\ce{GABA_b}}
			[.agonisti
				[.\node[farmaco]{bacoflen};
					{miorilassante usato\\ nella terapia del dolore}
				]
			]
			[.antagonisti
				[.\node[farmaco]{saclofen};
					{antiepilettico sperimentale}
				]
			]
		]
	]
\end{tikzpicture}

\subsection{GBH (Acido $\gamma$--idrossibutirrico)}

Proviene dalla sintesi del GABA. $\uparrow$rilascio GH, attiva le "vie della gratificazione", da euforia e disibinizione. Droga da strada.

\subsection{Melatonina}

\begin{tikzpicture}
	\Tree
	[.Sintesi
		[.{nella gh. pineale}
			{acetilazione di 5-HT}
		]
	]
\end{tikzpicture}

\begin{tikzpicture}
	\Tree
	[.recettori
		[.{associati a proteine G}
			{cervelletto}
			{???}
		]
	]
\end{tikzpicture}

\begin{tikzpicture}
	\Tree
	[.effetti
		{$\uparrow$ sonnolenza}
	]
\end{tikzpicture}

\begin{tikzpicture}
	\Tree
	[.utilizzo
		{jet lag}
	]
\end{tikzpicture}

\subsection{Glicina}

\begin{tikzpicture}
	\Tree
	[.recettore
		{canale \ce{Cl-} simil \ce{GABA_a}}
	]
\end{tikzpicture}

Nessun farmaco in uso agisce su questo recettore. Stricnina e tossina tetanica prevengono il rilascio di glicina

\newpage
	\section{Farmaci del sistema cardiovascolare e renale}

\subsection{Farmaci anti--ipertensivi}

\begin{tikzpicture}
	\Tree
	[.Anti-ipertensivi diuretici simpaticolitici vasodilatatori ]
\end{tikzpicture}

\begin{tikzpicture}
	\Tree
	[.{Diuretici\\ (capitolo ah hoc)}
		[.{Diuretici dell'ansa} \node[farmaco]{furosemide}; ]
		[.{Inibitori del simporto\\ \ce{Na+}-\ce{Cl-}} \node[farmaco]{tiazidici}; ]
		[. {Risparmiatori di \ce{K+}} \node[farmaco]{spironolattone}; ]
	]
\end{tikzpicture}

\begin{tikzpicture}
	\tikzset{level 3/.style={level distance=120pt}}
	\Tree
	[.Simpaticolitici
		[.{SNC}
			[.\node[farmaco]{$\alpha$-metildopa}; {Inibitore dopa-carbossilasi\\ emergenza ipertensiva \\ Da sedazione, tossicit\`a epatica\\ coombs positivo} ]
			[.\node[farmaco]{clonidina}; {Agonista $\alpha_2$. $\downarrow$noradrenalina\\ Usato in gravidanza \\ Da sonnolenza, depressione\\ $\downarrow$libido, secchezza fauci } ]
		]		
		[.{$\beta$--bloccanti}
			[.\node[farmaco]{propranololo}; {Usato in ipertensione, scompenso cardiaco, \\ aritmie, glaucoma. Produce $\downarrow$GC e renina. \\ Da affaticamento,$\downarrow$umore, insomnia, $\uparrow$glicemia, \\ alterazione assetto lipidico (i non ASI). \\ Interruzione improvvisa $\uparrow$infarto.} ]
		]		
		[.{$\alpha$--agonisti} \node[farmaco]{doxazosina}; ]
		[.{Misti $\alpha$/$\beta$}
			[.\node[farmaco]{labetalolo}; {ipertensione da feocromocitoma.\\ Da prurito intenso, $\downarrow$eiaculazione} ]
		]
	 ]
\end{tikzpicture}

\begin{tikzpicture}
	\tikzset{level distance=80pt, level 4/.style={level distance=100pt}}
	\Tree
	[.{Vasodilatatori}
		[.{diretti}
			[.{prevalentemente\\ arteriosi}
				[.{Inibitori IP3} \node[farmaco]{idralazina\\ (non pi\`u usato)}; ]
				[.{\ce{Ca^{2+}} antagonisti}  \node[farmaco]{nifedipina\footnotemark\\ (anche verapamil\\ e diltiazem\\ ma su cuore)}; ]
			]
			[.{arterovenosi} 
				[.{rilascio \ce{NO}} \node[farmaco]{nitroprussiato\footnotemark\\ nitroglicerina}; ]
			]
		]
		[.{indiretti}
			[.{ACE inibitori}
				[.\node[farmaco]{captopril\\ enalapril\\ fosinopril}; {Dilata arteriole e grandi vene. \\$\downarrow$pre/post carico. \\ Non inficia riflesso barocettivo\\ ne secrezione di aldosterone. \\ $\uparrow$bradichinina da tosse secca\\ e edema angioneurotico.} ]
			]
			[.{Antagonisti AT--1}
				[.{sartani} {Uso in ipertensione, ACC, \\ nefropatia diabetica\\ NO in gravidanza} ]
			]
		]
	]
\end{tikzpicture}

\footnotetext{Vedere farmaci angina}

\footnotetext{Vedere farmaci angina}

\newpage

\subsection{Farmaci nell'angina e infarto cardiaco}

\begin{tikzpicture}
	\Tree
	[.{angina\\ infarto} vasodilatatori simpaticomimetici ]
\end{tikzpicture}

\begin{tikzpicture}
	\tikzset{level distance=90pt, level 3/.style={level distance=130pt}}
	\Tree
	[.{Vasodilatatori}
		[.Nitrati
			[.\node[farmaco]{Isosorbide mononitrato}; {Duranta d'azione pi\`u lunga} ]
			[.\node[farmaco]{Nitroglicerina}; {Rilascio \ce{NO}, $\uparrow$cGMP, relax muscolatura lis.\\ Via sublinguale, transdermica, rapido assorbimento\\ grazie alla solubilit\`a lipidica}  ]	
		]
		[.{\ce{Ca^{2+}}  antagonisti}
			[.\node[farmaco]{verapamil\\ (diidropiridine)}; {$\downarrow$conduzione NSA. $\downarrow$ RVP} ]
			[.\node[farmaco]{diltiazem}; {$\downarrow$conduzione NSA. $\downarrow$ RVP} ]
			[.\node[farmaco]{nifedipina}; {$\updownarrow$conduzione NSA.  Possibile tachicardia riflessa\\ minori effetti cardiaci} ]
		]
	]
\end{tikzpicture}

\begin{tikzpicture}
	\tikzset{level distance=90pt, level 3/.style={level distance=130pt}}
	\Tree
	[.{Simpaticolitici}
		[.{$\beta$--bloccanti}
			[.\node[farmaco]{propranololo\footnotemark}; {$\downarrow$GC, $\downarrow$PA, $\downarrow$consumo \ce{O2} micardico} ]
		]
	]
\end{tikzpicture}

\footnotetext{vedi farmaci anti-ipertensivi}

\begin{tikzpicture}
	\node[chartnode,anchor=west] at(0,0)(mlck){MLCK} node[chartnode,xshift=125pt] (mlckstar){MLCK${}^*$};
	\draw[drawarrow](mlck)[yshift=10pt]--node[smallfont,yshift=6pt,midway](Ca){\ce{Ca^{2+}}}
				node[chartnode,yshift=50pt,midway](CCa){Canali \ce{Ca^{2+}}}
				node[smallfont,yshift=-15pt,midway](camp){cAMP}
				node[chartnode,yshift=-60pt,midway](atp){ATP}(mlckstar);
	\draw[drawarrow] (mlckstar) [yshift=-10pt] -- (mlck);
	\draw[drawarrow] (CCa)-- node[midway](CCCa){} node[smallfont,xshift=5em](bloc){bloccanti canali} (Ca);
	\draw[drawarrow] (bloc)-- node[midway, below]{$\ominus$} (CCCa);
	\draw[drawarrow] (atp)-- node[midway](catp){} node[smallfont,xshift=5em](beta){$\beta$-bloccanti} (camp);
	\draw[drawarrow] (beta)-- node[midway, above]{$\oplus$} (catp);
	
	\node[chartnode,below right=1em and 2em of mlckstar](mlc){MLC};
	\node[chartnode,right=50pt of mlc](mlcstar){MLC${}^*$} node[right=3pt of mlcstar](+){+}
		node[chartnode, right=3pt of +](actina){actina};
	\node[chartnode,above=20pt of actina](contrazione){contrazione};
	\node[chartnode,below=20pt of mlc](relax){relax};
	\draw[drawarrow] (mlc) [yshift=10pt]-- node[midway](a){}(mlcstar);
	\draw[drawarrow] (mlcstar) [yshift=-10pt]-- 
		node[smallfont,midway,yshift=-6pt](cgmp){cGMP} 
		node[chartnode,yshift=-57pt,midway](gtp){GTP}
		(mlc);
	\draw[drawarrow] (actina) -- (contrazione);
	\draw[drawarrow] (mlc) -- (relax);
	\draw[drawarrow] (mlckstar) -| (a);
	\draw[drawarrow] (gtp)-- node[midway](cgtp){} 
		node[smallfont,xshift=10pt](gcstar){GC${}^*$}
		node[chartnode,xshift=100pt](gc){Guanil ciclasi} 
		(cgmp);
	\draw[drawarrow] (gc)-- node[midway](cgc){} node[midway, chartnode,yshift=-50pt](no){NO} (gcstar);
	\draw[drawarrow] (no)--node[midway, right]{$\oplus$}(cgc);
	
	\node[smallfont,text width=12em,anchor=west] at(0,-4) {* $\equiv$ elemento attivato\\
	MLCK $\equiv$ Miosina Catena Leggera chinasi\\ MLC $\equiv$ Miosina Catena Leggera};
\end{tikzpicture}

\begin{tikzpicture}
	\Tree
	[.Angina 
		[.{ischemia cardiaca transitoria\\ senza danno al miocardio}
			{stabile}
			{instabile}
			{di prinzmetal}
			{silente}
			{cronica}
		]
	]		
\end{tikzpicture}

\begin{tikzpicture}
	\Tree
	[.Terapia
		comportamentale
		chirurgica
		[.farmacologica
			[.{vasodilatatori\\ \ce{NO} e \ce{Ca^{2+}} antagonisti}
				{per aumentare il flusso}
			]
			[.antiaggreganti {per evitare i trombi}
			]
			[.fibrinolitici {per distruggere i\\ trombi preesistenti}
			]
			[.{$\beta$--bloccanti} {per ridurre il fabbisogno energetico}
			]
			[.oppioidi {per ridurre il dolore}
			]
		]
	]
\end{tikzpicture}

\begin{tikzpicture}
	\Tree
	[.terapia
		[.stabile
			{nitrati organici}
			{$\beta$--bloccanti}
			{statina}
			{aspirina}
		]
		[.instabile
			{nitrati}
			{aspirina}
			{eparina}
		]
		[.variante
			{nitrati organici}
			{\ce{Ca^{2+}} antagonisti}
		]
	]	
\end{tikzpicture}

\subsubsection{Nitrati organici}

\begin{tikzpicture}
	\Tree
	[.{nitrati organici}
		{nitroglicerina}
		{isosorbide mononitrato}
	]
\end{tikzpicture}

\begin{tikzpicture}
	\Tree
	[.effetti
		[.{diminuzione della richiesta\\ di O${}_2$}
			{$\downarrow$ritorno venoso}
			{$\downarrow$volume intracardiaco}
			{$\downarrow$pressione arteriosa}
		]
		[.{scompasa spasmo arterioso} {vasodilatazione arterie\\ coronariche}
		]
	]
\end{tikzpicture}

\begin{tikzpicture}
	\Tree
	[.{effetti collaterali}
		{tachicardia riflessa}
		{aumento riflesso contrattile}
		{riduzione del tempo di perfusione\\ diastolica indotta da tachicardia}
	]
\end{tikzpicture}

\subsubsection{Calcio antagonisti}

\begin{tikzpicture}
	\Tree
	[.tipo
		[.L
			[.{Corrente lunga}
				[.Verapamil
					{cuore}
					{muscolo scheletrico\\ e liscio}
					{neuroni}
					{ossa}
				]
			]
		]
		[.T
			[.{Corrente breve}
				[.Flunarizina
					{cuore}
					{neuroni}
				]
			]
		]
		[.N
			[.{Corrente breve} {neuroni} 
			]
		]
		[.P
			[.{Corrente lunga} {neuroni} 
			]
		]
		[.{Q/R}
			[.{Segnapassi} {neuroni} 
			]
		]
	]
\end{tikzpicture}

\begin{tikzpicture}
	\tikzset{level 2/.style={level distance=120pt}}
	\tikzset{frontier/.style={distance from root=350pt}}
	\Tree
	[.effetti
		[.{muscolo liscio}
			[.{arteriole + sensibili delle venule.\\ Quindi meno effetto di ipotensione ortostatica}
				\node[farmaco]{nifedipina};
			]
		]
		[.{miocardio}
			\node[farmaco]{varapamil\\ diltiazem};
		]
		[.{muscolo scheletrico}
			[.{nessun effetto\\ il \ce{Ca^{2+}} \`e intracellulare} ]
		]
	]
\end{tikzpicture}

\subsubsection{$\beta$--bloccanti}

\begin{tikzpicture}
	\Tree
	[.effetti
		{$\downarrow$frequenza cardiaca}
		{$\downarrow$pressione arteriosa}
		{$\downarrow$contrattilit\`a}
	]
\end{tikzpicture}

\begin{tikzpicture}
	\Tree
	[.{effetti indesiderati}
		{$\uparrow$volume telediastolico}
		{$\uparrow$tempo di eiezione}
		{insomnia}
		{sonni spiacevoli}
		{senso di affaticamento}
		{disfunzione erettile}
	]
\end{tikzpicture}

\begin{tikzpicture}
	\Tree
	[.controindicazioni
		asma
		{affezioni broncospastiche}
		{grave bradicardia}
		{blocco atriventricolare}
		{insufficienza ventricolare sinistra}
	]
\end{tikzpicture}

\subsection{Insufficienza cardiaca}

\begin{tikzpicture}
	\Tree
	[.{I.C.}
		[.{gittata insufficiente\\ a fornire \ce{O2}\\ a organismo}
			[.{i. sistolica}
				{$\downarrow$contrattilità}
				{$\downarrow$fraz. di eiezione}
			]
			[.{i. diastolica}
				{rigidità}
				{perdità di rilasciamento}
			]
		]
	]
\end{tikzpicture}

\begin{tikzpicture}
	\Tree
	[.{scopo del\\ trattamento}
		[.{fase stabile\\(cronica)}
			{$\downarrow$sintomi}
			{rallentare progressione}
		]
		[.{fase scompensata\\(acuta)}
			{ricondurre il paziente\\ alla fare stabile}
		]
	]
\end{tikzpicture}

\begin{tikzpicture}
	\tikzset{level 2/.style={level distance=150pt}}
	\Tree
	[.terapia
		[.{fase cronica}
			{antagonisti aldosterone}
			{ACE inibitori}
			{sartani}
			{$\beta$--bloccanti}
			{digitalici}
			\node(diuretici){diuretici};
			\node(vasodilatatori){vasodilatatori};
		]
		[.\node(acuta){fase acuta};
			{$\beta$--agonisti}
		]
	]
	\draw[drawarrow] (acuta) to[out=0,in=180] (diuretici);
	\draw[drawarrow] (acuta) to[out=0,in=180] (vasodilatatori);
\end{tikzpicture}

\begin{tikzpicture}
	\tikzset{level distance=80pt}
	\Tree
	[.\node(git){$\downarrow$gittata cardiaca};
		[.{$\downarrow$P.A.}
			[.{attivazione barocettori}
				[.\node(simpatico){$\uparrow$simpatico};
					[.{inotropo+} {rimodellamento}
					]
					[.{cronotopo+}
					]
				]
			]
		]
		[.{$\downarrow$flusso renale}
			[.{$\uparrow$renina}
				[.{$\uparrow$angII}
					\node(pre){$\uparrow$ pre--carico};
					[.\node(post){$\uparrow$post--carico};
						\node(fraz){$\downarrow$fraz. eiezione};
					]
				]
			]
		]
	]
	\draw[drawarrow] (simpatico) to[out=0,in=180] (post);
	\draw[drawarrow] (simpatico) to[out=0,in=180] (pre);
	\node[below=1em of fraz](a){};
	\draw[drawarrow] (fraz) --  (a.north) -| (git);
\end{tikzpicture}

\begin{tikzpicture}
	\tikzset{level distance=120pt}
	\Tree
	[.{rimodellamento causato da\\ ipertrofia per riattivazione\\ fattori di crescita}
		[.concentrico
			{da sovraccarico pressorio per \upa post--carico}
		]
		[.eccentrico
			{da sovraccarico volume per \upa pre--carico}
		]
		[.compensato {se raggio della cavità ventricolare,\\ massa ventricolo e volume cavità\\ sono rispettati}
		]
		[.scompensato 
			[.{se tali rapporti non sono rispettati}
				{evolve in\\ scompenso cardiaco}
			]
		]
	]
\end{tikzpicture}

\begin{tikzpicture}
	\tikzset{level distance=80pt}
	\Tree
	[.{funzionalità\\ cardiaca}
		[.pre--carico
			[.{pressione riempimento\\ ventricolo sx}
				[.{$\uparrow$I.C.}
					[.vasodilatatori	
						{nitrati organici}
					]						
				]
			]
		]
		[.post--carico
			[.{resistenze vasc.\\ sistemiche e\\ impedenza aortica}
				[.{$\uparrow$I.C.}
					[.{farmaci $\downarrow$tono\\ arteriolare}	
						{???}
					]						
				]
			]
		]
		[.contrattilità
			[.{$\downarrow$I.C.}
				[.{farmaci $\uparrow$inotropismo}	
					{???}
				]						
			]
		]
		[.frequenza 
			{$\uparrow$I.C. per compensazione}
		]
	]
\end{tikzpicture}

\begin{tikzpicture}
	\Tree
	[.farmaci
		[.\node[farmaco]{digitale/digossina\\ (inotropo+)};
			[.{inibizione \ce{Na+}/\ce{K+} ATPasi}
				[.{\upa\ce{Ca^2+} per\\ blocco NCX}
					{inotropo+}
				]
				[.{\dwa condutt. \ce{K+}}
					[.{\dwa durata PdA da cui\\ \upa PR, depressione\\ a cucchiaio ST}
					]
				]
			]
		]
		[.\node[farmaco]{dobutamina\\ (agonista $\beta_1$ selett.)};
			{\upa GC}
			{\dwa pre--carico}
		]
		[.\node[farmaco]{furosemide\\ (diuretico)};
			{\dwa P.A.}
			{\dwa pre--carico}
		]
		[.\node[farmaco]{captopril\\ elanapril (ACE inibitore)};
			{\dwa post--carico}
		]
		[.\node[farmaco]{losartan (antagonista AT-1)};
			{\dwa post--carico}
		]
		[.\node[farmaco]{carvedilolo\\ metoprololo ($\beta$--bloccanti)};
			{cronotopo-}
			{\dwa rimodellamento per\\ inibizione catecolamine}
		]
	]
\end{tikzpicture}

\begin{tikzpicture}
	\Tree
	[.{digitale +}
		[.\ce{K+}
			[.iper {\dwa effetti digitale}
			]
			[.ipo {\upa effetti digitale}
			]
		]
		[.\ce{Ca^2+}
			[.iper {\upa effetti digitale}
			]
			[.ipo {\dwa effetti digitale}
			]
		]
		[.\ce{Mn}
			[.iper {\dwa effetti digitale}
			]
			[.ipo {\upa effetti digitale}
			]
		]
	]
\end{tikzpicture}

\begin{tikzpicture}
	\tikzset{level 2/.style={level distance=150pt}}
	\Tree
	[.{effetti avversi}
		[.\node[farmaco]{digitale/digossina\\ (a dosi elevate)};
			{\upa aritmie, tachicardia, extrasistole\\ torsione di punta, FV}
		]
	]
\end{tikzpicture}

\newpage

	\section{Farmaci del sistema respiratorio}

\subsection{Asma}

\begin{tikzpicture}
	\tikzset{level 2/.style={level distance=130pt}}
	\tikzset{level 3/.style={level distance=130pt}}
	\Tree
	[.Asma
		[.{malattia infiammatoria\\ delle vie aeree}
			{infiammazione}
			[.{ostruzione bronchiale}
				{solitamente reversibile}
				{in alcuni casi irreversibile}
			]
			{iperreattività agli allergeni}
		]
	]	
\end{tikzpicture}

\begin{tikzpicture}
	\Tree
	[.sintomi
		{sibili respiratori}
		{dispnea}
		tosse
		{costrizione torace}
	]
\end{tikzpicture}

\begin{tikzpicture}
	\tikzset{level distance=80pt}
	\Tree
	[.fisiopatogenesi
		[.{ingesso allergene}
			[.{APC presentano\\ antigeni ai \ce{T_H_2}}
				[.{\ce{T_H_2} stimolano B\\ a produrre IgE}
					[.{IgE si legano\\ agli allergeni}
						\node[dummyc]{};
					]
				]
			]
		]
	]
	\begin{scope}[yshift=-4em]
	\Tree
	[.\node[dummyc]{};
		[.{Mastociti legano IgE}
			{fase immediata}
			{fase tardiva}
		]
	]
	\end{scope}
\end{tikzpicture}

\begin{tikzpicture}
	\tikzset{level 1/.style={level distance=75pt}}
	\tikzset{level 2/.style={level distance=90pt}}
	\tikzset{level 3/.style={level distance=90pt}}
	\tikzset{level 4/.style={level distance=80pt}}
	\Tree
	[.{fase immediata}
		[.{mastociti\\ liberano}
			[.istamina
				[.{liberazione \ce{Ca^2+} nel REL}
					{broncospasmo}
				]
			]
			[.{leucotreni/citochina}
				[.IL5
					[.{attivazione eosinofili}
						{danno tissutale}
						{edema}
						{congestione}
					]
					[.{fase tardiva}
					]
				]
				[.IL4
					{stimolo a produrre IgE}
				]
			]
			[.{fattori di crescita}
			]
		]
	]
\end{tikzpicture}

\begin{tikzpicture}
	\tikzset{level distance=140pt}
	\Tree
	[.{fase tardiva}
		{inspessimento della parete\\ con restringimento del lume}
		{flogosi}
		rimodellamento
		{$\uparrow$produzione muco}
	]
\end{tikzpicture}

tutto ciò causa iperresponsività bronchiale futura.

\begin{tikzpicture}
	\tikzset{level distance=140pt}
	\Tree
	[.{farmaci}
		[.{broncodilatatori\\ (a breve durata d'azione)}
		]
		[.{glucocorticosteroidi\\ (in aerosol)}
		]
		[.{broncodilatatori\\(a lunga durata d'azione)}
		]
		[.{metilxantine o\\ antagonisti dei leucotreni}
		]
		[.{corticosteroidi orali}
		]
		[.{anticorpi monoclonali anti--IgE}
		]
	]
	\begin{scope}[xshift=18em]
		\draw[drawarrow] (0,2) -> (0,-2);
		\node[text width=8em] at (2,0){step operativi via via che la malattia diventa più grave};
	\end{scope}
\end{tikzpicture}

\begin{tikzpicture}
	\tikzset{level 1/.style={level distance=150pt}}
	\Tree
	[.{broncodilatatori}
		[.{$\beta_2$--agonisti\\(1a linea)}
			[.{a breve durata}
				\node[farmaco]{\index{salbutamolo}salbutamolo};
				\node[farmaco]{\index{terbutalina}terbutalina};
			]
			[.{a lunga durata}
				\node[farmaco](sal){\index{salmeterolo}salmeterolo};
				\node[farmaco](for){\index{formoterolo}formoterolo};
			]
		]
		[.{metilxantine\\(2a linea)}
			\node[farmaco]{\index{teofillina}teofillina};
			\node[farmaco]{\index{teobromina}teobromina	(cioccolato)};
			{caffeina};
		]
		[.{antagonisti muscarinici\\ (raramente usato. Più per BPCO)}
			\node[farmaco]{\index{ipratropio}ipratropio};
			\node[farmaco]{\index{tiotropio}tiotropio};
		]
	]
\end{tikzpicture}


\begin{tikzpicture}
	\Tree
	[.{$\beta_2$--agonisti}
		[.meccanismo
			{stimolazione $\beta_2$ muscolo bronchiale\\ con dilatazione}
			{inibizione del rilascio\\ dei mediatori dai mastociti}
			{$\downarrow$essudato}
		]
		[.somministrazione
			{inalazione per evitare\\ effetti sistemici}
		]
		[.metabolismo
			{epatico/renale}
		]
		[.{effetti collaterali}
			tachicardia
			{tremore muscolare}
		]
	]
\end{tikzpicture}

\begin{tikzpicture}
	\tikzset{level 2/.style={level distance=150pt}}
	\Tree
	[.{metilxantine}
		[.meccanismo
			{inibizione fosfodiesterasi IV (PDE)\\ che degrada cAMP quindi\\$\uparrow$cAMP$\Rightarrow$miorilassamento$\Rightarrow$broncodilatazione}
		]
		[.somministrazione
			os
		]
		[.metabolismo
			{epatico P450/renale}
		]
		[.{effetti collaterali}
			tachicardia
			{tremore muscolare}
			insonnia
			{$\uparrow$motilità intestinale}
		]
	]
\end{tikzpicture}

\begin{tikzpicture}
	\Tree
	[.{glucocorticoidi\\(GC)}
		\node[farmaco]{\index{idrocortisone}idrocortisone\\(EV)};
		\node[farmaco]{\index{beclometasone}beclometasone\\(aerosol)};
		\node[farmaco]{\index{budesonide}budesonide\\(aerosol)};
	]
\end{tikzpicture}

\begin{tikzpicture}
	\tikzset{level 2/.style={level distance=150pt}}
	\Tree
	[.{glucocorticoidi\\(GC)}
		[.meccanismo
			{$\downarrow$citochne}
			{inibizione COX2$\Rightarrow$inibizione\\ eosinofili e basofili}
			{inibizione leucotreni}
			{$\downarrow$edema}
			{$\uparrow$recettori $\beta_2$}
		]
		[.somministrazione
			{aerosol/EV}
		]
		[.{effetti collaterali}
			{sindrome di Cushing}
			disfonia
			{candidosi orofaringea\\(mughetto)}
		]
	]
\end{tikzpicture}

I GC vengono dati alle partorienti con figlio prematuro per velocizzare la funzione del surfactante polmonare che innalza la tensione degli alveoli e evita il collasso polmonare.

\begin{tikzpicture}
	\Tree
	[.{inibitori di\\ rilascio dei mediatori}
		[.\node[farmaco]{\index{cromolin}cromolin};
			[.meccanismo
				{inibizione del rilascio\\ di leucotreni e istamina\\ dai mastociti}
			]
			[.somministrazione
				{aerosol via\\ inalatori nasali}
			]
			[.{indicazioni terapeutiche}
				{allergie alimentari}
				{riniti allergiche}
				{congiuntiviti}
				{asma (raro utilizzo)}
			]
		]
	] 
\end{tikzpicture}

\begin{tikzpicture}
	\tikzset{level 1/.style={level distance=130pt}}
	\tikzset{level 2/.style={level distance=130pt}}
	\Tree
	[.{antagonisti dei leucotreni}
		[.meccanismo
			[.{bloccanti dei recettori\\ per i leucotreni}
				\node[farmaco]{\index{zafilukast}zafilukast};
				\node[farmaco]{\index{montelukast}montelukast};
			]
			[.{inibitori della lipossigenasi}
				\node[farmaco]{\index{zileuton}zileuton};
			]
		]
		[.somministrazione
			os
		]
		[.{indicazioni terapeutiche}
			{broncospasmo da antigene,\\ esercizio fisico o aspirina}
		]
		[.{effetti collaterali}
			{vasculite sistemica (leucotreni)}
			{$\uparrow$transaminasi (lipossigenasi)}
		]
	]
\end{tikzpicture}

\begin{tikzpicture}
	\Tree
	[.{anticorpo monoclonale\\ anti--IgE}
		[.\node[farmaco]{\index{omalizumab}omalizumab};
			[.somministrazione EV
			]
			[.difetti costoso
			]
		]
	]
\end{tikzpicture}

\newpage
	\section{Farmaci anti--ipertensivi}

\begin{tikzpicture}
	\Tree
	[.Anti-ipertensivi diuretici simpaticolitici vasodilatatori ]
\end{tikzpicture}

\begin{tikzpicture}
	\Tree
	[.{Diuretici\\ (capitolo ah hoc)}
		[.{Diuretici dell'ansa} \node[farmaco]{furosemide}; ]
		[.{Inibitori del simporto\\ $\text{Na}^+\text{-Cl}^-$} \node[farmaco]{tiazidici}; ]
		[. {Risparmiatori di $\text K^+$} \node[farmaco]{spironolattone}; ]
	]
\end{tikzpicture}

\begin{tikzpicture}
	\tikzset{level 3/.style={level distance=120pt}}
	\Tree
	[.Simpaticolitici
		[.{SNC}
			[.\node[farmaco]{$\alpha$-metildopa}; {Inibitore dopa-carbossilasi\\ emergenza ipertensiva \\ Da sedazione, tossicit\`a epatica\\ coombs positivo} ]
			[.\node[farmaco]{clonidina}; {Agonista $\alpha_2$. $\downarrow$noradrenalina\\ Usato in gravidanza \\ Da sonnolenza, depressione\\ $\downarrow$libido, secchezza fauci } ]
		]		
		[.{$\beta$--bloccanti}
			[.\node[farmaco]{propranololo}; {Usato in ipertensione, scompenso cardiaco, \\ aritmie, glaucoma. Produce $\downarrow$GC e renina. \\ Da affaticamento,$\downarrow$umore, insomnia, $\uparrow$glicemia, \\ alterazione assetto lipidico (i non ASI). \\ Interruzione improvvisa $\uparrow$infarto.} ]
		]		
		[.{$\alpha$--agonisti} \node[farmaco]{doxazosina}; ]
		[.{Misti $\alpha$/$\beta$}
			[.\node[farmaco]{labetalolo}; {ipertensione da feocromocitoma.\\ Da prurito intenso, $\downarrow$eiaculazione} ]
		]
	 ]
\end{tikzpicture}

\begin{tikzpicture}
	\tikzset{level distance=80pt, level 4/.style={level distance=100pt}}
	\Tree
	[.{Vasodilatatori}
		[.{diretti}
			[.{prevalentemente\\ arteriosi}
				[.{Inibitori IP3} \node[farmaco]{idralazina\\ (non pi\`u usato)}; ]
				[.{Ca${}^{2+}$ antagonisti}  \node[farmaco]{nifedipina\footnotemark\\ (anche verapamil\\ e diltiazem\\ ma su cuore)}; ]
			]
			[.{arterovenosi} 
				[.{rilascio NO} \node[farmaco]{nitroprussiato\footnotemark\\ nitroglicerina}; ]
			]
		]
		[.{indiretti}
			[.{ACE inibitori}
				[.\node[farmaco]{captopril\\ enalapril\\ fosinopril}; {Dilata arteriole e grandi vene. \\$\downarrow$pre/post carico. \\ Non inficia riflesso barocettivo\\ ne secrezione di aldosterone. \\ $\uparrow$bradichinina da tosse secca\\ e edema angioneurotico.} ]
			]
			[.{Antagonisti AT--1}
				[.{sartani} {Uso in ipertensione, ACC, \\ nefropatia diabetica\\ NO in gravidanza} ]
			]
		]
	]
\end{tikzpicture}

\footnotetext{Vedere farmaci angina}

\footnotetext{Vedere farmaci angina}

\newpage

\section{Farmaci nell'angina e infarto cardiaco}

\begin{tikzpicture}
	\Tree
	[.{angina\\ infarto} vasodilatatori simpaticomimetici ]
\end{tikzpicture}

\begin{tikzpicture}
	\tikzset{level distance=90pt, level 3/.style={level distance=130pt}}
	\Tree
	[.{Vasodilatatori}
		[.Nitrati
			[.\node[farmaco]{Isosorbide mononitrato}; {Duranta d'azione pi\`u lunga} ]
			[.\node[farmaco]{Nitroglicerina}; {Rilascio NO, $\uparrow$cGMP, relax muscolatura lis.\\ Via sublinguale, transdermica, rapido assorbimento\\ grazie alla solubilit\`a lipidica}  ]	
		]
		[.{Ca${}^{2+}$ antagonisti}
			[.\node[farmaco]{verapamil\\ (diidropiridine)}; {$\downarrow$conduzione NSA. $\downarrow$ RVP} ]
			[.\node[farmaco]{diltiazem}; {$\downarrow$conduzione NSA. $\downarrow$ RVP} ]
			[.\node[farmaco]{nifedipina}; {$\updownarrow$conduzione NSA.  Possibile tachicardia riflessa\\ minori effetti cardiaci} ]
		]
	]
\end{tikzpicture}

\begin{tikzpicture}
	\tikzset{level distance=90pt, level 3/.style={level distance=130pt}}
	\Tree
	[.{Simpaticolitici}
		[.{$\beta$--bloccanti}
			[.\node[farmaco]{propranololo\footnotemark}; {$\downarrow$GC, $\downarrow$PA, $\downarrow$consumo O${}_2$ micardico} ]
		]
	]
\end{tikzpicture}

\footnotetext{vedi farmaci anti-ipertensivi}

\begin{tikzpicture}
	\node[chartnode,anchor=west] at(0,0)(mlck){MLCK} node[chartnode,xshift=125pt] (mlckstar){MLCK${}^*$};
	\draw[drawarrow](mlck)[yshift=10pt]--node[smallfont,yshift=6pt,midway](Ca){Ca${}^{2+}$}
				node[chartnode,yshift=50pt,midway](CCa){Canali Ca${}^{2+}$}
				node[smallfont,yshift=-15pt,midway](camp){cAMP}
				node[chartnode,yshift=-60pt,midway](atp){ATP}(mlckstar);
	\draw[drawarrow] (mlckstar) [yshift=-10pt] -- (mlck);
	\draw[drawarrow] (CCa)-- node[midway](CCCa){} node[smallfont,xshift=5em](bloc){bloccanti canali} (Ca);
	\draw[drawarrow] (bloc)-- node[midway, below]{$\ominus$} (CCCa);
	\draw[drawarrow] (atp)-- node[midway](catp){} node[smallfont,xshift=5em](beta){$\beta$-bloccanti} (camp);
	\draw[drawarrow] (beta)-- node[midway, above]{$\oplus$} (catp);
	
	\node[chartnode,below right=1em and 2em of mlckstar](mlc){MLC};
	\node[chartnode,right=50pt of mlc](mlcstar){MLC${}^*$} node[right=3pt of mlcstar](+){+}
		node[chartnode, right=3pt of +](actina){actina};
	\node[chartnode,above=20pt of actina](contrazione){contrazione};
	\node[chartnode,below=20pt of mlc](relax){relax};
	\draw[drawarrow] (mlc) [yshift=10pt]-- node[midway](a){}(mlcstar);
	\draw[drawarrow] (mlcstar) [yshift=-10pt]-- 
		node[smallfont,midway,yshift=-6pt](cgmp){cGMP} 
		node[chartnode,yshift=-57pt,midway](gtp){GTP}
		(mlc);
	\draw[drawarrow] (actina) -- (contrazione);
	\draw[drawarrow] (mlc) -- (relax);
	\draw[drawarrow] (mlckstar) -| (a);
	\draw[drawarrow] (gtp)-- node[midway](cgtp){} 
		node[smallfont,xshift=10pt](gcstar){GC${}^*$}
		node[chartnode,xshift=100pt](gc){Guanil ciclasi} 
		(cgmp);
	\draw[drawarrow] (gc)-- node[midway](cgc){} node[midway, chartnode,yshift=-50pt](no){NO} (gcstar);
	\draw[drawarrow] (no)--node[midway, right]{$\oplus$}(cgc);
	
	\node[smallfont,text width=12em,anchor=west] at(0,-4) {* $\equiv$ elemento attivato\\
	MLCK $\equiv$ Miosina Catena Leggera chinasi\\ MLC $\equiv$ Miosina Catena Leggera};
\end{tikzpicture}

\newpage

\section{Farmaci dell'emostasi}

\begin{tikzpicture}
	\tikzset{frontier/.style={distance from root=300pt}} 
	\Tree 
		[ .Emostasi 
			[ .anticoagulanti 
				[ .iniettabili 
					\node[farmaco]{eparina};
				]
				[ .{inibitori della trombina} 
					\node[farmaco]{lepirudina\\ argatroban}; 
				]
				[ .orali \node[farmaco]{warfarin};  ]
			]
			[ .antiaggreganti 
				[ .FANS ]
				[ .{inibitori \\ della fosfodiesterasi} 
					\node[farmaco]{dipiridamolo\\ cilostazolo };
				]
				[ .{antagonisti recettori ADP} 
					\node[farmaco]{ticlopidina\\ clopidogrel }; 
				]
				[ .{inibitori recettore\\ Gp IIb/IIIA} 
					\node[farmaco]{abciximab\\ tirofiban\\ eptifibatide}; 				
				]
			]
			[ .trombolitici 
				[.{Attivatori tissutali\\ del plasminogeno (tPA)}
					\node[farmaco]{urochinasi\\ streptochinasi }; 
				]
			]
		]
			
\end{tikzpicture}

	\chapter{Farmaci epatici}

\section{Citocromo P450}

\ce{RH + O_2 +2H^+ + 2e^- ->[monoossigenasi] ROH + H2O}

\begin{tikzpicture}
	\tikzset{level distance=130pt}
	\Tree
	[.\node(inibitori){inibitori};
		[.CYP1A2
			\node[farmaco]{\index{ciprofloxacina}ciprofloxacina};
			\node[farmaco]{\index{tacrina}tacrina};
		]
		[.CYP2C9
			\node[farmaco]{\index{fluconazolo}fluconazolo};
		]
		[.{CYP3A4\\(principale nella\\ detossificazione da farmaci)}
			\node[farmaco]{\index{eritromicina}eritromicina};
			\node[farmaco]{\index{claritromicina}claritromicina};
			\node[farmaco]{\index{ketocomazolo}ketocomazolo};
			{succo di pompelmo}
		]
	]
	\begin{scope}[yshift=-10em]
		\Tree
		[.\node(metabolismo){metabolismo};
			\node[farmaco]{\index{corticosteroidi}corticosteroidi};
			\node[farmaco]{\index{warfarin}warfarin};
		]
	\end{scope}
	\begin{scope}[yshift=-20em]
		\Tree
		[.\node(attivatori){attivatori};
			barbiturici
			\node[farmaco]{\index{carbamazepina}carbamazepina};
		]
	\end{scope}
	\node[chartnode](rallenta) at (0,-5em) {rallenta};
	\node[chartnode](accellera) at (0em,-15em) {accellera};
	\draw[drawarrow] (attivatori) -> (accellera);
	\draw[drawarrow] (accellera) -> (metabolismo);
	\draw[drawarrow] (inibitori) -> (rallenta);
	\draw[drawarrow] (rallenta) -> (metabolismo);
\end{tikzpicture}

\newpage
	\printindex
\end{document}
