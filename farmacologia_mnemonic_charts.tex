\documentclass[12pt,paper=a4,twoside=false,parskip=half]{scrbook}
% aggiungere draft alla classe per vedere gli overfull hbox

% /-- Packages loading --------------------------------------------------------\
\usepackage[x11names]{xcolor}
\usepackage[italian]{babel}
\usepackage[utf8]{inputenc} 	% lettere accentate in documento UTF-8
\usepackage[T1]{fontenc} 		% doppi --
\usepackage{lmodern}
\usepackage{amsmath}
\usepackage{amsthm}
\usepackage{amsfonts}
\usepackage{units}
\usepackage{cancel}         
\usepackage{tikz}
\usepackage{tikz-qtree}
\usepackage{fancyhdr}           % Header e Footer 
\usepackage{chemfig}
\usepackage[version=3]{mhchem}
\usepackage{makeidx}			% Indice
\usepackage[colorlinks=true,allcolors=blue]{hyperref}
\usepackage[tikz]{bclogo}		% box simpatici
% \----------------------------------------------------------------------------/

% /-- title and authors -------------------------------------------------------\
\def\gtitle#1{\gdef\gtitle{#1}}
\def\gauthor#1{\gdef\gauthor{#1}}
\title{Appunti di Farmacologia}
\gtitle{Appunti di Farmacologia}
\author{Emiliano Bruni (info@ebruni.it)}
\gauthor{Emiliano Bruni (info@ebruni.it)}
\date{} %%If commented, the current date is used.

\pdfinfo {
/Title (\gtitle)
/Author (\gauthor)
/Subject (Appunti di Farmacologia)
/Keywords (farmacology, university, medicine, book, charts, mnemonic, flash, card, italian, Chieti, appunti)
/Copyright (copyrighted)
}
% \----------------------------------------------------------------------------/

\def\tikzbyncsa{
	\draw[fill=black] (2pt,2pt) rectangle (78pt,13pt);
	\begin{scope}
	\clip (0,0) rectangle (80pt,15pt);
	\fill[gray!67] (12pt,7.5pt) circle [x radius=19pt,y radius=14pt];
	\node[draw,
		circle,
		inner sep=2pt,
		outer sep=0ex,
		very thick, fill=white] at (15pt,7.5pt) {\small\bfseries\textsf{cc}};
	\end{scope}
	\draw[line width=1pt, white] (78pt,14pt)--(1pt,14pt)--(1pt,1pt) -- (78pt,1pt);
	\draw[line width=1pt] (0,0) rectangle (80pt,15pt);
	\node[white] at (55pt,7.5pt) {\scriptsize\bfseries\textsf{BY-NC-SA}};
}

\def\amminaq {\ce{R4N+}}
\def\amminat {\ce{R3N}}

% /-- Sezione/numeropag. nell'header e date e numero di revisione nel footer --\
\pagestyle{fancy}
\newcommand{\helv}{\fontfamily{phv}\fontseries{b}\fontsize{8}{10}\selectfont}
\fancyhf{}
\fancyhead[LO]{\helv \rightmark}
\fancyhead[LO]{\helv \leftmark}
\fancyhead[RO]{\helv \thepage}
\fancyfoot[ol]{\begin{tikzpicture}[scale=0.6, transform shape,baseline=2pt ]\tikzbyncsa\end{tikzpicture}\ \scriptsize\itshape Copyright \copyright\ 2016 \gauthor }
\fancyfoot[or]{\scriptsize\itshape Revisione del \today } % Date
% \----------------------------------------------------------------------------/

\usetikzlibrary{shapes,arrows,matrix,calc,automata,positioning,patterns}
\numberwithin{equation}{section}  % numeri delle equazioni del tipo x.y
\numberwithin{table}{section}     % numeri delle tabelle del tipo x.y
\numberwithin{figure}{section}    % numeri delle figure del tipo x.y

% ambiente per le nuove definizioni
\newenvironment{definizione}[1]{\begin{bclogo}[ arrondi =0.1, logo=\bccrayon, ombre=true, couleurBarre=gray]{\small #1}}{\end{bclogo}}



\makeindex							% generate the index


\begin{document}
	\tikzset{
	%Define standard arrow tip
	>=stealth',node distance=1cm, auto,font=\tiny,
	%Define style for boxes
	itm/.style={
		rectangle,
		rounded corners,
		draw=black, very thick,
		minimum width=5em,
		minimum height=1.5em,
		text centered,
		align=center,
		inner sep=3pt,
	},
	% Define arrow style
	ar/.style={
		<-,
		thick,
		shorten <=2pt,
		shorten >=2pt,},
	dummyar/.style={
		thick,
		shorten <=0pt,
		shorten >=2pt,},
	dummy/.style={
		minimum width=10ex,
		minimum height=0,
		inner sep=0,
		outer sep=0,
	},
	dummy0/.style={
		minimum width=0,
		minimum height=0,
		inner sep=0,
		outer sep=0,
	},
	count/.style={
		draw,
		circle,
		inner sep=.3ex,
		outer sep=.3ex,
		thick,
		anchor=south,fill=blue!30!white!10
	},
}
\def \lastitem {}
\def \dummy #1{\node[dummy,right=of #1] (#1_dummy) {};\gdef\lastitem{#1_dummy}}
\def \dummyz #1{\node[dummy0,right=of #1] (#1_dummy) {}}
\def \dummyzar #1{\node[dummy0,right=of #1] (#1_dummy) {} edge[ar] (#1.east);\gdef\lastitem{#1_dummy}}
\def \dummystart #1#2{\node[dummy0, below=of #2] (start#1) {};\gdef\lastitem{start#1}}
\def \dummyend #1{\node[dummy0,right=of #1] (#1_end) {} edge[ar] (#1.east);;\gdef\lastitem{#1_end}}
\def \itm #1#2{\gdef\lastitem{#1};\node[itm,fill=blue!40!white!40] (#1) {#2}}

\def \itmright #1#2#3#4{\node[itm, right=of #1] (#2) {#3} edge[ar] (#4.east);\gdef\lastitem{#2}}
%\def \itmabove #1#2#3#4{\node[itm, above=of #1] (#2) {#3} edge[ar] (#4.east);\gdef\lastitem{#2}}
\def \itmabove #1#2#3#4{
	\node[itm, above=1em of #1] (#2) {#3}; 
	\draw[thick,shorten <=2pt] (#4.east) -- ++(.3,0) [shorten >=2pt,shorten <=0pt,->] to[out=0,in=225] (#2.south west);
	\gdef\lastitem{#2}
}
\def \itmbelow #1#2#3#4{
	\node[itm, below=1em of #1] (#2) {#3};
	\draw[thick,shorten >=2pt,->] ($(#4.east) + (.3,0)$)  to[out=0,in=155] (#2.north west);
	\gdef\lastitem{#2};
}

\def \itmrightnoarrow #1#2#3{\node[itm, right=of #1] (#2) {#3}}

\def \itmrights #1#2#3{\itmright{#1}{#2}{#3}{#1}}
\def \itmaboves #1#2#3{
		\node[itm, above=of #1] (#2) {#3};
		\draw[ar]($(#2.south) + (.5cm,0)$) to[bend left] ($(#1.north)+ (.5cm,0)$);
		\gdef\lastitem{#2}
}
\def \itmbelows #1#2#3{
	\node[itm, below=of #1] (#2) {#3};
	\draw[ar]($(#2.north) + (-.5cm,0)$) to[bend left] ($(#1.south)+ (-.5cm,0)$);
	\gdef\lastitem{#2}
}

\def \itmsplit #1#2#3#4#5{\dummy{#1};
	\itmabove{#1_dummy}{#2}{#3}{#1};
	\itmbelow{#1_dummy}{#4}{#5}{#1}}

\def \itmmerge #1#2#3#4#5{\dummyz{#1_dummy}
			edge[dummyar] (#2.east)
			edge[dummyar] (#3.east);
	\node[itm, right=of #1_dummy_dummy] (#4) {#5}
			edge[ar,shorten >=0pt] (#1_dummy_dummy.north)}

\def \itmsplitthree #1#2#3#4#5#6#7{
	\itmrights{#1}{#4}{#5};
	\itmabove{#4}{#2}{#3}{#1};
	\itmbelow{#4}{#6}{#7}{#1};
}

\def \itmsplitfour #1#2#3#4#5#6#7#8#9{
	\node[itm,above right=0em and 3em of #1] (#4) {#5};
	\node[itm,above right=3em and 3em of #1] (#2) {#3};
	\node[itm,below right=0em and 3em of #1] (#6) {#7};
	\node[itm,below right=3em and 3em of #1] (#8) {#9};
	
	\draw[thick,shorten <=2pt] (#1.east) -- ++(.3,0) [shorten >=2pt,shorten <=0pt,->] to[out=0,in=225] (#2.south west);
	\draw[thick,shorten >=2pt,->] ($(#1.east) + (.3,0)$)  to[out=0,in=205] (#4.south west);
	\draw[thick,shorten >=2pt,->] ($(#1.east) + (.3,0)$)  to[out=0,in=155] (#6.north west);
	\draw[thick,shorten >=2pt,->] ($(#1.east) + (.3,0)$)  to[out=0,in=135] (#8.north west);
}

\def \armergetwo #1#2#3{ 
	\draw[thick,shorten <=2pt] (#1.east) to[out=0,in=180] ($(#3.west) + (-.3cm,0)$);
	\draw[thick,shorten <=2pt,shorten >=2pt,->] (#2.east) to[out=0,in=180] ($(#3.west) + (-.3cm,0)$) -- (#3.west);
}

\def \armergethree #1#2#3{
	\draw[thick,shorten <=2pt] (#1.east) to[out=0,in=180] ($(#3.west) + (-.3cm,0)$);
	\draw[thick,shorten <=2pt] (#2.east) to[out=0,in=180] ($(#3.west) + (-.3cm,0)$);
}

\def \itmcount #1#2{\node[count] at (#2.north) {#1};}

\newcounter{itmcvalue}
\def \itmcinit #1{\setcounter{itmcvalue}{#1}}
\def \itmcnoinc {\addtocounter{itmcvalue}{-1};}
\def \itmc #1{\itmcount{\arabic{itmcvalue}}{#1};\addtocounter{itmcvalue}{1};}

\def\itmr #1#2{\itmrights{\lastitem}{#1}{#2}}

% Set for tkiz-qtree
\tikzset{
	grow'=right,level distance=100pt,
	frontier/.style={distance from root=800pt},
	every tree node/.style={
		rectangle,
		rounded corners,
		draw=black, very thick,
		minimum width=5em,
		minimum height=1.5em,
		text centered,
		align=center,
		inner sep=3pt,
		font=\ttfamily\normalsize\tiny
	},
	every level 0 node/.style={
		top color=white, bottom color=blue!30
	},
	edge from parent/.append style={
		draw,->, thick,
		shorten <=0pt,
		shorten >=2pt,edge from parent path={
			(\tikzparentnode) to[out=0,in=180] (\tikzchildnode)
		}
	},
		farmaco/.style={
		top color=white, bottom color=green!30
	},
}

	\maketitle
	\begin{abstract}
	Questo articolo riassume con delle carte mnemoniche gli argomenti di farmacologia spiegati nel IV anno del corso di laurea in medicina e chirurgia a Chieti.
	L'uso di questo articolo non sostituisce la lettura e lo studio di un libro e degli appunti di farmacologia.
	Per errori, omissioni o altre note, non esitate a contattarmi via e-mail.
\end{abstract}

\newpage

\tableofcontents

\newpage\newpage
	\newpage\part{Flash Cards}

\section{Farmaci del SNC e del SNP}

\begin{tikzpicture}
	\tikzset{level distance=85pt,frontier/.style={distance from root=350pt}} 
	\Tree
	[.SNC
		[.Periferico
			[.Sensitivo ]
			[.Autonomo
		 		[ .{Simpatico\\ (toraco--addominale)} {gangli pre e para--vertebrali} ]
				[ .{Parasimpatico \\(nervi crani e sacrale)} {nell'intima degli organi} ]
			]
			[.Gastroenterico {plessi mioenterici (Auerbach) e\\ sottomucosi (Meissner)}
			]
		]
		[.Centrale ]
	]
\end{tikzpicture}

\begin{tikzpicture}
	\tikzset{level 1/.style={level distance=150pt}}
	\Tree
	[.{Neurotrasmettitori\\ SNP}
		 [.\node(acetilcolina){acetilcolina}; {recettori colinergici} ]
		 [.noradrenalina {recettori adrenergici} ]
		 [.\node(serotonina){serotonina\\5-HT 5-idrossitriptamina}; {recettori serotoninergici} ]
		 [.{monossido d'azoto (NO)} ] 
		 [.purine ]
	]
	\begin{scope}[yshift=-11em]
	\Tree
	[.\node(snc){Neurotrasmettitori\\ SNC};
		[.dopamina {recettori dopaminergici} ]
		[.{aminoacidi eccitatori}
			L--glutammato
			aspatato
			omocisteinato
		]
		GABA
	]
	\end{scope}
	\draw[drawarrow] (snc.east) to[in=180,out=40] (acetilcolina.west)
		(snc.east) to[in=180,out=0] (serotonina.west);
\end{tikzpicture}

\begin{tikzpicture}
	\Tree
	[.{Siti di azione\\ farmaci del SNC}
		[.{fibra pre--sinapica}
			{sintesi}
			{immagazzinamento}
			{rilascio}
			{ricaptazione}
			{metabolismo}
		]
		[.{fessura sinaptica}
			{recettore}
			{degradazione}
		]
		[.{fibra post--sinapica}
			{conducibilità ionica\\ del canale}
			{segnalazione retrograda}
		]
	]
\end{tikzpicture}

\subsection{Acetilcolina}

\begin{tikzpicture}
	\tikzset{level 1/.style={level distance=150pt}}
	\Tree
	[.\node(loc){localizzazione}; {tutte le fibre pregangliali sia para che orto\\ nicotiniche} {parasimpatiche post gangliali (quasi tutte).\\ muscariniche} {ghiandole sudoripare (simpatico)\\ muscariniche} {giunzione neuromuscolare\\ nicotiniche} 
		[.SNC
			proencefalo
			mesencefalo
			{tronco celebrale}
			cervelletto
			{interneuroni corpo striato}
		]
	]
\end{tikzpicture}

\begin{tikzpicture}
	\tikzset{level 2/.style={level distance=130pt}, level 3/.style={level distance=120pt}}
	\Tree
	[.Sintesi
		[.colina \edge node[smallfont,yshift=5pt,xshift=5.4em]{entra nel neurone} node[smallfont,yshift=-5pt,xshift=5.4em]{tappa limitante};
			[.{acetilCOA + colina} \edge node[smallfont,yshift=-5pt,xshift=5em]{acetiltrasferasi} node[smallfont,yshift=5pt,xshift=5em]{colina}; acetilcolina ]
		]
	]
\end{tikzpicture}

\begin{tikzpicture}
	\tikzset{level 2/.style={level distance=130pt}}
	\Tree
	[.Degradazione
		[.acetilcolina \edge node[smallfont, yshift=-5pt,xshift=5.5em]{acetilcolinesterasi}; {acetato + colina} ]
	]
\end{tikzpicture}

\begin{tikzpicture}
	\Tree
	[.Liberazione 
		[.{Ca${}^{2+}$ + VAMP/SNAPS}
			[.{fusione vescicole con\\ membrana neuronale} esocitosi
			]
		]
	]
\end{tikzpicture}

\begin{tikzpicture}
	\tikzstyle{cwhite}=[circle,shadedraw=yellow];
	\shade[ball color=yellow] node (ach) {\small Ach} circle[radius=.45];
	\draw (ach) -- +(.7,.7) node(vamp){} arc [start angle=225, end angle=270, radius=6pt]  (.7,.7) arc [start angle=225, end angle=180, radius=6pt];
	\draw (2,2) arc [start angle=45, end angle=0, radius=3cm] node[midway,above,sloped]{\tiny spazio sinaptico} ;
	\draw (2,2) 
		arc [start angle=45, end angle=60, radius=3cm] node(a){} 
		node[above right=2pt and 2pt of a] {\tiny Canale Ca${}^{2+}$}
		(a) arc [start angle=140, end angle=120, radius=1cm]
		(a) arc [start angle=140, end angle=160, radius=1cm];
	\path (2,2) arc [start angle=45, end angle=65, radius=3cm] node(b){};
	\draw (b) arc [start angle=-50, end angle=-30, radius=1cm]
		(b) arc [start angle=-50, end angle=-70, radius=1cm]
		(b) arc [start angle=65, end angle=80, radius=3cm];
	\draw (2,2) -- (1.3,1.3);
	\filldraw (1.3,1.3) circle[radius=3pt] node(snap){};
	\draw[->, shorten <=0pt,shorten >=2pt] (vamp)--(snap) node[midway,above,circle,draw,,yshift=3pt,xshift=-8pt]{\tiny 2};
	\draw[->, shorten <=0pt,shorten >=2pt] (ach) to[out=-20,in=225] node[midway,below,circle,draw,yshift=-5pt]{\tiny 3}  (4,1) ;
	\draw[->, shorten <=2pt,shorten >=6pt](3,3.5) to[out=205,in=90] node[near end,above,circle,draw,yshift=15pt]{\tiny 1}  node[very near end,above,yshift=15pt,xshift=-2pt]{\tiny Ca${}^{2+}$} (ach) ;
	\node at (1,.4) {\tiny VAMP};
	\node at (1.9,1.3) {\tiny SNAP};
\end{tikzpicture}

\begin{tikzpicture}
	\tikzset{level distance=75pt}
	\Tree
	[.Effetti
		[.SNP
			[.{$\uparrow$permeabilità\\ Na${}^+$, Ca${}^{2+}$}
				[.depolarizzazione
					[.{fibre post gangliari} PdA ]
					[.{fibre muscolari} {generazione\\ potenziale\\ di placca}
					]
				]
			]
		]
		[.SNC
			[.{$\uparrow$Ach}
				{veglia,\\ apprendimento,\\ memoria}
			]
		]
	]
\end{tikzpicture}

\begin{tikzpicture}
	\tikzset{level 1/.style={level distance=80pt},level 2/.style={level distance=110pt},level 3/.style={level distance=140pt}}
	\Tree
	[.{Recettore colinergico}
		[.{muscarinico\\ (metabotropo)}
			[.{M${}_1$ eccitatorio\\ $\uparrow\text{IP}_3, \uparrow\text{DAG},\uparrow\text{Ca}^{2+}$} {SNC, simpatico post--gangliare,\\ cellule parietali dello stomaco}
			]
			[.{M${}_2$ inibitorio $\downarrow$cAMP } {cuore, endotelio dei vasi}
			]
			[.{M${}_3$ eccitatorio\\ $\uparrow\text{IP}_3, \uparrow\text{DAG},\uparrow\text{Ca}^{2+}$} {ghiandole esocrine, muscolo liscio,\\ endotelio dei vasi}
			]
			[.{M${}_4$ come $\text{M}_2$} {SNC}
			]
			[.{M${}_5$  come $\text{M}_1$} {endotelio vasale, cervello,\\ SNC (facilita rilascio\\ glutammato e dopamina)}
			]
		]
		[.{nicotinico\\ (ionotropo)}
			[.{N${}_\text{N}$ gangliare} {para e ortosimpatico gangliare} ]
			[.{N${}_\text{M}$ muscolare} {giunzione\\ neuromuscolare} ]
		]
	]
\end{tikzpicture}

\subsubsection{Agonisti colinergici}

\begin{tikzpicture}
	\begin{scope}
	\tikzset{
		level distance=80pt,
		level 1/.style={level distance=70pt},
		level 2/.style={level distance=70pt},
		frontier/.style={distance from root=400pt} 
	}
	\Tree
	[.\node(main){Tipo}; 
		[.diretti
			[.{attivano recettori} 
				[.{esteri della colina} 
					\node(ach){acetilcolina };
					metacolina
					carbacolo
					\node[farmaco](beta){\index{betanecolo}betanecolo};
				]
				[.alcaloidi
					[.muscarinici
						muscarina
						\node[farmaco]{\index{pilocarpina}pilocarpina};
					]
					[.nitotinici \node[farmaco]{\index{nicotina}nicotina}; ]				
				]
			]
		]
		[.indiretti
			[.\node(AchEI){inibitori AchE}; 
			]
		]
	]	
	\node[right=5pt of ach] (achn) {};
	\node[right=5pt of beta] (betan) {};
	\draw[drawarrow] (achn) -- (betan) node[midway,above,sloped] {\tiny $\uparrow$resist. idrolisi e quindi durata azione};
\end{scope}
\begin{scope}[yshift=-14em]
	\Tree
	[.\node(AchEIa){inibitori AchE};
		[.{alcool+gruppo N quaternario}
			[.\node[farmaco]{\index{edrofonio}edrofonio}; 
				[.{legame idrogeno\\ o ionico con AchE} {idrolisi in minuti}
				]
			]
		]
		[.carbammati 
			[.\node[farmaco]{\index{neostigmina}neostigmina\\ \index{fisostigmina}fisostigmina\footnotemark}; 
				[.{legame covalente con AchE} {idrolisi in ore}
				]
			]
		]
		[.organofosfati
			[.\node[farmaco](ecotiopato){\index{ecotiopato}ecotiopato\footnotemark};
				[.\node(fAchE){fosforilazione AchE}; \node(idro){idrolisi in giorni};
				]
			]
			\node(somar){somar\\ (gas nervino)};
		]
	]		
	\draw[drawarrow] (somar) to[out=0,in=180] (fAchE);
	\node[chartnode,below=1em of fAchE] (invec) {invecchiamento\\rottura legame O-P\\ con raffozamento\\ legame con AchE}; 
	\node[chartnode,below=1em of idro] (pral) {pralidossima pu\`o\\ scindere la\\ fosforilazione};
	\draw[drawarrow] (pral)--(fAchE) node[midway,above,sloped] {\tiny qui si};
	\draw[drawarrow] (pral)--(invec) node[midway,above,sloped] {\tiny qui no};
	\node[below right=8pt and 5pt of somar] (somary) {};
	\node[above=9em of somary] (neox) {};
	\draw[drawarrow] (neox) -- (somary) node[midway,above,sloped] {\tiny $\uparrow$durata azione};
	\draw[drawarrow] (AchEI.south) to[out=-90,in=90] (AchEIa);
\end{scope}
\end{tikzpicture}

Nota: Mettere da qualche parte il tacrida (si usa nell'Alzheimer) che è un inibitore della colinesterasi che poi si ritrova come inibitori del citocromo P450.

\footnotetext{Presente nella fava del Calabar}
\footnotetext{Unico degli organofosfati perch\`e altamente polare e pu\`o essere preparato come soluzione acquosa. Era utilizzato per il glaucoma, ora in disuso.}

\begin{tikzpicture}
	\tikzset{level distance=120pt, level 2/.style={level distance=150pt}}
	\Tree
	[.\node(sub){Effetti}; 
		[.{SNC ($\text M_1$)} {tremore, ipotermina, $\uparrow$capacità cognittive}	
		]
		[.{occhio ($\text M_3$)}
			{costrizione muscolo sfintere dell'iride (miosi)\\ contrazione muscolo ciliare (accomodamento da vicino)}
		]
		[.{cuore ($\text M_2$)}
			{$\downarrow$frequenza (cronotopo-), $\downarrow$forza (inotropo-),\\ $\downarrow$vel. conduzione (dromotropo-), $\uparrow$periodo refrattario, NAV}
		]
		[.{vasi ($\text M_2$)}
			{vasodilatazione a basse dosi\\ vasocontrazione a alte dosi}
		]
		[.{polmone ($\text M_3$)}
			 {broncocostrizione, $\uparrow$secrezione}
		]
		[.{intestino ($\text M_3$)}
			{$\uparrow$motilit\`a, $\downarrow$ muscolatura sfinteri, $\uparrow$ secrezioni} 
		]
		[.vescica
			 {contrazione destrusore ($\text M_3$), rilascio trigono ($\text M_2$)}
		]
		[.{ghiandole esocrine ($\text M_3$)}
			{$\uparrow$secrezioni}
		]
		[.{giunzione neuromuscolare\\ (indiretti)}
			{basse concentrazioni: $\uparrow$forza contrazione utile \\ se intossicazioni da curaro o miastenia grave\\ alte concentrazioni: fibrillazione fibre muscolari}
		]
	]
\end{tikzpicture}

\begin{tikzpicture}
	\tikzset{frontier/.style={distance from root=320pt}, level 2/.style={level distance=100pt}}
	\Tree
	[.{usi clinici}
		[.\node[farmaco](pilocarpina){\index{pilocarpina}pilocarpina};
			[.{xerostomia da\\ sindrome di Sjogren} {$\uparrow$secrezioni salivali}
			]
		]
		[.\node[farmaco](ecotiopato){\index{ecotiopato}ecotiopato};
			[.\node(glaucoma){glaucoma};
				{nelle emergenze ad angolo chiuso,\\ agonista muscarinico + inibitore colinesterasi. \\	nel glaucoma cronico ora si usano i $\beta$-bloccanti}
			]
		]
		[.\node[farmaco](betanecolo){\index{betanecolo}betanecolo};
				\node(ritenzione){ritenzioni urinarie e ileo\\ depressione dell'attivit\`a senza ostruzione};
		]
		[.\node[farmaco](neostigmina){\index{neostigmina}neostigmina};
			[.\node(miastenia){miastenia grave};
				{cura e mezzo diagnostico}
			]
		]
		[.\node[farmaco](edrofonio){\index{edrofonio}edrofonio};
			{tachiaritmia parossistica sopraventricolare\\ in disuso, ora si usa l'adenosina.}
			\node(post){post anestesia per revertire i\\ neurobloccanti muscolari (vedi)};
		]
		[.\node[farmaco](fisostigmina){\index{fisostigmina}fisostigmina};
			{intossicazione da farmaci antimuscarinici\\ (intossicazione da atropina)}
		]
	]
	\draw[drawarrow](pilocarpina) to[out=0, in=180] (glaucoma);
	\draw[drawarrow](neostigmina) to[out=0, in=180] (ritenzione);
	\draw[drawarrow](neostigmina) to[out=0, in=180] (post);
	\draw[drawarrow](betanecolo) to[out=0, in=180] (ritenzione);
	\draw[drawarrow](edrofonio) to[out=0, in=180] (miastenia);
\end{tikzpicture}

\subsubsection{Antagonisti colinergici}

\begin{tikzpicture}
	\Tree
	[.{antagonisti\\ colinergici}
		[.antimuscarinici 
			[.\node[farmaco]{\index{atropina}atropina}; {deriva da Belladonna e\\ Datura Stramonium}
			]
			\node[farmaco]{\index{scopolamina}scopolamina};
			\node[farmaco]{\index{tiotropio}tiotropio};
			\node[farmaco]{\index{oxibutina}oxibutina};
		]
		[.antinicotinici 
			[.{bloccanti gangli\\ ganglioplegici} \node[farmaco]{\index{trimetafano}trimetafano}; 
			\node[farmaco]{\index{tossina botulinica}tossina botulinica};
			]
			[.{bloccanti neuromuscolari}
			]
		]
		[.{rigeneratori dell'AchE} \node[farmaco]{\index{pralidossima}pralidossima\footnotemark};
		]
	]
\end{tikzpicture}

\footnotetext{vedi inibitori dell'AchE}

\begin{tikzpicture}
	\Tree
	[.Assorbimento
		[.\node(am3){Ammine III${}^\circ$};
			Transcutaneo
		]
		[.{Ammine IV${}^\circ$}
			\node(intestino){intestino};
			\node(occhio){occhio};
		]
	]
	\draw[drawarrow]
		(am3) to[out=0, in=180] (intestino)
		(am3) to[out=0, in=180] (occhio);
\end{tikzpicture}

\textsc{Antimuscarinici}

\begin{tikzpicture}
	\tikzset{level distance=150pt}
	\Tree
	[.{effetti}
		[.{SNC ($\text M_1$)} {effetto stimolante (-atropina +scopolamina) $\downarrow$tremore Parkinson\\ Parkinson \`e causato da un $\uparrow$ attivit\`a colinergica}
		]
		[.{occhio ($\text M_3$)} {$\uparrow$attivit\`a simpatica $\Rightarrow$ midriasi\\ (belladonna $\equiv$ occhi dilatati\\ incapacit\`a di adattamento\\ visione da vicino)}
		]
		[.{cuore ($\text M_2$)} {tachicardia, blocco vagale, $\downarrow$PR per $\uparrow$dromotropo}
		]
		[.{vasi ($\text M_2/\text M_3$)} incerta ]
		[.{apparato respiratorio ($\text M_3$)} {broncodilatazione, $\downarrow$secrezioni\\ (ma meglio i $\beta$-adrenergici)}
		]
		[.{gastrointestinale ($\text M_3$)} {$\downarrow$secrezioni salivali, minori su tutto il resto} ]
		[.{gh. sudoripare ($\text M_3$)} {soppressione termoregolazione\\ (febbre da atropina)} ]
	]
\end{tikzpicture}

\begin{tikzpicture}
	%\tikzset{level 1/.style={level distance=130pt}}
	\Tree
	[.{usi clinici} 
		[.\node[farmaco](atropina){\index{atropina}atropina};
			{malattia di Parkinson}
			{esame oftalmico}
			{pre-operatorio\\ antilaringospasmo}
			{sincope vagale\\ da dolore infarto}
			{ulcera peptidica\\ in disuso}
		]
		[.\node[farmaco]{\index{scopolamina}scopolamina};
			{chinetosi\\ mal di mare}
		]
		[.\node[farmaco]{\index{tiotropio}tiotropio};
			\node(bpco){BPCO};
		]
		[.\node[farmaco]{\index{oxibutina}oxibutina};
			{tenismo urinario}
		]
		[.\node[farmaco]{\index{pralidossima}pralidossima};
			[.{iperfunzione colinergica}
				{da organofosfati, gas nervino\\ o intossicazione funghi}
			]
		]
	]
\end{tikzpicture}

\begin{tikzpicture}
	\Tree
	[.{Effetti avversi} {febbre da atropina} tachicardia {vasodilatazione con\\ esantema da atropina\\ testa, collo, arti, tronco} ]
\end{tikzpicture}

\textsc{Ganglioplegici}

\begin{tikzpicture}
	\tikzset{level 2/.style={level distance=150pt}}
	\Tree
	[.effetti
		[.SNC {sedazione, tremore}
		]
		[.occhio {perdita accomodamento, effetto su pupilla incerto\\ per innervazione para e orto del m. sfintere}
		]
		[.cardiocirc {$\downarrow$pressione con ipotensione ortostatica marcata}
		]
		[.gastro {$\downarrow$motilit\`a}
		]
		[.urinario {ritardo nella minzione, problemi erezione e eiaculazione}
		]
	]
\end{tikzpicture}

\begin{tikzpicture}
	\tikzset{level 2/.style={level distance=150pt}}
	\Tree
	[.{usi clinici}
		[.\node[farmaco]{\index{trimetafano}trimetafano};
			{ipotensivo nelle anestesie}
		]
		[.\node[farmaco]{\index{tossina botulinica}tossina botulinica};
			{iniezione intravescicale\\ contro incontinenza}
		]
	]
\end{tikzpicture}

\textsc{Bloccanti neuromuscolari}

\begin{tikzpicture}
	\tikzset{level 3/.style={level distance=130pt}}
	\Tree
	[.{influenza sui muscoli}
		[.{bloccanti neuromuscolari}
			{per indurre paralisi\\ durante gli interventi}
		]
		[.spasmolitici
			[.{per ridurre dolore\\ in varie situazioni}
				{vedi cap. 27 sull'argomento\\ o poi se trattato\\(benzodiazepine, \index{clonidina}clonidina,\\ antiepilettici, \index{dantrolene}dantrolene (anche contro\\ ipertermia maligna), \index{tossina botulinica}tossina botulinica)}
			]
		]
	]
\end{tikzpicture}

\begin{tikzpicture}
	\Tree
	[.{bloccanti\\ neuromuscolari}
		[.{non depolarizzanti}
			[.{impediscono l'accesso dell'Ach\\ sul recettore $\text M_M$}
				{tubocuranina\\ (curaro)}
				\node[farmaco]{\index{rocuronio}rocuronio};
			]
		]
		[.depolarizzanti
			[.{eccesso di Ach o simile}
				\node[farmaco]{\index{succinilcolina}succinilcolina\\(2 Ach legate tra loro)};
			]
		]
	]
\end{tikzpicture}

\begin{tikzpicture}
	\tikzset{level 2/.style={level distance=130pt}}
	\Tree
	[.Farmacocinetica
		[.assunzione EV ]
		[.metabolismo epatico ]
		[.degradazione \edge node[smallfont,yshift=-5pt,xshift=5em]{AchE (plasma)} node[smallfont,yshift=5pt,xshift=5em]{BuchE (fegato)}; {acido succinico + colina}
		]
	]		
\end{tikzpicture}

Una mutazione del gene che codifica la pseudocolinesterasi plasmatica rende alcuni pazienti pi\`u sensibili a metabolizzare la succinilcolina.

Il n. di dibucaina \`e un parametro per definire tali anomalie e dipende dal fatto che la dibucaina inibisce la pseudoAchE normale per l'80\% mentre l'inibizione \`e solo del 20\% in quella modificata.

\begin{tikzpicture}
	\tikzset{frontier/.style={distance from root=300pt}}
	\Tree
	[.funzionamento
		[.{non depolarizzanti} {stimolo tetanico} ]
		[.depolarizzanti
			[.{fase 1: depolarizzazione} fascicolazione ]
			[.{fase 2: desensibilizzazione} {stimolo tetanico} ]
		]
	]
\end{tikzpicture}

\begin{tikzpicture}
	\Tree
	[.{sequenza tetanica}
		[.{da muscoli piccoli a grandi}
			{m. occhio}
			{m. facciali}
			{m. arti}
			{faringe}
			\node(diaframma){diaframma};
		]
	]
	\node[below right=10pt and 10pt of diaframma] (da) {};
	\node[above=12em of da] (db) {};
	\draw[drawarrow] (da) -- (db) node[midway,above,sloped] {\tiny recupero sequ. inversa};
\end{tikzpicture}

\begin{tikzpicture}
		\tikzset{level distance=150pt}
	\Tree
	[.{effetti avversi}
		iperkalinemia
		{dolore muscolare post operatorio\\ (depolarizzanti)}
		{rilascio istamina e quindi ipotensione}
	]
\end{tikzpicture}

\subsection{Noradrenalina}

\begin{tikzpicture}
	\tikzset{level 1/.style={level distance=90pt}}
	\tikzset{level 2/.style={level distance=90pt}}
	\tikzset{level 3/.style={level distance=90pt}}
	\Tree
	[.noradrenalina 
		[.SNP
			[.{simpatiche postgangliari} 
				[.escluso
					{muscolatura\\ vasi renali ($\text D_1$)}
					{ghiandole sudoripare\\ (Ach)}
				]
			]		
		]
		[.SNC
			[.{$\alpha$ eccitatori/inibitori\\ recettoti $\beta$ inibitori}
				[.{$\uparrow$stato veglia,\\ $\alpha_2$ causano ipotensione}
					ponte
					{locus ceruleus}
					{midollo spinale}
				]
			]
		]
	]
\end{tikzpicture}

\begin{tikzpicture}
	\begin{scope}
	\tikzset{level distance=90pt,level 3/.style={level distance=110pt},
	level 2/.style={level distance=130pt},level 4/.style={level distance=60pt}}
	\Tree 
	[.Sintesi 
		[.Tirosina  \edge node[smallfont,yshift=-5pt,xshift=5.5em]{tirosin--idrossilasi} node[smallfont,yshift=5pt,xshift=5.5em]{tappa limitante}; 
			[.{L-Dopa} \edge node[smallfont,yshift=5pt,xshift=4.5em]{DOPA} node[smallfont,yshift=-5pt,xshift=4.5em]{decarbossilasi};
				[.\node[farmaco]{\index{dopamina}dopamina}; \node[dummyc]{};]
			]
		]
	]
	\end{scope}
	\begin{scope}[yshift=-3em,xshift=1em]
	\tikzset{level distance=80pt}
	\Tree
	[.\node[dummyc]{}; 
		[.\node[farmaco]{\index{noradrenalina}noradrenalina}; \node[farmaco]{\index{adrenalina}adrenalina};]
	]	
	\end{scope}				
	
\end{tikzpicture}

\begin{tikzpicture}
	\tikzset{level distance=130pt}
	\Tree 
	[.Degradazione
		[.{MAO (mono-ammino ossidasi)\\ in fegato e cellule ?????} ]
		[.{COMT (catecolo O-metiltransferasi)\\ nei neuroni e enterociti} ]
	]
\end{tikzpicture}

\begin{tikzpicture}
	\tikzset{level 2/.style={level distance=150pt}}
	\Tree	
	[.{Recettore adrenergico\\ (metabotropo \\ a proteine G)} 
		[.{$\alpha$}
			[ .{$\alpha_1\,\text{G}_{\text{q}} \uparrow\text{IP}_3, \uparrow\text{Ca}^{2+}$ (postsinaptiche muscolo liscio)} ]
			[ .{$\alpha_2\,\text{G}_{\text{i}} \downarrow\text{cAMP}$ (presinaptiche muscolo liscio)} ]
		]
		[.{$\beta$}
			[.{$\beta_1\,\text{G}_{\text{s}} \uparrow\text{cAMP}$ (postsinaptiche cuore, adipociti,\\ iuxaglomerulare, epitelio corpi ciliari)} ]
			[.{$\beta_2\,\text{G}_{\text{s}} \uparrow\text{cAMP}$ (postsinaptiche muscolo liscio e\\ cuore dove qualche volta sono \ce{G_i} inibitorie)} ]
			[.{$\beta_3\,\text{G}_{\text{s}} \uparrow\text{cAMP}$ (postsinaptiche cuore, adipociti, vescica)} ]		
		]
	]
\end{tikzpicture}

\begin{tabular}{|c|c|c|c|}
\hline 
\textbf{Organo} & \textbf{Tipo} & \textbf{Recettore} & \textbf{Azione} \\ 
\hline\hline 
M. radiale & simpatico & $\alpha_1$ & costrizione \\ 
\hline 
M. circolare & parasimpatico & M${}_3$ & costrizione pupilla \\ 
\hline 
M. ciliare & simpatico & $\beta$ & rilasciamento \\ 
\hline 
M. ciliare & parasimpatico & M${}_2$ & contrazione \\ 
\hline 
Nodo SA & simpatico & $\beta_1\beta_2$ & accellerazione \\ 
\hline 
Nodo SA & parasimpatico & M${}_2$ & rallentamento \\ 
\hline 
Forza contrazione & simpatico & $\beta_1\beta_2$ & aumento \\ 
\hline 
Forza contrazione & parasimpatico & M${}_2$ & diminuzione \\ 
\hline 
vasi muscolari & simpatico & $\beta$ & rilasciamento \\ 
\hline 
muscolo gastrointestinale & simpatico & $\alpha_2\beta_2$ & rilasciamento \\ 
\hline 
muscolo gastrointestinale & parasimpatico & M${}_3$ & contrazione \\ 
\hline 
sfinteri gastrointestinali & simpatico & $\alpha_1$ & contrazione \\ 
\hline 
sfinteri gastrointestinali & parasimpatico & M${}_3$ & rilasciamento \\ 
\hline 
\end{tabular} 

\subsubsection{Simpaticomimetici}

\begin{tikzpicture}
	\tikzset{frontier/.style={distance from root=350pt},level 2/.style={level distance=150pt}}
	\Tree
	[.simpaticomimetici
		[.diretta 
			[.{interazione con i recettori} 
				\node[farmaco]{\index{adrenalina}adrenalina\\
				\index{fenilefrina}fenilefrina\\
				\index{clonidina}clonidina\\
				\index{dobutamina}dobutamina\\
				\index{salbutamolo}salbutamolo\\
				\index{oximetazolina}oximetazolina};
			]
		]
		[.indiretta 
			[.{rilascio catecolamine immagazzinate} 
				\node[farmaco]{\index{tiramina}tiramina};			
			]
			[.{riduzione clearance della noradrenalina} ]
			[.{inibizione della ricaptazione\\ della noradrenalina} ]
			[.{inibizione del catabolismo\\ enzimatico via blocco MAO e COMT} ]
		]
		[.miste
			\node[farmaco]{\index{efedrina}efedrina};
		]
	]
\end{tikzpicture}

\begin{tikzpicture}
	\tikzset{frontier/.style={distance from root=350pt}}
	\Tree
	[.effetti
		[.diretti
			[.{dipendono da} 
		 		{vie somministrazione} 
		 		{selettivit\`a per i sottotipi\\ recettoriali}
		 		{espressione dei sottotipi\\ nei tessuti}
		 	]
		]
		[.indiretti {effetto proporzionale\\ all'attivazione del simpatico}
		]
	]
\end{tikzpicture}

\begin{tikzpicture}
	\Tree
	[.{Eliminazione da fessura}
		[.ricaptazione {NET (90\%)} ]
		[.metabolismo {COMT (10\%)} ]
	]
\end{tikzpicture}

\begin{tikzpicture}
	\Tree
	[.{chimica dei\\ simpaticomimetici}
		[.catecolamine
			{adrenalina\\
			noradrenalina\\
			dopamina}
		]
		[.{non catecolamine}
			{fenilefrina\\ efedrina\\ anfetamina}
		]
	]
\end{tikzpicture}

\begin{tikzpicture}
	\setatomsep{2em}
	\schemestart
	\chemname{\chemfig[][scale=0.8]{[:30]*6(-=-=(-OH)-(-OH)=-)}}{Catecolo}\quad
	\chemname{\chemfig[][scale=0.8]{[:30]*6(-=(-CH([6]-OH)-CH_2-NH_2)-=(-OH)-(-OH)=-)}}{Noradrenalina}
	\chemname{\chemfig[][scale=0.8]{[:30]*6(-=(-CH([6]-OH)-CH_2-NH-CH_3)-=(-OH)-(-OH)=-)}}{Adrenalina}
	\schemestop
\end{tikzpicture}

\begin{tikzpicture}
	\setatomsep{2em}
	\schemestart
	\chemname{\chemfig[][scale=0.8]{[:30]*6(-=(-CH([6]-OH)-CH_2-NH-CH_3)-=(-OH)-=-)}}{Fenilefrina}\quad
	\chemname{\chemfig[][scale=0.8]{[:30]*6(-=(-CH([6]-OH)-CH([6]-CH_3)-NH-CH_3)-=-=-)}}{Efedrina}\quad
	\chemname{\chemfig[][scale=0.8]{[:30]*6(-=(-CH_2-CH([6]-CH_3)-NH2)-=-=-)}}{Anfetamina}
	\schemestop
\end{tikzpicture}

Le catecolamine sono degradate da COMT a livello intestinale e epatico per cui l'assorbimento per os \`e praticamente nulla.

L'assenza di uno o di ambedue i gruppi \chemfig{-[,.5]OH} ne aumenta la disponibilit\`a per os.

La metilazione sul primo carbonio a sx del gruppo ammino, comporta un'azione mista dei farmaci come nell'efedrina e l'anfetamina che hanno azione diretta e indiretta e quindi dipendono anche dalla presenza del neurotrasmettitore.

\begin{tikzpicture}
	\tikzset{level 3/.style={level distance=130pt}}
	\Tree
	[.{effetti}
		[.{occhio}
			[.$\alpha_1$ {contrazione muscolo pupillare\\ con dilatazione della pupilla} ]
		]
		[.{cuore}
			[.$\alpha_1$ {inotropo+, porta a ipertrofia} ]
			[.$\beta_1$ {effetto cronotropo+ e inotropo+} ]
		]
		[.{sistema circolatorio}
			[.$\alpha_1$ {contrazione vasi} ]
			[.$\alpha_2$ {aggregazione piastrinica e\\ nel SNC causa ipotensione} ]
			[.$\beta_2$ {rilasciamento vasi} ]
		]			
		[.{muscolo liscio}
			[.{$\alpha_1,\alpha_2$ endogena o IV} contrazione ]
			[.{$\alpha_2$ per os} {riduzione del tono simpatico per accumulo SNC} ]
			[.{$\beta_2$} {rilasciamento della muscolatura} ] 
		]
		[.{respiratorio}
			[.{$\beta_2$} {rilasciamento della muscolatura\\ liscia bronchiale$\Rightarrow$broncodilatazione} ] 
		]
		[.{gastrointestinale} 
			[.{$\alpha, \beta$} {rilasciamento muscolatura} ]
		]
		[.{rene}
			[.$\beta_1$ {rilascio di renina} ]
		]
		[.{metabolismo}
			[.{$\beta_2$} {promozione della iperglicemia} ]
			[.{$\beta_3$} {lipolisi e quindi iperlipidemia} ]
		]
		[.{sistema immunitario}
			[.{$\beta$} {$\uparrow$lifociti, $\uparrow$killing, $\uparrow$citochine} ]
		]
	]
\end{tikzpicture}

\begin{tikzpicture}
	\tikzset{level 2/.style={level distance=130pt},level 3/.style={level distance=130pt}}
	\Tree
	[.{usi clinici\\(diretti)}
		[.\node[farmaco]{\index{adrenalina}adrenalina\\($\alpha,\beta$)};
			[.vasocostrizione
				{prolungamento anestetici locali}
				{shock anafilattico}
			]
			[.{stimolazione cardiaca}
				{arresto cardiaco}
			]
		]
		[.\node[farmaco]{\index{fenilefrina}fenilefrina\\($\alpha_1$)};
			{$\downarrow$prurito}
			{esame retina (midriasi)}
			\node(congestione){$\downarrow$congestione mucose};
		]
		[.\node[farmaco](oxi){\index{oximetazolina}oximetazolina\\($\alpha_2$)};
		]
		[.\node[farmaco]{\index{clonidina}clonidina\\($\alpha_2$)};
			{ipertensione nelle gestanti}
			{$\downarrow$vampate calore in menopausa}
			{disintossicazione da droghe}
			{stimolazione $\alpha_2$ vagale con vasocostrizione\footnotemark}
		]
		[.\node[farmaco]{\index{metildopa}metildopa\\($\alpha_2$)};
			{emergenze ipertensive\footnotemark}
		]
		[.\node[farmaco]{\index{dobutamina}dobutamina\\($\beta_1$)};
			[.{$\uparrow$gittata cardiaca ma non tachicardia}
				{insufficienza cardiaca}
				{shock cardiogeno}
				{test farmacologico da sforzo\\ quando non si pu\`o usare\\ la cyclette}
			]
		]
		[.\node[farmaco]{\index{salbutamolo}salbutamolo\\($\beta_2$)};
			{asma come broncodilatatore}
			{inibizione parti prematuri\\ per relax muscolatura uterina}
		]
	]
	\draw[drawarrow](oxi) to[out=0,in=180] (congestione);
\end{tikzpicture}

\footnotetext{Per cui pu\`o dare anche un aumento della pressione e per questo non si usa nelle emergenge da ipertensione}
\footnotetext{\`E anche un inibitore della DOPA decarbossilasi per cui $\downarrow$dopamina.}

\begin{tikzpicture}
	\tikzset{level 3/.style={level distance=150pt}}
	\Tree
	[.{usi clinici\\(indiretti)}
		[.\node[farmaco]{anfetamine};
			[.{rilascio noradrenalina\\ e dopamina} {stimolante SNC}
			]
		]
		[.\node[farmaco]{\index{tiramina}tiramina};
			[.{simile a noradrenalina} {$\uparrow$pressione in pazienti trattati\\ con inibitori delle MAO\\ che dovrebbero degradarle.\\ Prodotti dal metabolismo\\ della tirosina. Non assumere\\ cibi come formaggi,...\\ in terapia da inibitori MAO}
			]
		]
		[.\node[farmaco]{\index{cocaina}cocaina};
			[.{blocco ricaptazione noradrenalina\\ e dopamina}
				{anestetico locale}
				{nel SNC il blocco della ricaptazione\\ della dopamina provoca piacere}
			]
		]
	]
\end{tikzpicture}

\begin{tikzpicture}
	\Tree
	[.{Effetti avversi}
		[.{Accentuazione dell'effetto\\ farmacologico}
			{vasocostrizione eccessiva}
			{aritmie}
			{infarto miocardio}
			{edema polmonare}
			{emorragie polmonari}
		]
	]
\end{tikzpicture}

\subsubsection{Inibitori dei recettori adrenergici}

\begin{tikzpicture}
	\Tree
	[.tipo
		[.{$\alpha$-bloccanti} 
		]
		[.{$\beta$-bloccanti}
		]
	]
\end{tikzpicture}

\textsc{$\alpha$-bloccanti}

\begin{tikzpicture}
	\tikzset{level distance=150pt}
	\Tree
	[.effetti
		[.{riduzione delle resistenze periferiche\\ e tachicardia riflessa\\ ($\alpha$)} 
			{$\downarrow$pressione} 
			{ipotensione posturale}
			{inversione dell'adrenalina\footnotemark}
		]
		[.{rilasciamento muscoli vescica, prostata\\($\alpha_1$)}
			{ritenzione urinaria\\ da iperplasia prosatic benigna}
		]
	]	
\end{tikzpicture}

\footnotetext{Attiva sia gli $\alpha$ che i $\beta_2$. Se si immette un $\alpha$-bloccante questo neutralizzer\`a l'effetto vasocostrittore dell'adrenalina lasciando la sola attivazione dei $\beta_2$ che quindi causer\`a una vasodilatazione da cui un azione inversa a quella usuale dell'adrenalina}

\begin{tikzpicture}
	\tikzset{level distance=80pt}
	\tikzset{level 3/.style={level distance=100pt}}
	\tikzset{level 4/.style={level distance=120pt}}
	\Tree
	[.{usi clinici}
		[.{non selettivi}
			[.\node[farmaco]{\index{fenossibenzamina}fenossibenzamina\\ \index{fentolamina}fentolamina};
				[.{riduzione pressione soprattutto\\ in presenza di elevato tono simpatico\\ (passaggio da prono a ortopostura)} {feocromocitoma\\ pre intervento}
				]
			]
		]
		[.{$\alpha_1$}
			[.\node[farmaco]{\index{prazosina}prazosina};
				{antiipertensivo}
				{ritenzione urinaria\\ da iperplasia prosatic benigna}
			]
			[.\node[farmaco]{\index{labetalolo}labetalolo}; {blocca principalmente i $\beta$.\\ Descritto in seguito}
			]
		]
		[.{$\alpha_2$} {solo sperimentali}
		]
	]
\end{tikzpicture}

\textsc{$\beta$-bloccanti}

\begin{tikzpicture}
	\tikzset{level 2/.style={level distance=120pt}}
	\tikzset{level 3/.style={level distance=140pt}}
	\Tree
	[.effetti
		[.cuore
			[.{inibizione sistema renina-angiotensina} 
				{$\downarrow$pressione arteriosa} 
			]
			[.{inibizione recettori adrenergici cardiaci}
				{effetto inotropo-, cronotopo-}
				{allungamento PR}
			]
		]
		[.polmone
			[.{blocco $\beta_2$ nella\\ muscolatura liscia bronchiale}
				{BPCO (soprattutto se associate\\ a cardiopatie ischemiche)}
			]
		]
		[.occhio		
			[.{riduzione della produzione\\ di umor acqueo} {glaucoma}
			]
		]
		[.{blocco dei canali Na${}^+$} {azione anestetica locale}
		]
	]	
\end{tikzpicture}

\begin{tikzpicture}
	\tikzset{level 1/.style={level distance=120pt}}
	\tikzset{level 2/.style={level distance=150pt}}
	\Tree
	[.{farmaci}
		[.\node[farmaco]{\index{propanololo}propanololo};
			{angina pectoris}
			{infarto del miocardio}
			{aritmie}
			{insufficienza cardiaca}
			{ipertiroidismo}
			{tempesta tiroidea}
		]
		[.\node[farmaco]{\index{metoprololo}metoprololo\\ \index{atenololo}atenololo};
			{emicrania}
			{tremore muscolare}
			{stati d'ansia}
		]
		[.\node[farmaco]{\index{labetalolo}labetalolo\\ \index{carvedilolo}carvedilolo\\(sono sia $\alpha$ che $\beta$};
			{feocromocitoma}
			{feoc. + scompenso + ipertensione}
		]
		[.\node[farmaco]{\index{timololo}timololo}; 
			{glaucoma}
		]
	]
\end{tikzpicture}


\subsection{Dopamina}

Ricorda anche la dopamina \`e una catecolamina quindi anche i recettori dopaminergici sono recettori adrenergici

\begin{tikzpicture}
	\Tree
	[.SNC
		[.{corpo striato}
			[.{via nigrostriale}
				[.{dalla sostanza nera\\ al corpo striato}
					{via deficitaria nel parkinson}
				]
			]
		]
		[.{sistema limbico}
			[.{via mesolimbica}
				{da mesencefalo\\ a nucleo accumbens}
			]
		]
		[.{amigdala e corteccia\\ prefrontale}
			{via della gratificazione}
		]
		[.{ipotalamo} 
			[.{via tuberoinfundibolare}
				{da ipotalamo ventrale\\ a ipofisi}
			]
		]
	]
\end{tikzpicture}

\begin{tikzpicture}
	\tikzset{level distance=130pt}
	\Tree
	[.{Recettore dopaminergico}
		[.{$\text D_1$, $\text D_5$, eccitatorio, $\uparrow$cAMP} {cervello (striato e ipotalamo),\\ muscolatura vasi rene} 
		]
		[.{$\text D_2$, inibitorio,\\ apertura canali $\text K^+$\\ blocco dei canali \ce{Ca^2+}} { muscolatura liscia} 
		]
		[.{$\text D_3$, inibitorio,\\ apertura canali $\text K^+$\\ blocco dei canali \ce{Ca^2+}} {cervello (sistema limbico)} 
		]
		[.{$\text D_4$, inibitorio,\\ apertura canali $\text K^+$\\ blocco dei canali \ce{Ca^2+}} {cervello, sistema cardio vascolare} 
		]
	]
\end{tikzpicture}

\begin{tikzpicture}
	\Tree
	[.{$\uparrow$dopamina}
		{effetto stereotipato}
		{schizzofrenia}
		{$\uparrow$vomito e nausea}
		{$\uparrow$GH}
	]
\end{tikzpicture}

\subsection{Serotonina (5-idrossitriptamina)}

\begin{tikzpicture}
	\Tree
	[.SNC
		ponte
		{midollo allungato}
		{nuclei del rafe}
	]
\end{tikzpicture}

\begin{tikzpicture}
	\tikzset{level 3/.style={level distance=130pt}}
	\Tree
 	[.Recettore
 		[.\ce{5-HT_1}
 			[.SNC {GpCR,$\downarrow$cAMP, inibizione presinaptica} ]
 		]
 		[.\ce{5-HT_2}
 			[.{muscolo liscio\\ piastrine, corteccia e\\ sistema limbico} {$\uparrow$IP3, DAG, GpCR} ]
 		]
 		[.\ce{5-HT_3}
 			[.{SNP (nocicettori\\ neuroni enterici)} {canale ionico stimolatore} ]
 		]
 		[.\ce{5-HT_4}
 			[.{SNC,vescica\\ cuore, striato\\ e cervelletto} {$\uparrow$cAMP,GpCR,eccitazione} ]
 		]
 		[.\ce{5-HT_5}
 			{ippocampo}
 		]
 		[.\ce{5-HT_7}
 			[.{regolazione termica\\ ed endocrina}
 				corteccia
 				ippocampo
 				talamo
 			]
 		]
 	]
\end{tikzpicture}

\begin{tikzpicture}
	\tikzset{level distance=80pt,level 3/.style={level distance=130pt},
	level 2/.style={level distance=130pt}}
	\Tree 
	[.Sintesi 
		[.Triptofano  \edge node[smallfont,yshift=5pt,xshift=5.5em]{triptofano} node[smallfont,yshift=-5pt,xshift=5.5em]{idrossilasi} ; 
			[.{5-idrossitriptofano} \edge node[smallfont,yshift=5pt,xshift=5.5em]{amminoacido} node[smallfont,yshift=-5pt,xshift=5.5em]{decarbossilasi}; 5-HT 
			]
		]
	]	
\end{tikzpicture}

\begin{tikzpicture}
	\tikzset{level distance=130pt}
	\Tree 
	[.Degradazione {MAO (mono-ammino ossidasi)}
	]
\end{tikzpicture}

\begin{tikzpicture}
	\tikzset{level distance=160pt}
	\Tree
	[.Effetti {piastrine: aggregazione} {terminazioni nervose: dolore} {SNC: eccitatorio 5-HT4,\\ inibitorio 5-HT1} {vasi: costrizione} {gastroenterico: attivazione secrezione\\ e peristalsi} ]
\end{tikzpicture}

\subsection{Neurotrasmettitori purinici}

\begin{tikzpicture}
	\tikzset{level 2/.style={level distance=130pt}}
 	\Tree
 	[.{Neurotrasmettitori\\ purinici} 
 		[.ATP {Aumento della permeabilit\`a di membrana} ]
 		[.Adenosina {vasodilatatore tranne che nel rene} {inibizione dell'aggreg. piastrinica} {blocco della conduzione AV}
 		]
 	]
\end{tikzpicture}

\subsection{Monossido d'azoto (NO)}

\begin{tikzpicture}
	\Tree
	[.Tipi
		[.iNOS {prodotto dai macrofagi tramite IF$\gamma$}
		]
		[.eNOS {endotelio e piastrine}
		]
		[.nNOS 
			[.neuroni
				cervello
				ippocampo
			]	
		]
	]
\end{tikzpicture}

\begin{tikzpicture}
	\Tree
	[.{$\uparrow$\ce{nNOS}}
		{??? di $\uparrow$\ce{Ca^2+}}
		{favorisce eventi ischemici}
		{neurogenerazione}
		{parkinson}
		{demenza senile}
	]
\end{tikzpicture}

\begin{tikzpicture}
	\tikzset{level distance=160pt}
	\Tree
	[.Causa {vasodilatazione} {inibizione dell'aggregazione piastrinica} {plasticit\`a sinaptica} {difesa da cellule neoplastiche, batteri, parassiti} ]
\end{tikzpicture}

Per via inalatoria $\downarrow$shunt, $\downarrow$broncocostrizione, $\downarrow$ipertensione polmonare e quindi utile anche nella cura dell'asma.

Utile nel trattamento delle malattie neurovegetative e shock settico dove aumenta e nell'ateorscelosi e ipercolesterolemia dove diminuisce.

\subsection{L-glutammato}

Neurotrasmettitore ubiquitario eccitatorio del SNC

\begin{tikzpicture}
	\Tree
	[.sintesi
		[.glucosio
			[.{ciclo di Krebs}
				{glutammato}
			]
		]
		[.neurone \edge[<->] node {};
			[.glutammato \edge[<->] node {};
				[.{cellule della glia\\(astrocita)} \edge[<->] node {};
					glutammina
				]
			]
		]
	]
\end{tikzpicture}

\begin{tikzpicture}
	\Tree
	[.rilascio
		[.vescicolare \edge node[smallfont,above,xshift=+4em] {\ce{Ca^2+}dip.};
			esocitosi
		]
	]
\end{tikzpicture}

\begin{tikzpicture}
	\Tree
	[.recettori
		[.ionotropi
			[.ANPA
				[.ubiquitari	
					{trasmissione sinaptica veloce}
				]
			]
			[.{KA (kainato)}
				[.{ippocampo, cervelletto\\ e midollo spinale}
					{trasmissione sinaptica veloce}
				]
			]
			[.NMDA
				[.ubiquitari
					{plasticità sinaptica}
				]
			]
		]
		[.metabotropi
			[.{\ce{G_q} ($\uparrow$\ce{IP_3},$\uparrow$DAG,$\uparrow$\ce{Ca^2+}}
				{modulazione a\\ medio/lungo termine}
			]
		]
	]
\end{tikzpicture}

\begin{tikzpicture}
	\Tree
	[.PdA
		[.{bassa frequenza}
			[.AMPA
				attivazione
			]
			[.NMDA
				{bloccato da \ce{Mg^2+}}
			]
		]
		[.{alta frequenza}
			[.AMPA
				attivazione
			]
			[.NMDA
				[.{espulsione \ce{Mg^2+}} \edge node[smallfont,above,xshift=3.6em] {+glicina};
					[.\node(ca){$\uparrow$\ce{Ca^2+}\\ Long Term Potentiation\\ plasticità sinaptica};
					]
				]
			]
			[.\ce{G_q}
				\node(a){attivazione};
			]
		]
	]
	\draw[drawarrow] (a) to[out=0,in=180] (ca);
\end{tikzpicture}

\begin{tikzpicture}
	\Tree
	[.utilizzo
		[.{inefficacia causa\\ ubiquità del recettore}
			{lesioni traumatiche}
			epilessia
			Alzheimer
		]
	]
\end{tikzpicture}

\begin{tikzpicture}
	\Tree
	[.farmaci
		[.agonisti
			[.????
				[.{$\uparrow$AMPA}
					{$\uparrow$memoria}
					{$\uparrow$capacità cognittive}
				]
			]
		]
		[.antagonisti
			[.{recettore NMDA}
				[.\node[farmaco]{\index{memantina}memantina};
					{malattie neurovegetative}
				]
				[.\node[farmaco]{\index{ketamina}ketamina};
					{anestetico dissociativo}
				]
			]
			[.{recettore glicina in NMDA}
				\node[farmaco]{\index{acido chinuretico}acido chinuretico};
			]
			[.AMPA
				{causano depressione generale\\ del SNC e crisi respiratorie}
			]
			[.metabotropo
				{usi futuri}
			]
		]
	]
\end{tikzpicture}


\subsection{GABA (Acido $\gamma$--amminobutirrico)}

\begin{tikzpicture}
	\Tree
	[.GABA
		[.{inibitorio}
			[.{ubiquitario della\\ sostanza grigia}
				[.{nigrostriato}
				]
			]
		]
	]
\end{tikzpicture}

\begin{tikzpicture}
	\Tree
	[.Sintesi
		[.glutammato
			GABA
		]
	]
\end{tikzpicture}

\begin{tikzpicture}
	\Tree
	[.degradazione
		[.GABA
			[.{astrociti}
				{acido succinico}
			]
		]
	]
\end{tikzpicture}

Enzima GABA--transaminasi (o GABA amminotransferasi). Utile informazione relativamente al valproato (farmaco antiepilettico, vedi).

\begin{tikzpicture}
	\tikzset{level 3/.style={level distance=150pt}}
	\Tree
	[.recettori
		[.{\ce{GABA_a}}
			[.{accoppiato a canale \ce{Cl-}}
				{presinaptico: inibizione lenta}
				{postsinaptico: inibizione veloce}
			]
		]
		[.{\ce{GABA_b}}
			[.{Accoppiato a \ce{G_i}}
				{$\downarrow$adenilato ciclasi $\Rightarrow\uparrow$ingresso \ce{K+}}
			]
		]
	]
\end{tikzpicture}

\begin{tikzpicture}
	\Tree
	[.farmaci
		[.{\ce{GABA_a}}
			[.agonisti
				benzodiazepine
				barbiturici
				anestetici
			]
		]
		[.{\ce{GABA_b}}
			[.agonisti
				[.\node[farmaco]{\index{bacoflen}bacoflen};
					{miorilassante usato\\ nella terapia del dolore}
				]
			]
			[.antagonisti
				[.\node[farmaco]{\index{saclofen}saclofen};
					{antiepilettico sperimentale}
				]
			]
		]
	]
\end{tikzpicture}

\subsection{GBH (Acido $\gamma$--idrossibutirrico)}

Proviene dalla sintesi del GABA. $\uparrow$rilascio GH, attiva le "vie della gratificazione", da euforia e disibinizione. Droga da strada.

\subsection{Melatonina}

\begin{tikzpicture}
	\Tree
	[.Sintesi
		[.{nella gh. pineale}
			{acetilazione di 5-HT}
		]
	]
\end{tikzpicture}

\begin{tikzpicture}
	\Tree
	[.recettori
		[.{associati a proteine G}
			{cervelletto}
			{???}
		]
	]
\end{tikzpicture}

\begin{tikzpicture}
	\Tree
	[.effetti
		{$\uparrow$ sonnolenza}
	]
\end{tikzpicture}

\begin{tikzpicture}
	\Tree
	[.utilizzo
		{jet lag}
	]
\end{tikzpicture}

\subsection{Glicina}

\begin{tikzpicture}
	\Tree
	[.recettore
		{canale \ce{Cl-} simil \ce{GABA_a}}
	]
\end{tikzpicture}

Nessun farmaco in uso agisce su questo recettore. Stricnina e tossina tetanica prevengono il rilascio di glicina

\newpage
	\section{Farmaci delle patologie del SN}

\subsection{Farmaci sedativo/ansioliti e ipnotici}

\begin{tikzpicture}
	\Tree
	[.{scopo}
		[.\node(sedazione){sedazione};
			[.{ansiolitico\\ (\dwa stato ansia)}
				[.tranquillante
					\node(aum){aumentando dosaggio};
				]
			]
		]
		[.\node(ipnotico){ipnotico};
			[.{sonnolenza\\ (\upa sonno)}
				\node(dep){depressione SN maggiore};
			]
		]
	]
	\draw[drawarrow] (aum) to[out=-145,in=-20] (ipnotico);
	\draw[drawarrow] (dep) to[out=0,in=20] (sedazione);
\end{tikzpicture}

\begin{tikzpicture}
	\Tree
	[.{relazione \\ dose/risposta}
		[.sedazione
			[.ipnotico
				[.anestesia
				]
			]
		]
	]
	\begin{scope}[yshift=-3em,xshift=3em]
	\Tree
	[.\node[dummyc]{};
		[.{depressione centri\\ respiro}
			[.coma
				morte
			]
		]
	]
	\end{scope}
\end{tikzpicture}

\begin{tikzpicture}
	\tikzset{level 1/.style={level distance=130pt}}
	\tikzset{frontier/.style={distance from root=250pt}}
	\Tree
	[.{tipologie}
		[.benzodiazepine
			\node[farmaco]{\index{diazepam}diazepam\\ emivita $12\sim72$h\\(valium - ansiol.)};
			\node[farmaco]{\index{lorazepam}lorazepam\\ emivita $2\sim6$h\\(tavor - ansiol.)};
			\node[farmaco]{\index{flunitrazepam}flunitrazepam\\ emivita $8\sim24$h\\(roipnol - ansiol./ipnot.)};
		]
		[.barbiturici
			\node[farmaco]{\index{pentobarbital}pentobarbital\\(ipnotico)};
			\node[farmaco]{\index{fenobarbital}fenobarbital\\(più come antiepilettico)\\(ipnotico)};
			\node[farmaco]{\index{tiopental}tiopental\\(più come anestetico generale\\ siero della verità)\\(ipnotico)};
		]
		\node[farmaco]{\index{buspirone}buspirone\\ (ansiol.)};
		[.{ipnotici\\ non benzodiazepinici}
			\node[farmaco]{\index{zolpidem}zolpidem};
			\node[farmaco]{\index{zaleplon}zaleplon};
		]
		[.antipsicotici
			\node[farmaco]{\index{ATC}ATC\\(vedi sezione)};
		]
		[.antidepressivi
			{i più recenti SSRI e SNRI\\ (vedi sezione)}
		]
		[.antiepilettici
			\node[farmaco]{\index{valproato}valproato\\(vedi sezione)\\(ansiol.)};
		]
	]
\end{tikzpicture}

\begin{tikzpicture}
	\Tree
	[.{bezodiazepine}
		[.meccanismo
			[.{recettore \ce{GABA_a}\\ nel sito BZ}
				[.{isoforma\\ con subunità $\alpha_1$}
					{effetto sedazione}
				]
				[.{con subunità $\alpha_2$}
					{isoforma\\ effetto ansiolitico}
				]
			]
		]
		[.assunzione
			os EV
		]
		[.metabolismo
			[.epatico
				[.{lunga emivita}
					{desmetilazione con formazione\\ metabolita attivo\\ con emivita $>40$h}
				]
				[.{breve emivita}
					[.{glucuronazione e\\ escrezione renale}
					]
				]
			]
		]
		[.{effetti terapeutici}
			{$\downarrow$ansia\\ $\downarrow$aggressicità\\ induzione sonno}
		]
		[.{effetti collaterali}
			{amnesia anterograda\footnotemark\\ tolleranza ($\uparrow$dose)\\ dipendenza psicofisica\\ depressione sistema respiratorio }
		]
	]
\end{tikzpicture}

\footnotetext{Incapacità a ricordare azioni avvenute durante l'azione del farmaco. Roipnol, la droga dello stupro.}

Per anatagonizzare i sovradosaggi delle benzodiazepine si usa il flumazenil, anche lui una benzodiazepina ma con azione antagonista su BZ.

\begin{tikzpicture}
	\Tree
	[.{flumazenil}
		[.{antagonista\\ recettore BZ}
			[.{blocca l'azione di}
				benzodiazepine
				{non--benzodiazepine}
			]
			[.{non blocca l'azione di}
				barbiturici
				etanolo
			]
		]
	]
\end{tikzpicture}

\begin{tikzpicture}
	\Tree
	[.\node[farmaco]{\index{buspirone}buspirone\\(non in Italia)};
		[.meccanismo
			{agonista parziale \ce{5-HT_{1A}} e \ce{D_2}}
		]
		[.assunzione os
		]
		[.metabolismo epatico
		]
		[.{effetti terapeutici}
			[.vantaggi
				{ansiolitico\\ no sedazione\\ no dipendenza\\ no ansia da interruzione}
			]
			[.svantaggi
				{necessario settimane prima\\ che inizino gli effetti}
			]
		]
		[.{effetti collaterali}
			{nausea\\ vertigini\\ irrequietezza}
		]
	]
\end{tikzpicture}

\begin{tikzpicture}
	\Tree
	[.barbiturici
		[.meccanismo
			{recettore \ce{GABA_a}\\ in sito differente da BZ}
		]
		[.assunzione {os,EV}
		]
		[.metabolismo epatico
		]
		[.{effetti terapeutici}
			{depressione non selettiva del SNC\\ attualmente usati come\\
			antiepilettici e anestetici
			}
		]
		[.{effetti collaterali}
			{alta tolleranza\\ alta dipendenza\\ induttori P450\\ depressione cardiopolmonare}
		]
	]
\end{tikzpicture}

\begin{tikzpicture}
	\tikzset{level 1/.style={level distance=130pt}}
	\Tree
	[.{ipnotici non--benzodiazepinici}
		[.meccanismo
			[.{recettore \ce{GABA_a}\\ con sole unità $\alpha_1$}
				{sedazione ma non ansiolitico}
			]
		]
		[.assunzione {os}
		]
		[.metabolismo epatico
		]
		[.{effetti terapeutici}
			{trattamento insonnia\\ no tolleranza\\ no dipendenza
			}
		]
		[.{effetti collaterali}
			{interferenza con alcol\\ depressione SNC ad alte dosi}
		]
	]
\end{tikzpicture}

\begin{tikzpicture}
	\Tree
	[.{qualità sonno}
		{1. latenza prima inizio sonno}
		{2. fase non--REM (NREM)}
		{3. fase REM}
		{4. fase NREM a onde lente}
	]
	\begin{scope}[xshift=14em]
		\node(par){$\left.\rule{0pt}{46pt}\right\}$};
		\node[text width=10em] at (2.5,0){ideale $\downarrow$1. $\uparrow$2. $\uparrow$3. $\uparrow$4. };
	\end{scope}
\end{tikzpicture}

\begin{tikzpicture}
	\Tree
	[.{effetto farmaci\\ sul sonno}
		[.benzodiazepine
			{$\downarrow$1. $\uparrow$2. $\downarrow$3. $\downarrow$4. }
		]
		[.{non--benzodiazepine}
			{$\downarrow$1. ininfluente sugli altri }
		]
	]
\end{tikzpicture}

Tutti questi farmaci si legano al \ce{GABA_A} tranne il buspirone che agisce come agonista parziale per \ce{5-HT_{1A}} e \ce{D_2}.

\ce{GABA_B} attivato da agenti spasmolitici come baclofene.

Tutti \upa inibizione GABAergica \upa permeabilità al \ce{Cl-} del canale \ce{GABA_A} tramie interazione allosterica con GABA.

Le benzodiazepine \upa frequenza apertura del canale.
I barbiturici \upa durata apertura del canale.
I barbiturici sono anche GABA--mimetici quindi funzionano teoricamente anche senza GABA.

\subsection{Farmaci anti--psicotici}

\begin{tikzpicture}
	\Tree
	[.{schizzofrenia}
		{persistenza di alterazioni del pensiero,\\ del comportamente e delle affettività}
	]
\end{tikzpicture}

\begin{tikzpicture}
	\Tree
	[.{sintomi}
		[.{positivi\\(caratt. psicol. aggiuntive)}
			delirio
			allucinazioni
			{pensieri disorganizzati}
			aggressività
		]
		[.{negativi\\(caratt. psicol. perse)}
			{isolamento sociale}
			apatia
			{mancanza di iniziativa}
			{insensibilità emotiva}
		]
	]
\end{tikzpicture}

\begin{tikzpicture}
	\tikzset{level distance=80pt}
	\Tree
	[.{fisiopatologia}
		[.positivi
			[.{inibizione GABAesrgica\\ \ce{D_2} mediata}
				inibizione
			]
		]
		[.negativi
			[.{\dwa n. di recettori \ce{D_1}}
				[.{iperattività GABA}
					[.{inibizione\\ glutammatergica}
						{ipostimolazione del\\ recettore NMDA}
					]
				]
			]
		]
	]
\end{tikzpicture}

\begin{tikzpicture}
	\Tree
	[.{teorie fisiopatologiche}
		[.{teoria dopaminergica}
			[.{spiega sintomi positivi}
				{\upa stimolazione \ce{D_2}/\ce{D_4}\\ oppure \upa stim. \ce{D_2}, \dwa stim. \ce{D_4}\\ pazienti schizzofrenici hanno \\\upa densità \ce{D_2} e \upa dopamina}
			]
		]
		[.{teoria glutammatergica}
			[.{bassi livelli\\ di glutammato}
				[.{spiega sintomi negativi}
					{\dwa stimolazione NMDA}
				]
			]
		]
		[.{teoria serotoninergica}
			[.{sintomi negativi}
				{antagonisti serotoninergici\\ sono anti--psicotici}
			]
			[.{sinitomi positivi}
				{LSD (agonista 5-HT) fa\\ venire i sintomi positivi}
			]
		]
	]
\end{tikzpicture}

\begin{definizione}{EPS} 
Effetti collaterali extrapiramidali
\end{definizione}
 

\begin{definizione}{Agente neurolettico}
alta incidenza di EPS a dosi efficaci
\end{definizione}

\begin{tikzpicture}
	\tikzset{frontier/.style={distance from root=300pt}}
	\Tree
	[.{farmaci}
		[.tipici
			[.fenotiazidici
				\node[farmaco]{\index{clorpromazina}clorpromazina\\ agente neurolettico};
			]
			[.butirrofenonico
				\node[farmaco]{\index{aloperidolo}aloperidolo\\ (serenase)};
			]
		]
		[.atipici
			\node[farmaco]{\index{clozapina}clozapina\\(dibenzodiazepina)};
			\node[farmaco]{\index{olanzapina}olanzapina};
		]
	]
\end{tikzpicture}

I farmaci anti--psicotici impiegano settimane per l'effetto, segno che vi sia un effetto secondario tipo incremento dei recettori \ce{D_2} a livello limbico.

Tutti hanno effetti a lunga durata (mesi) dopo l'ultima somministrazione. Solo \index{clozapina}clozapina non va mai sospesa bruscamente sie per le recidive rapide sia per effetti sfavorevoli quali miocarditi e agranulocitosi.

\begin{tikzpicture}
	\tikzset{level 2/.style={level distance=150pt}}
	\Tree
	[.{tipici}
		[.meccanismo
			{blocco recettori \ce{D_1}/\ce{D_2}}
		]
		[.assorbimento
			{os, intramuscolo}
		]
		[.metabolismo
			{epatico via cit. P450\\ isoforme CYP2D6, CYP1A2/A4\\ escrezione renale}
		]
		[.{usi clinici}
			{schizzofrenia: \dwa sintomi positivi\\ i più recenti anche \dwa sintomi negativi\\ anti--emetici}
		]
		[.{effetti collaterali}
			{EPS con distonia acuta\\ nelle prime settimane reversibili.\\ Discinesia tardiva con movimento\\ involontario di volto e lingua\\ causati da \upa n. \ce{D_2}}
		]
	]
\end{tikzpicture}

\begin{tikzpicture}
	\Tree
	[.{atipici}
		[.meccanismo
			{blocco \ce{D_2}, blocco $\alpha$--adrenergico\\ blocco 5--HT e \ce{D_4}\\ blocco recettori muscarinici}
		]
		[.assorbimento
			{intrauscolo}
		]
		[.metabolismo
			{epatico}
		]
		[.{effetti collaterali}
			{aumento di peso,\\ secchezza fauci,\\ stipsi,\\ ritenzione idrica,\\ ipotensione posturale\\ leucopenia}
		]
	]
\end{tikzpicture}

I farmaci atipici danno meno effetti collaterali motori perchè bloccano selettivamente la via mesolimbica (della gratificazione) invece del nigro--striato. Impegnati quindi se i sintomi extrapiramidali dei tipici fossero problematici.

\subsection{Farmaci antidepressivi}

\begin{tikzpicture}
	\tikzset{frontier/.style={distance from root=330pt}}
	\Tree
	[.{depressione}
		[.{disturbo dell'umore}
			[.{sintomi emozionali}
				{malessere, apatia, pessimismo}
				{bassa autostima}
				{incapacità a prendere decisioni}
				{idee di autolesionismo o\\ suicide}
			]
			[.{sintomi biologici}
				{ritardi nel pensiero o\\ nell'azione}
				{perdita di libibo}
				insonnia
				{perdita di appetito}
			]
		]
		[.{ma usati anche per}
			{attacchi panico}
			{disturbo d'ansia generalizzato}
			{disturbo post--traumatico da stress}
			{disturbo ossessivo compulsivo}
			{trattamento dolore neuropatico}
			{fibromialgia}
			{sindrome disforica pre--metruale}
		]
	]
\end{tikzpicture}


\begin{tikzpicture}
	\tikzset{level distance=90pt}
	\Tree
	[.{depressione}
		[.unipolare
			{variazioni di umore\\ sempre nello stesso verso}
		]
		[.{bipolare\\(sindrome maniaco\\ depressiva)}
			[.\node(d){depressione};
				[.\node(m){mania};
					\node(es){esuberanza};
				]
			]
		]
	]
	\node[below of=es,chartnode](e){entusiasmo};
	\node[below of=m,chartnode](im){impazienza};
	\node[below of=d,chartnode](c){collera};
	\draw[drawarrow] (es) -- (e) (e)--(im) (im)--(c) (c)--(d);
\end{tikzpicture}

\begin{definizione}{BDNF}
Brain--derivated neurotrophic factor: fattore di crescita nervoso
\end{definizione}


\begin{tikzpicture}
	\Tree
	[.{teorie\\ fisiopatologiche}
		[.{teoria monoaminergica\\ (vecchia)}
			[.\node[farmaco]{\dwa\index{noradrenalina}noradrenalina\\\dwa\index{serotonina}serotonina\dwa\index{dopamina}dopamina};
				[.\upa depressione ]
				[.\dwa {elevazione dell'umore} ]
			]
		]
		[.{teoria neutrofica\\ (più accreditata)}
			[.{perdita di neuroni e\\ di glia in ippocampo e\\ corteccia pre--frontale}
				[.depressione
					{\index{BDNF}BDNF porta a\\ elevazione dell'umore}
				]
			]
		]
	]
\end{tikzpicture}

\begin{tikzpicture}
	\Tree
	[.{teoria monaminergica}
		[.{\index{noradrenalina}noradrenalina, \index{serotonina}serotonina\\ innervano neuroni ipotalamici\\ che controllano\\ $\ominus$ l'ipofisi}
			[.{\upa CRH}
				[.{\upa ADCH}
					{\upa cortisolo\\(ormone dello stress)}
				]
			]
		]
	]
\end{tikzpicture}

\begin{tikzpicture}
\tikzset{frontier/.style={distance from root=400pt}}
	\Tree
	[.{farmaci}
		[.{inibitori\\ della ricaptazione}
			[.{della serotonina\\ blocco recettore SERT}
				[.SSRI
					\node[farmaco]{\index{fluoxetina}fluoxetina\\(prozac)};
				]
			]
			[.{della serotonina e\\ noradrenalina\\ blocco recettore NET}
				[.SNRI
					\node[farmaco]{\index{venlafaxina}venlafaxina};
				]
				[.{ATC\\(antidepressivi triciclici)}
					\node[farmaco]{\index{amitriptilina}amitriptilina};
					\node[farmaco]{\index{imipramina}imipramina};
				]
			]
		]
		[.{inibitori\\ delle monoamminossidasi\\(IMAO)}
			\node[farmaco]{\index{fenelzina}fenelzina\\(ritirato in Italia)};
			\node[farmaco]{\index{moclobemide}moclobemide};
		]
	]
\end{tikzpicture}

\begin{tikzpicture}
	\Tree
	[.{SSRI}
		[.meccanismo
			{\upa serotonina per\\ inibizione della ricaptazione}
		]
		[.assorbimento
			{os}
		]
		[.metabolismo
			[.{passa per \index{norfluoxetina}norfluoxetina,\\ composto attivo con\\ emivita 3x fluoxetina}
				{epatico con interazione con\\ ATC, CYP2D6}
			]
		]
		[.{usi clinici}
			{depressione, ansia, \\ panico,\\ disordine ossessivo--compulsivo}
		]
		[.{effetti collaterali}
			{nausea, vomito, insonnia,\\ sindrome serotoninergica (tremore\\ collasso cardiocirc., ipertermia)}
			{4 settimane di interruzione\\ prima di usaun un IMAO}
		]
	]
\end{tikzpicture}

\begin{tikzpicture}
	\Tree
	[.{ATC}
		[.meccanismo
			{\upa serotonina per\\ inibizione della ricaptazione}
			{noradrenalina per competizione\\ con i siti di legame}
		]
		[.assorbimento
			{os}
		]
		[.metabolismo
			{epatico con grande volume\\ di distribuzione e lento\\ metabolismo e quindi\\ stretta finestra terapeutica}
		]
		[.{usi clinici}
			{panico, antidepressivo, \\ analgesico nella cura\\ del dolore neuropatico}
		]
		[.{effetti collaterali}
			{atropina--like,\\ ipotensione posturale,aritmie\\ depressione respiratoria\\ con alcol}
			{ATC + IMAO MAI per rischio\\ ipertensione acuta}
		]
	]
\end{tikzpicture}

Il 10\% dei caucasici ha una mutazione del gene della CYP2D6 con effetti collaterali agli ATC molto più pesanti

\begin{tikzpicture}
	\tikzset{level 2/.style={level distance=150pt}}
	\Tree
	[.{IMAO}
		[.meccanismo
			[.{vecchi farmaci\\(fenelzina)}
				{inibizione irreversibile\\ dell'enzima MAO.\\ }
			]
			[.{nuovi farmaci\\(moclobemide)}
				{inibizione reversibile\\ dell'enzima MAO.\\ }
			]
		]
		[.{usi clinici}
			{\upa conc. plasm. 5-HT, \ce{Na+} e dopamina\\ Euforia e eccitazione}
		]
		[.{effetti collaterali}
			{ipotensione caratteristica per\\ dwa rilascio noradrenalina.}
			{downreg. recettori $\beta$--adrenergici e 5-HT}
			{cheese reaction con crisi ipertensiva acuta\\ perchè \index{tiramina}tiramina contenuta nei\\ formaggi non viene degradata\ dalle MAO inibite}
			{ATC + IMAO MAI per rischio ipertensione acuta}
		]
	]
\end{tikzpicture}

Di solito quindi la terapia è ATC + SSRI.

\begin{tikzpicture}
	\Tree
	[.{stabilizzazione\\ dell'umore}
		\node[farmaco]{\index{litio}litio};
		[.antiepilettici
			\node[farmaco]{\index{carbamazepina}carbamazepina};
			\node[farmaco]{\index{valproato}valproato};
		]
	]
\end{tikzpicture}

\begin{tikzpicture}
	\tikzset{level 2/.style={level distance=130pt}}
	\tikzset{level 3/.style={level distance=130pt}}
	\Tree
	[.{Litio}
		[.meccanismo
			[.{$\ominus$ inositolo monofosfato}
				{depressione cirtuiti iperattivi}
			]
			[.{$\ominus$ gliglogenosinetasi chinasi (GSK-3)}
				{attivazione enzimi apoptotici}
			]
			{\dwa cAMP}
		]
		[.assorbimento
			{os sotto forma di carbonato}
		]
		[.metabolismo
			[.{renale lenta\\ 1$\sim$2 settimane}	
				{stretta finestra terapeutica\\ pericolo sovradosaggio}
			]
		]
		[.{usi clinici}
			{profilassi e trattamento di\\ depressione uni e bipolare}
		]
		[.{effetti collaterali}
			{tremore (si riduce \\con \index{propranololo}propranololo)}			
			[.{+ diuretici}
				{\dwa \ce{Na+}}
			]
			{nausea, vomito, diarrea}
			[.{ritensione di \ce{Na+} e\\ \upa aldosterone}
				{danni tubulari}
			]
			{\dwa funzione tiroidea\\ e ipertiroidismo}
			{\upa peso corporeo e edema}
		]
	]
\end{tikzpicture}

Carbamazepina: $\ominus$ IP

Valproato: $\ominus$ IP + $\ominus$ GSK-3


\subsection{Farmaci antiepilettici}

\begin{tikzpicture}
	\tikzset{frontier/.style={distance from root=400pt}}
	\Tree
	[.{epilessie}
		[.parziali
			[.{inizio locus\\ encefalico specifico}
				[.{resta specifico}
					{semplice}
				]
				[.{si diffonde}
					{complesso}
				]
			]
			[.{evolve in grande male}
				{secondarie}
			]
		]
		{spasmi infantili\\(sindrome epilettica)}
		[.generalizzate
			{tonico--cloniche\\(grande male)}
			toniche
			{(mio)cloniche}
			{atoniche\\(perdita tono posturale)}
			{assenze\\(piccolo male)}
		]
	]
\end{tikzpicture}

\begin{tikzpicture}
	\tikzset{level distance=150pt}
	\Tree
	[.{farmaci}
		[.\node[farmaco](fenitoina){\index{fenitoina}fenitoina};
			\node(parziali){parziali};
		]
		[.\node[farmaco](carba){\index{carbamazepina}carbamazepina};
			\node(tc){tonico--cloniche};
		]
		[.\node[farmaco](val){\index{valproato}valproato};
			\node(assenze){assenze};
		]
		[.\node(benzo){benzodiazepine};
			\node(miocloniche){miocloniche};
		]
		[.\node[farmaco]{\index{etosuccimide}etosuccimide};
			\node(atoniche){atoniche};
		]
		[.\node[farmaco]{\index{corticotropina}corticotropina};
			\node(spasmi){spasmi infantili};
		]
		[.\node[farmaco]{\index{diazepam}diazepam\\ benzodiazepina + \\ fosfenitoina (simil\\ fenitoina ma IV o IM)};
			{emergenze epilettiche}
		]
		[.\node[farmaco]{\index{acetazolamide}acetazolamide\\(diuretico inibizione\\ anidrasi carbonica)}; tutti
		]
	]
	\draw[drawarrow] (fenitoina) to[out=0,in=180] (tc);
	\draw[drawarrow] (carba) to[out=0,in=180] (parziali);
	\draw[drawarrow] (val) to[out=0,in=180] (tc);
	\draw[drawarrow] (val) to[out=0,in=180] (miocloniche);
	\draw[drawarrow] (val) to[out=0,in=180] (atoniche);
	\draw[drawarrow] (benzo) to[out=0,in=180] (atoniche);
	\draw[drawarrow] (benzo) to[out=0,in=180] (spasmi);
\end{tikzpicture}

\begin{tikzpicture}
	\tikzset{level 2/.style={level distance=150pt}}
	\tikzset{level 3/.style={level distance=80pt}}
	\tikzset{level 4/.style={level distance=80pt}}
	\Tree
	[.\node[farmaco]{\index{fenitoina}fenitoina\\(più antico)};
		[.meccanismo
			[.{inibizione canali \ce{Na+}}
				[.{+$\uparrow$freq. scarica}
					{+$\uparrow$blocco}
				]
			]
		]
		[.assunzione {os, EV solo fosfenitoina}
		]
		[.metabolismo {epatico dose--dipendente fino a un limite\\ oltre il quale piccoli $\uparrow$dosi\\ producono grandi $\uparrow$conc. ematica}
		]
		[.{effetti terapeutici}
			{parziali e tonico--cloniche.\\ Non assenze che può causare
			}
		]
		[.{effetti collaterali}
			{induttore P450}
			{trasp. da albumina ma\\ \index{valproato}valproato ha maggiore affinità e\\ quindi causa $\uparrow$conc. farmaco libero}
			{vertigini, cefalee, nistagmo}
			{possibile effetto teratogeno}
		]
	]
\end{tikzpicture}

\begin{tikzpicture}
	\Tree
	[.\node[farmaco]{\index{{carbamazepina}}carbamazepina};
		[.meccanismo
			{come fenitoina + blocco \ce{Ca^2+}}
		]
		[.assunzione {os}
		]
		[.metabolismo ???
		]
		[.{effetti terapeutici}
			{tutti tranne assenze}
		]
		[.{effetti collaterali}
			[.{induttore P450}
				{accellera il\\ metabolismo della\\ fenitoina e warfarin}
			]
			[.{epossido}
				sonnolenza
				{ritenzione idrica}
				epatotossicità
			]
		]
	]
\end{tikzpicture}

\begin{tikzpicture}
	\tikzset{level 2/.style={level distance=150pt}}
	\Tree
	[.\node[farmaco]{\index{{valproato}}valproato};
		[.meccanismo
			[.{come fenitoina + inibizione\\ di due enzimi che inattivano\\ il GABA}
				{$\uparrow$GABA}
			]
		]
		[.assunzione {os}
		]
		[.metabolismo ???
		]
		[.{effetti terapeutici}
			{tonico--cloniche e assenze}
			{bassa tossicità}
			{no azione sedativa}
		]
		[.{effetti collaterali}
			{rari casi assottigliamento capelli}
			{rari casi teratogenicità\\ con spina bifida}
			{spiazza fenitoina\\ dalle proteine plasmatiche}
			[.{inibisce il metabolismo di}
				\node[farmaco]{\index{fenobarbital}fenobarbital};
				\node[farmaco]{\index{fenitoina}fenitoina};
				\node[farmaco]{\index{carbamazepina}carbamazepina};
			]
		]
	]
\end{tikzpicture}

\begin{tikzpicture}
	\tikzset{level 2/.style={level distance=150pt}}
	\Tree
	[.\node[farmaco]{\index{etosuccimide}etosuccimide};
		[.meccanismo
			[.{inibizione dei canali \ce{Ca^2+}\\ a bassa soglia}
				{responsabili delle correnti\\ corticali ritmiche\\ nelle assenze}
			]
		]
		[.assunzione {os}
		]
		[.metabolismo epatico
		]
		[.{effetti terapeutici}
			assenze
		]
		[.{effetti collaterali}
			{nausea\\ letargie\\ crisi tonico--cloniche}
		]
	]
\end{tikzpicture}

\begin{tikzpicture}
	\Tree
	[.{altri farmaci recenti}
		[.\node[farmaco]{\index{vigabatrin}vigabatrin};
			[.{si lega irreversibilmente\\ a GABA--amministranferasi}
				{$\uparrow$GABA nel cervello}
			]
		]
		[.\node[farmaco]{\index{lamotrigina}lamotrigina};
			{blocco \ce{Na+} per assenze}
		]
		[.\node[farmaco]{\index{felbamato}felbamato};
			{blocco uso--dipendente NMDA.\\ Si usa solo nelle epilessie\\ intrattabili a causa\\
			dei suoi effetti collaterali\\ (anemia aplastica e epatite)}
		]
	]
\end{tikzpicture}

\subsection{Malattia di Alzheimer (AD)}

\begin{tikzpicture}
	\Tree
	[.{AD}
		[.idiopatica
			{demenza per--senile}
		]
	]
\end{tikzpicture}

\begin{tikzpicture}
	\Tree
	[.{patogenesi}
		[.\node(a){perdita di neuroni\\ colinergici};
			ippocampo
			{corteccia frontale}
			{nuclei della base}
		]
		[.\node(b){presenza placche\\ amiloidi};
			[.{depositi intracellulari\\ di $\beta$--amiloidi (A$\beta$)}
				apoptosi
			]
		]
	]
	\draw[drawarrow] (b)--(a) node[midway, below, sloped,smallfont] {causa};
\end{tikzpicture}

\begin{tikzpicture}
	\tikzset{frontier/.style={distance from root=200pt}}
	\Tree
	[.{farmaci}
		[.{inibitori della\\ colinesterasi}
			\node[farmaco]{\index{tacrina}tacrina};
			\node[farmaco]{\index{rivastigmina}rivastigmina};
		]
		\node[farmaco]{\index{diidroergotamina}diidroergotamina};
		[.{antagonisti recettore\\ NMDA}
			\node[farmaco]{\index{memantina}memantina};
		]
	]
\end{tikzpicture}

\begin{tikzpicture}
	\Tree
	[.{inibitori della\\ colinesterasi}
		[.meccanismo
			{$\uparrow$trasmissione colinergica}
		]
		[.assorbimento os
		]
		[.metabolismo epatico
		]
		[.{usi clinici}
			{piccolo miglioramento\\ funzionalità}
			{no effetti su progressione\\ malattia}
		]
		[.{effetti collaterali}
			{nausea\\ crampi addominali\\ apatotossicità (tacrina)}
		]
	]
\end{tikzpicture}

\begin{tikzpicture}
	\Tree
	[.{antagonisti recettore\\ NMDA}
		[.meccanismo
			{\dwa eccitossicità da glutammato}
		]
		[.assorbimento os
		]
		[.metabolismo epatico
		]
		[.{usi clinici}
			{migliore memoria}
		]
		[.{effetti collaterali}
			{diarrea\\ insonnia\\ incontinenza}
		]
	]
\end{tikzpicture}

\begin{tikzpicture}
	\Tree
	[.{eccitossicità glutammato}
		[.{\upa glutammato}
			[.{\upa stimolaziona AMPA\\ con sblocco NMDA}
				[.{liberazione \ce{Ca^2+}}
					\node[dummyc]{};
				]
			]
		]
	]
	\begin{scope}[yshift=-3em,xshift=3em]
	\Tree
	[.\node[dummyc]{};
		[.{\upa rilascio glutammato}
			[.{attivazione proteasi\\ \index{calpaina}calpaina e lipasi}
				{produzione di \ce{NO}\\ nel cervello}
				{rilascio acido arachidonico}
			]
		]
	]
	\end{scope}
\end{tikzpicture}

\begin{tikzpicture}
	\Tree
	[.\node[farmaco]{\index{diidroergotamina}diidroergotamina};
		[.meccanismo
			{vasodilatazione celebrale}
		]
		[.assorbimento {os,EV}		
		]
		[.metabolismo epatico
		]
		[.{usi clinici}
			{trattamento generale della demenza\\ efficacia limitata}
		]
	]
\end{tikzpicture}

\begin{tikzpicture}
	\Tree
	[.{altri farmaci}
		[.{FANS}
			\node[farmaco]{\index{ibuprofene}ibuprofene};
			\node[farmaco]{\index{indometacina}indometacina};
		]
		[.{agenti chelanti}
			{A$\beta$ include \ce{Cu},\ce{Zn}}
		]
		{immunizzazione contro le A$\beta$}
	]
\end{tikzpicture}

Anche impianto di cellule ingegnerizzate per produrre il fattore di crescita dei neuroni (fattore neurotrofico derivato dal cervello \index{BDNF}BDNF).

\subsection{Malattia di Parkinson (PD)}

\begin{tikzpicture}
	\Tree
	[.{PD}
		[.{malattia neurodegenerativa\\ cronica}
			[.{via extrapiramidale}
				[.{\dwa DOPA}, 
					{\upa azione Ach,\\ noradrenalina,\\ 5-HT e GABA}
				]
			]
		]
	]
\end{tikzpicture}

\begin{tikzpicture}
	%\draw[step=.5,very thin,yellow] (-2,0) grid (6,3);
	\node at (-2,.5) {Normale};
	\node(a) at (0,1) {};
	\node(b) at (2,0) {};
	\node(c) at (3,0) {};
	\node(d) at (5,0) {};
	\fill[Burlywood1] (-1,1.5) rectangle (1,-.5);
	\node[smallfont] at (0,1.75) {substantia nigra};
	\fill[LemonChiffon3] (1.5,1.5) rectangle (5.5,-.5);
	\node[smallfont] at (3.5,1.75) {corpo striato};
	\draw[neurone,red] (a.center) -| (c) node[near start,above,smallfont,black] {dopamina};
	\draw[neurone,green] (b.center) -- (c) node[midway,below,smallfont,black] {Ach.};
	\draw[neurone] (c.center) -- (d) node[at end,below,smallfont,black] {GABA};
	\begin{scope}[yshift=-6em]
		\node at (-2,.5) {Parkinson};
		\node(a) at (0,1) {};
		\node(b) at (2,0) {};
		\node(c) at (3,0) {};
		\node(d) at (5,0) {};
		\fill[Burlywood1] (-1,1.5) rectangle (1,-.5);
		\fill[LemonChiffon3] (1.5,1.5) rectangle (5.5,-.5);
		\draw[neurone,red, dashed] (a.center) -| (c);
		\draw[neurone,green] (b.center) -- (c);
		\draw[neurone] (c.center) -- (d);
		\end{scope}
\end{tikzpicture}

\begin{tikzpicture}
	\Tree
	[.{patogenesi}
		[.farmacologica
			[.{farmaci \dwa dopamina}
				\node[farmaco]{\index{reserpina}reserpina};
			]
			[.{farmaci bloccano\\ recettori dopaminergici}
				\node[farmaco]{\index{clorpromazina}clorpromazina\\ (antipsicotico)};
			]
		]
		[.{eventi neurotossici}
			{erbicidi, tossine ambientali...}
		]
		[.{cause genetiche}
			[.{corpi di Lewy}
				[.{accumulo di $\alpha$--sinvdeina\\ che precipita}
					{accumulo di\\ dopamina alterata}
				]
			]
			{alterazioni di\\ una parkina}
		]
	]
\end{tikzpicture}

\begin{tikzpicture}
	\Tree
	[.{sintomi}
		{tremore a riposo (Ach.)}
		{rigidità muscolare}
		{ipocinesia con difficoltà a\\ iniziare movimento (\dwa DOPA)}
		{andatura strisciante}
		{demenza senile}
	]
\end{tikzpicture}

\begin{tikzpicture}
	\tikzset{frontier/.style={distance from root=200pt}}
	\Tree
	[.{farmaci}
		[.{inibitore selettivo\\ delle MAO}
			\node[farmaco]{\index{selegilina}selegilina};
		]
		[.{agonisti dei \ce{D_2}}
			\node[farmaco]{\index{bromocriptina}bromocriptina};
			\node[farmaco]{\index{pergolo}pergolo};
		]
		\node[farmaco]{\index{levodopa}levodopa (L-DOPA) + \index{carbidopa}carbidopa};
		\node[farmaco]{\index{amantadina}amantadina\\(antivirale)};
		[.{antagonisti dell'Ach.}
			\node[farmaco]{\index{atropina}atropina};
			\node[farmaco]{\index{scopolamina}scopolamina};
		]
	]
\end{tikzpicture}

La dopamina come farmaco per os o parenterale non attraversa la barriera emato--encefalica (BEE).

\begin{tikzpicture}
	\Tree
	[.{L-dopa}
		\edge node[smallfont,yshift=1.2em,xshift=3.5em]{nel sangue};
		[.decarbossilata
			[.{dopamina nel sangue}
				{non attraversa BEE}
			]
		]
		\edge node[smallfont,yshift=-1em,xshift=4em]{1\%};
		[.{attraversa BEE}
			[.decarbossilata
				{dopamina nel cervello}
			]
		]				
	]
\end{tikzpicture}

ma carbidopa inibisce la decarbossilasi e non attraversa la BEE quindi non inibisce la decarbossilasi nel cervello per cui

\begin{tikzpicture}
	\Tree
	[.{L-dopa}
		\edge node[smallfont,yshift=-.5em,xshift=4.3em]{10\%};
		[.{attraversa BEE}
			[.decarbossilata
				{dopamina nel cervello}
			]
		]				
	]
\end{tikzpicture}

\begin{tikzpicture}
	\Tree
	[.\node[farmaco]{\index{levodopa}levodopa (L-DOPA) +\\ \index{carbidopa}carbidopa};
		[.meccanismo
			{\upa dopamina nel cervello}
		]
		[.assorbimento
			os
		]
		[.metabolismo
			{inattivazione intestinale e renale}
		]
		[.{usi clinici}
			{migliore rigidità e ipocinesia\\ nel 20\% spariscono tutti i sintomi\\ effetto\dwa nel tempo}
		]
		[.{effetti collaterali}
			{coree}
			{fluttuazioni dello\\ stato patologico\\ effetto on-off}
			[.{nelle prime settimane}
				{nausee, vomito, tachicardia}
				{sindrome simil--schizzofreniche}
				{sospensione periodiche\\ (drug holidays)\\ aiuta a ridurre\\ gli effetti collaterali\\ ma non gli on-off}
			]
		]
	]
\end{tikzpicture}

\begin{tikzpicture}
	\Tree
	[.\ce{MAOI_b}
		[.meccanismo
			[.{inibizione MAO}
				{\upa dopamina}
				{\dwa on-off}
			]
		]
		[.assorbimento
			{os}
		]
		[.{usi clinici}
			{migliora effetto levodopa}
		]
		[.{effetti collaterali}
			{insonnia, discinesie}
			{NON usare con altri MAO!!!\\ rischio ipertensione elevata}
		]
	]
\end{tikzpicture}

\begin{tikzpicture}
	\Tree
	[.{Agonisti dei \ce{D_2}}
		[.meccanismo
			{\upa stimolazione \ce{D_2}}
		]
		[.assorbimento
			{os}
		]
		[.metabolismo
			{epatico}
		]
		[.{usi clinici}
			{effetto duraturo. Per pazienti che\\ non tollerano L--dopa o che\\ sono diventati refrattari}
		]
		[.{effetti collaterali}
			{nausea, anoressia\\ ipotensione, confusione, allucinazioni}
		]
	]
\end{tikzpicture}

\begin{tikzpicture}
	\Tree
	[.\node[farmaco]{\index{amantadina}amantadina};
		[.meccanismo
			{\upa sintesi e rilascio dopamina}
			{\dwa ricaptazione dopamina}
		]
		[.assorbimento
			{os}
		]
		[.{usi clinici}
			{migliora bradicinesia e\\ rigidità ma si diventa\\ velocemente refrattari}
		]
		[.{effetti collaterali}
			{simili a L--dopa + ritenzione\\ urinaria, edema periferico}
		]
	]
\end{tikzpicture}

\begin{tikzpicture}
	\Tree
	[.{antagonisti dell'Ach.}
		[.meccanismo
			{\dwa effetti colinergici	}
		]
		[.assorbimento
			{os, EV}
		]
		[.metabolismo
			{idrolisi}
		]
		[.{usi clinici}
			{migliorano la rigidità ed\\ il tremore ma non hanno effetti su\\ ipo e bradicinesia}
		]
		[.{effetti collaterali}
			{discinesie, sonnolenza\\ stato confusionale}
		]
	]
\end{tikzpicture}

\newpage
	\section{Farmaci del sistema cardiovascolare e renale}

\subsection{Farmaci anti--ipertensivi}

\begin{tikzpicture}
	\Tree
	[.Anti-ipertensivi diuretici simpaticolitici vasodilatatori ]
\end{tikzpicture}

\begin{tikzpicture}
	\Tree
	[.{Diuretici\\ (capitolo ah hoc)}
		[.{Diuretici dell'ansa} \node[farmaco]{furosemide}; ]
		[.{Inibitori del simporto\\ \ce{Na+}-\ce{Cl-}} \node[farmaco]{tiazidici}; ]
		[. {Risparmiatori di \ce{K+}} \node[farmaco]{spironolattone}; ]
	]
\end{tikzpicture}

\begin{tikzpicture}
	\tikzset{level 3/.style={level distance=120pt}}
	\Tree
	[.Simpaticolitici
		[.{SNC}
			[.\node[farmaco]{$\alpha$-metildopa}; {Inibitore dopa-carbossilasi\\ emergenza ipertensiva \\ Da sedazione, tossicit\`a epatica\\ coombs positivo} ]
			[.\node[farmaco]{clonidina}; {Agonista $\alpha_2$. $\downarrow$noradrenalina\\ Usato in gravidanza \\ Da sonnolenza, depressione\\ $\downarrow$libido, secchezza fauci } ]
		]		
		[.{$\beta$--bloccanti}
			[.\node[farmaco]{propranololo}; {Usato in ipertensione, scompenso cardiaco, \\ aritmie, glaucoma. Produce $\downarrow$GC e renina. \\ Da affaticamento,$\downarrow$umore, insomnia, $\uparrow$glicemia, \\ alterazione assetto lipidico (i non ASI). \\ Interruzione improvvisa $\uparrow$infarto.} ]
		]		
		[.{$\alpha$--agonisti} \node[farmaco]{doxazosina}; ]
		[.{Misti $\alpha$/$\beta$}
			[.\node[farmaco]{labetalolo}; {ipertensione da feocromocitoma.\\ Da prurito intenso, $\downarrow$eiaculazione} ]
		]
	 ]
\end{tikzpicture}

\begin{tikzpicture}
	\tikzset{level distance=80pt, level 4/.style={level distance=100pt}}
	\Tree
	[.{Vasodilatatori}
		[.{diretti}
			[.{prevalentemente\\ arteriosi}
				[.{Inibitori IP3} \node[farmaco]{idralazina\\ (non pi\`u usato)}; ]
				[.{\ce{Ca^{2+}} antagonisti}  \node[farmaco]{nifedipina\footnotemark\\ (anche verapamil\\ e diltiazem\\ ma su cuore)}; ]
			]
			[.{arterovenosi} 
				[.{rilascio \ce{NO}} \node[farmaco]{nitroprussiato\footnotemark\\ nitroglicerina}; ]
			]
		]
		[.{indiretti}
			[.{ACE inibitori}
				[.\node[farmaco]{captopril\\ enalapril\\ fosinopril}; {Dilata arteriole e grandi vene. \\$\downarrow$pre/post carico. \\ Non inficia riflesso barocettivo\\ ne secrezione di aldosterone. \\ $\uparrow$bradichinina da tosse secca\\ e edema angioneurotico.} ]
			]
			[.{Antagonisti AT--1}
				[.{sartani} {Uso in ipertensione, ACC, \\ nefropatia diabetica\\ NO in gravidanza} ]
			]
		]
	]
\end{tikzpicture}

\footnotetext{Vedere farmaci angina}

\footnotetext{Vedere farmaci angina}

\newpage

\subsection{Farmaci nell'angina e infarto cardiaco}

\begin{tikzpicture}
	\Tree
	[.{angina\\ infarto} vasodilatatori simpaticomimetici ]
\end{tikzpicture}

\begin{tikzpicture}
	\tikzset{level distance=90pt, level 3/.style={level distance=130pt}}
	\Tree
	[.{Vasodilatatori}
		[.Nitrati
			[.\node[farmaco]{Isosorbide mononitrato}; {Duranta d'azione pi\`u lunga} ]
			[.\node[farmaco]{Nitroglicerina}; {Rilascio \ce{NO}, $\uparrow$cGMP, relax muscolatura lis.\\ Via sublinguale, transdermica, rapido assorbimento\\ grazie alla solubilit\`a lipidica}  ]	
		]
		[.{\ce{Ca^{2+}}  antagonisti}
			[.\node[farmaco]{verapamil\\ (diidropiridine)}; {$\downarrow$conduzione NSA. $\downarrow$ RVP} ]
			[.\node[farmaco]{diltiazem}; {$\downarrow$conduzione NSA. $\downarrow$ RVP} ]
			[.\node[farmaco]{nifedipina}; {$\updownarrow$conduzione NSA.  Possibile tachicardia riflessa\\ minori effetti cardiaci} ]
		]
	]
\end{tikzpicture}

\begin{tikzpicture}
	\tikzset{level distance=90pt, level 3/.style={level distance=130pt}}
	\Tree
	[.{Simpaticolitici}
		[.{$\beta$--bloccanti}
			[.\node[farmaco]{propranololo\footnotemark}; {$\downarrow$GC, $\downarrow$PA, $\downarrow$consumo \ce{O2} micardico} ]
		]
	]
\end{tikzpicture}

\footnotetext{vedi farmaci anti-ipertensivi}

\begin{tikzpicture}
	\node[chartnode,anchor=west] at(0,0)(mlck){MLCK} node[chartnode,xshift=125pt] (mlckstar){MLCK${}^*$};
	\draw[drawarrow](mlck)[yshift=10pt]--node[smallfont,yshift=6pt,midway](Ca){\ce{Ca^{2+}}}
				node[chartnode,yshift=50pt,midway](CCa){Canali \ce{Ca^{2+}}}
				node[smallfont,yshift=-15pt,midway](camp){cAMP}
				node[chartnode,yshift=-60pt,midway](atp){ATP}(mlckstar);
	\draw[drawarrow] (mlckstar) [yshift=-10pt] -- (mlck);
	\draw[drawarrow] (CCa)-- node[midway](CCCa){} node[smallfont,xshift=5em](bloc){bloccanti canali} (Ca);
	\draw[drawarrow] (bloc)-- node[midway, below]{$\ominus$} (CCCa);
	\draw[drawarrow] (atp)-- node[midway](catp){} node[smallfont,xshift=5em](beta){$\beta$-bloccanti} (camp);
	\draw[drawarrow] (beta)-- node[midway, above]{$\oplus$} (catp);
	
	\node[chartnode,below right=1em and 2em of mlckstar](mlc){MLC};
	\node[chartnode,right=50pt of mlc](mlcstar){MLC${}^*$} node[right=3pt of mlcstar](+){+}
		node[chartnode, right=3pt of +](actina){actina};
	\node[chartnode,above=20pt of actina](contrazione){contrazione};
	\node[chartnode,below=20pt of mlc](relax){relax};
	\draw[drawarrow] (mlc) [yshift=10pt]-- node[midway](a){}(mlcstar);
	\draw[drawarrow] (mlcstar) [yshift=-10pt]-- 
		node[smallfont,midway,yshift=-6pt](cgmp){cGMP} 
		node[chartnode,yshift=-57pt,midway](gtp){GTP}
		(mlc);
	\draw[drawarrow] (actina) -- (contrazione);
	\draw[drawarrow] (mlc) -- (relax);
	\draw[drawarrow] (mlckstar) -| (a);
	\draw[drawarrow] (gtp)-- node[midway](cgtp){} 
		node[smallfont,xshift=10pt](gcstar){GC${}^*$}
		node[chartnode,xshift=100pt](gc){Guanil ciclasi} 
		(cgmp);
	\draw[drawarrow] (gc)-- node[midway](cgc){} node[midway, chartnode,yshift=-50pt](no){NO} (gcstar);
	\draw[drawarrow] (no)--node[midway, right]{$\oplus$}(cgc);
	
	\node[smallfont,text width=12em,anchor=west] at(0,-4) {* $\equiv$ elemento attivato\\
	MLCK $\equiv$ Miosina Catena Leggera chinasi\\ MLC $\equiv$ Miosina Catena Leggera};
\end{tikzpicture}

\begin{tikzpicture}
	\Tree
	[.Angina 
		[.{ischemia cardiaca transitoria\\ senza danno al miocardio}
			{stabile}
			{instabile}
			{di prinzmetal}
			{silente}
			{cronica}
		]
	]		
\end{tikzpicture}

\begin{tikzpicture}
	\Tree
	[.Terapia
		comportamentale
		chirurgica
		[.farmacologica
			[.{vasodilatatori\\ \ce{NO} e \ce{Ca^{2+}} antagonisti}
				{per aumentare il flusso}
			]
			[.antiaggreganti {per evitare i trombi}
			]
			[.fibrinolitici {per distruggere i\\ trombi preesistenti}
			]
			[.{$\beta$--bloccanti} {per ridurre il fabbisogno energetico}
			]
			[.oppioidi {per ridurre il dolore}
			]
		]
	]
\end{tikzpicture}

\begin{tikzpicture}
	\Tree
	[.terapia
		[.stabile
			{nitrati organici}
			{$\beta$--bloccanti}
			{statina}
			{aspirina}
		]
		[.instabile
			{nitrati}
			{aspirina}
			{eparina}
		]
		[.variante
			{nitrati organici}
			{\ce{Ca^{2+}} antagonisti}
		]
	]	
\end{tikzpicture}

\subsubsection{Nitrati organici}

\begin{tikzpicture}
	\Tree
	[.{nitrati organici}
		{nitroglicerina}
		{isosorbide mononitrato}
	]
\end{tikzpicture}

\begin{tikzpicture}
	\Tree
	[.effetti
		[.{diminuzione della richiesta\\ di O${}_2$}
			{$\downarrow$ritorno venoso}
			{$\downarrow$volume intracardiaco}
			{$\downarrow$pressione arteriosa}
		]
		[.{scompasa spasmo arterioso} {vasodilatazione arterie\\ coronariche}
		]
	]
\end{tikzpicture}

\begin{tikzpicture}
	\Tree
	[.{effetti collaterali}
		{tachicardia riflessa}
		{aumento riflesso contrattile}
		{riduzione del tempo di perfusione\\ diastolica indotta da tachicardia}
	]
\end{tikzpicture}

\subsubsection{Calcio antagonisti}

\begin{tikzpicture}
	\Tree
	[.tipo
		[.L
			[.{Corrente lunga}
				[.Verapamil
					{cuore}
					{muscolo scheletrico\\ e liscio}
					{neuroni}
					{ossa}
				]
			]
		]
		[.T
			[.{Corrente breve}
				[.Flunarizina
					{cuore}
					{neuroni}
				]
			]
		]
		[.N
			[.{Corrente breve} {neuroni} 
			]
		]
		[.P
			[.{Corrente lunga} {neuroni} 
			]
		]
		[.{Q/R}
			[.{Segnapassi} {neuroni} 
			]
		]
	]
\end{tikzpicture}

\begin{tikzpicture}
	\tikzset{level 2/.style={level distance=120pt}}
	\tikzset{frontier/.style={distance from root=350pt}}
	\Tree
	[.effetti
		[.{muscolo liscio}
			[.{arteriole + sensibili delle venule.\\ Quindi meno effetto di ipotensione ortostatica}
				\node[farmaco]{nifedipina};
			]
		]
		[.{miocardio}
			\node[farmaco]{varapamil\\ diltiazem};
		]
		[.{muscolo scheletrico}
			[.{nessun effetto\\ il \ce{Ca^{2+}} \`e intracellulare} ]
		]
	]
\end{tikzpicture}

\subsubsection{$\beta$--bloccanti}

\begin{tikzpicture}
	\Tree
	[.effetti
		{$\downarrow$frequenza cardiaca}
		{$\downarrow$pressione arteriosa}
		{$\downarrow$contrattilit\`a}
	]
\end{tikzpicture}

\begin{tikzpicture}
	\Tree
	[.{effetti indesiderati}
		{$\uparrow$volume telediastolico}
		{$\uparrow$tempo di eiezione}
		{insomnia}
		{sonni spiacevoli}
		{senso di affaticamento}
		{disfunzione erettile}
	]
\end{tikzpicture}

\begin{tikzpicture}
	\Tree
	[.controindicazioni
		asma
		{affezioni broncospastiche}
		{grave bradicardia}
		{blocco atriventricolare}
		{insufficienza ventricolare sinistra}
	]
\end{tikzpicture}

\subsection{Insufficienza cardiaca}

\begin{tikzpicture}
	\Tree
	[.{I.C.}
		[.{gittata insufficiente\\ a fornire \ce{O2}\\ a organismo}
			[.{i. sistolica}
				{$\downarrow$contrattilità}
				{$\downarrow$fraz. di eiezione}
			]
			[.{i. diastolica}
				{rigidità}
				{perdità di rilasciamento}
			]
		]
	]
\end{tikzpicture}

\begin{tikzpicture}
	\Tree
	[.{scopo del\\ trattamento}
		[.{fase stabile\\(cronica)}
			{$\downarrow$sintomi}
			{rallentare progressione}
		]
		[.{fase scompensata\\(acuta)}
			{ricondurre il paziente\\ alla fare stabile}
		]
	]
\end{tikzpicture}

\begin{tikzpicture}
	\tikzset{level 2/.style={level distance=150pt}}
	\Tree
	[.terapia
		[.{fase cronica}
			{antagonisti aldosterone}
			{ACE inibitori}
			{sartani}
			{$\beta$--bloccanti}
			{digitalici}
			\node(diuretici){diuretici};
			\node(vasodilatatori){vasodilatatori};
		]
		[.\node(acuta){fase acuta};
			{$\beta$--agonisti}
		]
	]
	\draw[drawarrow] (acuta) to[out=0,in=180] (diuretici);
	\draw[drawarrow] (acuta) to[out=0,in=180] (vasodilatatori);
\end{tikzpicture}

\begin{tikzpicture}
	\tikzset{level distance=80pt}
	\Tree
	[.\node(git){$\downarrow$gittata cardiaca};
		[.{$\downarrow$P.A.}
			[.{attivazione barocettori}
				[.\node(simpatico){$\uparrow$simpatico};
					[.{inotropo+} {rimodellamento}
					]
					[.{cronotopo+}
					]
				]
			]
		]
		[.{$\downarrow$flusso renale}
			[.{$\uparrow$renina}
				[.{$\uparrow$angII}
					\node(pre){$\uparrow$ pre--carico};
					[.\node(post){$\uparrow$post--carico};
						\node(fraz){$\downarrow$fraz. eiezione};
					]
				]
			]
		]
	]
	\draw[drawarrow] (simpatico) to[out=0,in=180] (post);
	\draw[drawarrow] (simpatico) to[out=0,in=180] (pre);
	\node[below=1em of fraz](a){};
	\draw[drawarrow] (fraz) --  (a.north) -| (git);
\end{tikzpicture}

\begin{tikzpicture}
	\tikzset{level distance=120pt}
	\Tree
	[.{rimodellamento causato da\\ ipertrofia per riattivazione\\ fattori di crescita}
		[.concentrico
			{da sovraccarico pressorio per \upa post--carico}
		]
		[.eccentrico
			{da sovraccarico volume per \upa pre--carico}
		]
		[.compensato {se raggio della cavità ventricolare,\\ massa ventricolo e volume cavità\\ sono rispettati}
		]
		[.scompensato 
			[.{se tali rapporti non sono rispettati}
				{evolve in\\ scompenso cardiaco}
			]
		]
	]
\end{tikzpicture}

\begin{tikzpicture}
	\tikzset{level distance=80pt}
	\Tree
	[.{funzionalità\\ cardiaca}
		[.pre--carico
			[.{pressione riempimento\\ ventricolo sx}
				[.{$\uparrow$I.C.}
					[.vasodilatatori	
						{nitrati organici}
					]						
				]
			]
		]
		[.post--carico
			[.{resistenze vasc.\\ sistemiche e\\ impedenza aortica}
				[.{$\uparrow$I.C.}
					[.{farmaci $\downarrow$tono\\ arteriolare}	
						{???}
					]						
				]
			]
		]
		[.contrattilità
			[.{$\downarrow$I.C.}
				[.{farmaci $\uparrow$inotropismo}	
					{???}
				]						
			]
		]
		[.frequenza 
			{$\uparrow$I.C. per compensazione}
		]
	]
\end{tikzpicture}

\begin{tikzpicture}
	\Tree
	[.farmaci
		[.\node[farmaco]{digitale/digossina\\ (inotropo+)};
			[.{inibizione \ce{Na+}/\ce{K+} ATPasi}
				[.{\upa\ce{Ca^2+} per\\ blocco NCX}
					{inotropo+}
				]
				[.{\dwa condutt. \ce{K+}}
					[.{\dwa durata PdA da cui\\ \upa PR, depressione\\ a cucchiaio ST}
					]
				]
			]
		]
		[.\node[farmaco]{dobutamina\\ (agonista $\beta_1$ selett.)};
			{\upa GC}
			{\dwa pre--carico}
		]
		[.\node[farmaco]{furosemide\\ (diuretico)};
			{\dwa P.A.}
			{\dwa pre--carico}
		]
		[.\node[farmaco]{captopril\\ elanapril (ACE inibitore)};
			{\dwa post--carico}
		]
		[.\node[farmaco]{losartan (antagonista AT-1)};
			{\dwa post--carico}
		]
		[.\node[farmaco]{carvedilolo\\ metoprololo ($\beta$--bloccanti)};
			{cronotopo-}
			{\dwa rimodellamento per\\ inibizione catecolamine}
		]
	]
\end{tikzpicture}

\begin{tikzpicture}
	\Tree
	[.{digitale +}
		[.\ce{K+}
			[.iper {\dwa effetti digitale}
			]
			[.ipo {\upa effetti digitale}
			]
		]
		[.\ce{Ca^2+}
			[.iper {\upa effetti digitale}
			]
			[.ipo {\dwa effetti digitale}
			]
		]
		[.\ce{Mn}
			[.iper {\dwa effetti digitale}
			]
			[.ipo {\upa effetti digitale}
			]
		]
	]
\end{tikzpicture}

\begin{tikzpicture}
	\tikzset{level 2/.style={level distance=150pt}}
	\Tree
	[.{effetti avversi}
		[.\node[farmaco]{digitale/digossina\\ (a dosi elevate)};
			{\upa aritmie, tachicardia, extrasistole\\ torsione di punta, FV}
		]
	]
\end{tikzpicture}

\newpage

	\section{Farmaci dell'emostasi}

\begin{tikzpicture}
	\tikzset{frontier/.style={distance from root=300pt}} 
	\Tree 
		[ .Emostasi 
			[ .anticoagulanti 
				[ .\node[farmaco]{\index{eparina}eparina};
					{frazionata o LMWH o\\ a basso peso molecolare}
					{non frazionata o HMWM\\(meno usata causa eff. coll.}
				]
				[ .{inibitori della trombina} 
					\node[farmaco]{\index{lepirudina}lepirudina\\ \index{argatroban}argatroban}; 
				]
				[ .orali \node[farmaco]{\index{warfarin}warfarin};  ]
			]
			[ .antiaggreganti 
				[ .FANS ]
				[ .{inibitori \\ della fosfodiesterasi} 
					\node[farmaco]{\index{dipiridamolo}dipiridamolo\\ \index{cilostazolo}cilostazolo };
				]
				[ .{antagonisti recettori ADP} 
					\node[farmaco]{\index{ticlopidina}ticlopidina\\ \index{clopidogrel}clopidogrel }; 
				]
				[ .{inibitori recettore\\ Gp IIb/IIIA} 
					\node[farmaco]{\index{abciximab}abciximab\\ \index{tirofiban}tirofiban\\ \index{eptifibatide}eptifibatide}; 				
				]
			]
			[ .trombolitici 
				[.{Attivatori tissutali\\ del plasminogeno (tPA)}
					\node[farmaco]{\index{urochinasi}urochinasi\\ \index{streptochinasi}streptochinasi }; 
				]
			]
		]
\end{tikzpicture}

\begin{tikzpicture}
		\Tree
		[.\node[farmaco]{\index{eparina}eparina};
			[.meccanismo
				[.{HMWH}
					{si lega a AT3 con \\\upa cinetica del\\ fattore Xa e trombina}
					{causa \upa aPTT per cui\\ continuo monitoraggio}
				]
				[.LMWH
					{si lega a AT3 con \\\upa cinetica del solo\\ fattore Xa quindi\\ no monitoraggio}
				]
			]
			[.assorbimento
				{sottocutaneo}
				{no intramuscolo ove\\ causa ematomi}
			]
			[.metabolismo
				{renale}
			]
			[.{usi clinici}
				{anticoagulante immediato\\(pochi sec. per agire)}
			]
			[.{effetti collaterali}
				{emorragia, trombocitenia,\\ osteoporosi in uso cronico}
				{antidoto: \index{protamina solfato}protamina solfato}
			]
		]
\end{tikzpicture}

\begin{tikzpicture}
	\Tree
	[.{inibitori diretti\\ della trombina}
		[.meccanismo
			{si legano al sito\\ attivo della trombina}
		]
		[.assorbimento
			{parenterale}
		]
		[.{usi clinici}
			{azione anticoagulante}
			{monitoraggio aPTT}
		]
		[.{effetti collaterali}
			{emorragia}
			{formazione anticorpi}
		]
	]
\end{tikzpicture}

\section{Farmaci antianemici}

\begin{tikzpicture}
	\Tree
	[.{anemia}
		[.microcitiche
			{carenza Fe}
			talassemia
			emoglobinopatie
		]
		[.normocitiche
			aplasie
			{anemie emolitiche}
			emorragie
		]
		[.macrocitiche
			{carenza \ce{B_12}}
			{carenza folati}
		]
	]
\end{tikzpicture}

\begin{tikzpicture}
	\Tree
	[.{farmaci}
		[.ferro
			\node[farmaco]{\index{solfato ferroso}solfato ferroso};
			\node[farmaco]{\index{ferro destrano}ferro destrano};
		]
		[.\ce{B_12}
			\node[farmaco]{\index{cianocobalamina}cianocobalamina};
		]
		{acido folico}
		[.{fattori di crescita\\ eritrocitari}
			\node[farmaco]{\index{eritropoietina}eritropoietina};
			\node[farmaco]{\index{darbepoetina $\alpha$}darbepoetina $\alpha$};
		]
		[.{fattori megacariocitari}
			\node[farmaco]{\index{oprelvekin}oprelvekin\\(IL-11 ricombinante)};
		]
	]
\end{tikzpicture}

Chelanti del ferro (\index{desferrioxamina}desferrioxamina) nel caso di intossicazione dal ferro.

\begin{tikzpicture}
	\Tree
	[.{\ce{B_12}}
		[.meccanismo
			{sopperisce alla carenza}
		]
		[.assorbimento
			[.{se causa da malassorbimento\\(a. perniciosa, carenza\\ fattore intrinseco)}
				{parenterale}
			]
			[.{altrimenti}
				os
			]
		]
		[.metabolismo
			{epatico}
		]
		[.{usi clinici}
			{carenza \ce{B_12} si da anche\\ acido folico ma solo\\ \ce{B_12} sopperisce\\ agli effetti neurologici }
		]
		[.{effetti collaterali}
			{diarrea, policitemia\\ insuff. cardiaca, shock anafilattico}
		]
	]
\end{tikzpicture}

\begin{tikzpicture}
	\Tree
	[.{folati}
		[.meccanismo
			{come \ce{B_12}}
		]
		[.assorbimento
			{os}
		]
		[.metabolismo
			{epatico}
		]
		[.{usi clinici}
			{\ce{B_12} ma nessun sintomo\\ neurologico}
		]
		[.{effetti collaterali}
			{\ce{B_12}}
		]
	]
\end{tikzpicture}

\begin{tikzpicture}
	\Tree
	[.{fattori stimolanti\\ eritropoiesi}
		[.meccanismo
			{stimola eritropoiesi che\\ rigenerano gli eritrocito}
		]
		[.assorbimento
			{EV}
			{sottocutaneo}
		]
		[.metabolismo
			{epatico}
		]
		[.{usi clinici}
			{anemie da insuff. renale che\\ causa \dwa EPO sierica}
			{chemioterapici}
			{trapianti midollo osseo}
		]
		[.{effetti collaterali}
			ipertensione
			iperviscosità
			prurito
			convulsioni
		]
	]
\end{tikzpicture}

\newpage

	\section{Farmaci del sistema respiratorio}

\subsection{Asma}

\begin{tikzpicture}
	\tikzset{level 2/.style={level distance=130pt}}
	\tikzset{level 3/.style={level distance=130pt}}
	\Tree
	[.Asma
		[.{malattia infiammatoria\\ delle vie aeree}
			{infiammazione}
			[.{ostruzione bronchiale}
				{solitamente reversibile}
				{in alcuni casi irreversibile}
			]
			{iperreattività agli allergeni}
		]
	]	
\end{tikzpicture}

\begin{tikzpicture}
	\Tree
	[.sintomi
		{sibili respiratori}
		{dispnea}
		tosse
		{costrizione torace}
	]
\end{tikzpicture}

\begin{tikzpicture}
	\tikzset{level distance=80pt}
	\Tree
	[.fisiopatogenesi
		[.{ingesso allergene}
			[.{APC presentano\\ antigeni ai \ce{T_H_2}}
				[.{\ce{T_H_2} stimolano B\\ a produrre IgE}
					[.{IgE si legano\\ agli allergeni}
						\node[dummyc]{};
					]
				]
			]
		]
	]
	\begin{scope}[yshift=-4em]
	\Tree
	[.\node[dummyc]{};
		[.{Mastociti legano IgE}
			{fase immediata}
			{fase tardiva}
		]
	]
	\end{scope}
\end{tikzpicture}

\begin{tikzpicture}
	\tikzset{level 1/.style={level distance=75pt}}
	\tikzset{level 2/.style={level distance=90pt}}
	\tikzset{level 3/.style={level distance=90pt}}
	\tikzset{level 4/.style={level distance=80pt}}
	\Tree
	[.{fase immediata}
		[.{mastociti\\ liberano}
			[.istamina
				[.{liberazione \ce{Ca^2+} nel REL}
					{broncospasmo}
				]
			]
			[.{leucotreni/citochina}
				[.IL5
					[.{attivazione eosinofili}
						{danno tissutale}
						{edema}
						{congestione}
					]
					[.{fase tardiva}
					]
				]
				[.IL4
					{stimolo a produrre IgE}
				]
			]
			[.{fattori di crescita}
			]
		]
	]
\end{tikzpicture}

\begin{tikzpicture}
	\tikzset{level distance=140pt}
	\Tree
	[.{fase tardiva}
		{inspessimento della parete\\ con restringimento del lume}
		{flogosi}
		rimodellamento
		{$\uparrow$produzione muco}
	]
\end{tikzpicture}

tutto ciò causa iperresponsività bronchiale futura.

\begin{tikzpicture}
	\tikzset{level distance=140pt}
	\Tree
	[.{farmaci}
		[.{broncodilatatori\\ (a breve durata d'azione)}
		]
		[.{glucocorticosteroidi\\ (in aerosol)}
		]
		[.{broncodilatatori\\(a lunga durata d'azione)}
		]
		[.{metilxantine o\\ antagonisti dei leucotreni}
		]
		[.{corticosteroidi orali}
		]
		[.{anticorpi monoclonali anti--IgE}
		]
	]
	\begin{scope}[xshift=18em]
		\draw[drawarrow] (0,2) -> (0,-2);
		\node[text width=8em] at (2,0){step operativi via via che la malattia diventa più grave};
	\end{scope}
\end{tikzpicture}

\begin{tikzpicture}
	\tikzset{level 1/.style={level distance=150pt}}
	\Tree
	[.{broncodilatatori}
		[.{$\beta_2$--agonisti\\(1a linea)}
			[.{a breve durata}
				\node[farmaco]{\index{salbutamolo}salbutamolo};
				\node[farmaco]{\index{terbutalina}terbutalina};
			]
			[.{a lunga durata}
				\node[farmaco](sal){\index{salmeterolo}salmeterolo};
				\node[farmaco](for){\index{formoterolo}formoterolo};
			]
		]
		[.{metilxantine\\(2a linea)}
			\node[farmaco]{\index{teofillina}teofillina};
			\node[farmaco]{\index{teobromina}teobromina	(cioccolato)};
			{caffeina};
		]
		[.{antagonisti muscarinici\\ (raramente usato. Più per BPCO)}
			\node[farmaco]{\index{ipratropio}ipratropio};
			\node[farmaco]{\index{tiotropio}tiotropio};
		]
	]
\end{tikzpicture}


\begin{tikzpicture}
	\Tree
	[.{$\beta_2$--agonisti}
		[.meccanismo
			{stimolazione $\beta_2$ muscolo bronchiale\\ con dilatazione}
			{inibizione del rilascio\\ dei mediatori dai mastociti}
			{$\downarrow$essudato}
		]
		[.somministrazione
			{inalazione per evitare\\ effetti sistemici}
		]
		[.metabolismo
			{epatico/renale}
		]
		[.{effetti collaterali}
			tachicardia
			{tremore muscolare}
		]
	]
\end{tikzpicture}

\begin{tikzpicture}
	\tikzset{level 2/.style={level distance=150pt}}
	\Tree
	[.{metilxantine}
		[.meccanismo
			{inibizione fosfodiesterasi IV (PDE)\\ che degrada cAMP quindi\\$\uparrow$cAMP$\Rightarrow$miorilassamento$\Rightarrow$broncodilatazione}
		]
		[.somministrazione
			os
		]
		[.metabolismo
			{epatico P450/renale}
		]
		[.{effetti collaterali}
			tachicardia
			{tremore muscolare}
			insonnia
			{$\uparrow$motilità intestinale}
		]
	]
\end{tikzpicture}

\begin{tikzpicture}
	\Tree
	[.{glucocorticoidi\\(GC)}
		\node[farmaco]{\index{idrocortisone}idrocortisone\\(EV)};
		\node[farmaco]{\index{beclometasone}beclometasone\\(aerosol)};
		\node[farmaco]{\index{budesonide}budesonide\\(aerosol)};
	]
\end{tikzpicture}

\begin{tikzpicture}
	\tikzset{level 2/.style={level distance=150pt}}
	\Tree
	[.{glucocorticoidi\\(GC)}
		[.meccanismo
			{$\downarrow$citochne}
			{inibizione COX2$\Rightarrow$inibizione\\ eosinofili e basofili}
			{inibizione leucotreni}
			{$\downarrow$edema}
			{$\uparrow$recettori $\beta_2$}
		]
		[.somministrazione
			{aerosol/EV}
		]
		[.{effetti collaterali}
			{sindrome di Cushing}
			disfonia
			{candidosi orofaringea\\(mughetto)}
		]
	]
\end{tikzpicture}

I GC vengono dati alle partorienti con figlio prematuro per velocizzare la funzione del surfactante polmonare che innalza la tensione degli alveoli e evita il collasso polmonare.

\begin{tikzpicture}
	\Tree
	[.{inibitori di\\ rilascio dei mediatori}
		[.\node[farmaco]{\index{cromolin}cromolin};
			[.meccanismo
				{inibizione del rilascio\\ di leucotreni e istamina\\ dai mastociti}
			]
			[.somministrazione
				{aerosol via\\ inalatori nasali}
			]
			[.{indicazioni terapeutiche}
				{allergie alimentari}
				{riniti allergiche}
				{congiuntiviti}
				{asma (raro utilizzo)}
			]
		]
	] 
\end{tikzpicture}

\begin{tikzpicture}
	\tikzset{level 1/.style={level distance=130pt}}
	\tikzset{level 2/.style={level distance=130pt}}
	\Tree
	[.{antagonisti dei leucotreni}
		[.meccanismo
			[.{bloccanti dei recettori\\ per i leucotreni}
				\node[farmaco]{\index{zafilukast}zafilukast};
				\node[farmaco]{\index{montelukast}montelukast};
			]
			[.{inibitori della lipossigenasi}
				\node[farmaco]{\index{zileuton}zileuton};
			]
		]
		[.somministrazione
			os
		]
		[.{indicazioni terapeutiche}
			{broncospasmo da antigene,\\ esercizio fisico o aspirina}
		]
		[.{effetti collaterali}
			{vasculite sistemica (leucotreni)}
			{$\uparrow$transaminasi (lipossigenasi)}
		]
	]
\end{tikzpicture}

\begin{tikzpicture}
	\Tree
	[.{anticorpo monoclonale\\ anti--IgE}
		[.\node[farmaco]{\index{omalizumab}omalizumab};
			[.somministrazione EV
			]
			[.difetti costoso
			]
		]
	]
\end{tikzpicture}

\newpage
	\chapter{Farmaci epatici}

\section{Citocromo P450}

\ce{RH + O_2 +2H^+ + 2e^- ->[monoossigenasi] ROH + H2O}

\begin{tikzpicture}
	\tikzset{level distance=130pt}
	\Tree
	[.\node(inibitori){inibitori};
		[.CYP1A2
			\node[farmaco]{\index{ciprofloxacina}ciprofloxacina};
			\node[farmaco]{\index{tacrina}tacrina};
		]
		[.CYP2C9
			\node[farmaco]{\index{fluconazolo}fluconazolo};
		]
		[.{CYP3A4\\(principale nella\\ detossificazione da farmaci)}
			\node[farmaco]{\index{eritromicina}eritromicina};
			\node[farmaco]{\index{claritromicina}claritromicina};
			\node[farmaco]{\index{ketocomazolo}ketocomazolo};
			{succo di pompelmo}
		]
	]
	\begin{scope}[yshift=-10em]
		\Tree
		[.\node(metabolismo){metabolismo};
			\node[farmaco]{\index{corticosteroidi}corticosteroidi};
			\node[farmaco]{\index{warfarin}warfarin};
		]
	\end{scope}
	\begin{scope}[yshift=-20em]
		\Tree
		[.\node(attivatori){attivatori};
			barbiturici
			\node[farmaco]{\index{carbamazepina}carbamazepina};
		]
	\end{scope}
	\node[chartnode](rallenta) at (0,-5em) {rallenta};
	\node[chartnode](accellera) at (0em,-15em) {accellera};
	\draw[drawarrow] (attivatori) -> (accellera);
	\draw[drawarrow] (accellera) -> (metabolismo);
	\draw[drawarrow] (inibitori) -> (rallenta);
	\draw[drawarrow] (rallenta) -> (metabolismo);
\end{tikzpicture}

\newpage
	\part{Esami}
\section{Temi svolti}

\subsection{$\beta$--bloccanti}

I farmaci $\beta$--bloccanti agiscono sul recettore adrenergico~$\beta$, recettore metabotropo a proteina~G di tipo~\ce{G_s} prevalentemente stimolatorio presente nel cuore, adipociti, apparato iuxaglomerulare~($\beta_1$), nel muscolo liscio~($\beta_2$) e nella vescica~($\beta_3$).

Il recettore attiva la cascata di segnalazione intracellulare tramite aumento di cAMP.

Gli effetti dei farmaci $\beta$--bloccanti agiscono nell'aparato cardiocircolatorio sia inibendo il sistema renica---angiotensina con riduzione del tono arteriolare con conseguente diminuzione della pressione e del post--carico, sia come effetto diretto inotropo e cronotropo negativo sul muscolo cardiaco.

Il \index{propanololo}propanololo agisce su titti i recettori di questa famiglia ed è usato principalmente nell'angina, nell'infarto e nell'insufficienza cardiaca per ridurre le richieste metaboliche del miocardio, nelle aritmie, come farmaco di classe~II per ridurre il potenziale d'aziene e aumentare il periodo refrattario~AV.

Il \index{metoprololo}metoprololo viene usato nelle emicranie e nel tremore muscolare.

Il \index{labetalolo}labetalolo (anche $\alpha$--bloccante) è usato per bloccare la cascata adrenergica introdotta dal feocromocitoma. Il \index{timololo} viene usato per la gestione del glaucoma.

I $\beta$--bloccanti lipofilici tipo il \index{propanololo}propanololo sono assunti per os e ben assorbiti, con intenso metabolismo epatico.

I $\beta$--bloccanti idrofilici tipo l'\index{atenololo}atenololo, non sono ben assorbiti per os. 

Eventuali effetti collaterali sono il blocco della conduzione~AV soprattutto in unione con i~\ce{Ca^2+}--antagonisti, reazioni broncocostrittive e aumento della glicemia.

\subsection{Farmaci anti--psicotici}

La schizzofrenia è una persistente alienazione del pensiero che da sia disturbi positivi, con caratteristiche psicologiche aggiunte quali delirio, allucinazioni e aggressività, sia sintomi negativi, con caratteristiche psicologiche perse quali isolamento sociale, apatia e mancanza di iniziativa.

Le teorie fisiopatologiche di questo disturbo sono tre. Una teoria dopaminergica che spiega i sintomi positivi che deriva la patologia da una iperstimolazione dei recettori~\ce{D_2}/\ce{D_4}; una teoria glutammatergica che deriva la patologi da bassi livelli di glutammato e da una conseguente iperstimolazione del recettore~NMDAM; una teoria serotoninergica derivata dall'osservazione che gli antagonisti serotoninergici sono antipsicotici e che l'LSD, un agonista~5-HT, fa venire i sintomi positivi.

I farmaci usati si dividono in tipici a atipici. I tipici quali i fenotiazidici come la \index{clorpromazina}clorpromazina e i butirrofenonici come l'\index{aloperidolo}aloperidolo (Serenase), agiscono bloccando i recettori dopaminergici diminuendo i sintomi positivi e, i più recenti, anche quelli negativi ma hanno effetti collaterali sul sistema extrapiramidale come distonie acute e tardive. 

Gli atipici non hanno effetti sulla via extrapiramidale in quando bloccano selettivamente la via mesolimbica (della gratificazione) ignorando la via nigrostriata e sono quindi usati principalmente se gli effetti collaterali dei tipici sono eccessivi. Tali farmaci bloccano anche i recettori $\alpha$--adrenergici e i~5-HT e sono la \index{clorapina}clorapina (una dibenzodiazepina) e l'\index{olanzapina}olanzapina.

Il difetto di tutti i farmaci anti--psicotici descritti è che impiegano settimane prima del loro effetto terapeutico e questo è un segno che vi deve essere un qualche altro effetto secondario ad agire come, ad esempio, l'aumento dei~\ce{D_2} a livello limbico.

\subsection{Farmaci antianemici}

L'anemia è una riduzione della massa eritrocitaria nel sangue misurabile come riduzione dei valori di Hb. Le cause principali sono carenza di ferro, talassemie e emoglobinopatie con caratteristiche microcitiche (MCV < 80 fl), emolitiche, aplastiche o emorragiche con caratteristiche normocitiche ( 100 < MCV < 80 fl) e deficit di \ce{B_12} o folati con caratteristiche macrocitiche (MCV > 100 fl).

La carenza di ferro si ha nel caso di amoraggie croniche, aumento del fabbisogno o diminuzione nell'assorbimento e si cura con sali ferrosi per os come il solfato ferroso o parenterale com il ferro destrano nel caso di intolleranza alla terapia per os. L'assorbimento segue le stesse vie del ferro alimentare. Le reazioni avverse vanno da disfunzioni intestinali quali diarrea, vomito e nausea fino alla gastrite necrotizzante nell'intossicazione acuta. Per via parenterale vi possono essere anche casi di reazione anafilattica.

Nel caso di sovradosaggi può essere utile l'impiego di chelanti del ferro come la \index{desferrioxamina}desferrioxamina.

Nel caso di anemia megaloblastica, dato che i sintomi del deficit di \ce{B_12} e di folati sono molto simili, va valutato dapprima l'eventuale deficit di \ce{B_12} eliminando dubbi sulla presenza di anemia perniciosa da deficit del fattore primario, malassorbimento primitivo, gravidanza e eliminazione dalla dieta della vitamina \ce{B_12}. Il deficit di tale vitamina, al contrario di quella dei folati, causa anche problemi neurologici che non si risolvono una volta risolto il deficit.

Le reazioni avverse alla vitamina~\ce{B_12} sono la trombosi, policitemia e insufficienza cardiaca fino allo shock anafilattico. Non si registrano grosse reazioni avverse all'uso dei folati.

Nel caso di anemie aplastiche, disordini midollari e insufficienza renale sono utili i fattori di crescita emopoietici come l'\index{eritropoietina}eritropoietina e la \index{darbepoetina $\alpha$}darbepoetina $\alpha$ con reazioni avverse un eventuale aumento dell'ematocrito e aumentata viscosità oltre che al prurito.

\subsection{Farmaci per il trattamento dell'obesità}

L'obesità è una malattia multifattoriale e poligenica in cui l'apporto calorico nel lungo periodo è superiore al consumo energetico causando un aumento del BMI, l'indice di massa corporea.

I principali fattori che entrano nella regolazione del cibo e del consumo energetico sono la leptina, la colecistochina (CCK), l'insulina, il sistema nervoso simpatico e fattori psico--socio--economici.

La leptina è sintetizzata dalle cellule adipose e il suo aumento dovrebbe portare ad un effetto anoressizzante ma nei pazienti obesi tale effetto è mancante per qualche forma di resistenza dovuta a degradazione, a difetto del trasportatore o inefficacia dei recettori.

La sintesi della leptina è regolata positivamente da glucocorticoidi, insulina. Una regolazione negativa è data da agonisti $\beta$--adrenergici.

La CCK agisce sul rilascio di bile, stimolando la secrezione di insulina e attiva la stimolazione vagale portando un effetto di sazietà.

L'insulina stimola la leptina ma nell'obeso, essendo insensibile ciò causa ipertensione.

Il sistema nervoso simpatico invece causa un effetto termogenico grazie alla fosforilazione ossidativa disaccoppiata nelle cellule brune e relativo aumento de consumi energetici.

L'obesità causa patologie secondarie quali il diabete mellito la cui terapia, l'insulina, causa un ulteriore aumento di assunzione di cibo, malattie cardiovascolari, tumori ormoni dipendenti, probelmi digestivi e respiratori e osteoartriti.

I farmaci usati sono la \index{sibutramina}sibutramina che inibisce la ricaptazione della serotonina e noradrenalina (IRSN) agendo negativamente sui siti regolanti l'appetito con aumento della sazietà, diminuzione del BMI, diminuzione di LDL e aumento di HDL. La sibutramina ha però controindicazioni quali un aumentato rischio cardiovascolare, costipazione e insonnia.

Un altro farmaco usato per trattare l'obesità è l'\index{orlistat}orlistat che blocca il sito delle lipasi gastriche e pancreatiche bloccando la degradazione dei grassi e l'assorbimento che vengono quindi eliminate dalle feci con steatorrea, crampi addominali e flatuenza.

Rimedi chirurgici sono il bypass e il bendaggio gastrico. Attività fisica e dieta controllata sono i primi approcci terapeutici imprescindibili anche se coadiuvati da eventuale terapia farmacologica.

\subsection{Le displidemie o iperlipidemie}

La displidemia indica un elevato livello di lipidi nel sangue. 

I lipidi presenti nel sangue arrivano da una via esogena e da una via endogena.

Dalla via esogena, dal cibo presente nel lume intestinale, i lipidi vengono internalizzati da un recettore chiamato NPC1L1 presente sull'orletto a spazola degli enterociti e qui esterificati e inglobati in chilomiconi che, attraverso il sangue, raggiungono muscolo, tessuto adiposo e fegato.

La via endogena prevede la sintesi nel fegato da parte, tra l'altro, di un enzima, l'HMG-CoA reduttasi che risulta catalizzare la tappa limitante della sintesi dei grassi.

I grassi vengono poi inglobate da lipoproteine a formare micelle classifficate sulla base della densità in HDL-C, LDL-C, VLDL.

Le displidemie possono essere primarie o secondarie per diabete mellito, alcolemia, insufficienza renale cronica o per effetto collaterale da farmaci.

I farmaci che agiscono sulla via endogena sono le statine (\index{simvastina}simvastina) che inibisce la HMG-CoA reduttasi, i fibrati (\index{benzofibrato}benzofibrato) che attivano un gruppo di geni che trascrivono per le lipasi, le apoA1 (quindi HDL) e apoA5 che a sua volta stimola la produzione di lipasi.

I farmaci che inibiscono l'assorbimento di colesterolo sono l'\index{ezetimide}ezetimide che blocca il recettore NPC1L1 e, un po' in disuso, le resine leganti gli acidi biliari che, voluminose e di cattivo gusto, sequestrano gli acidi biliari a livello del lume intestinale evitandone il riassorbimento ma causano diarrea per iperosmolarità del contenuto intestinale.

Da citare che le statine sono anche usate nella prevenzione dell'infarto del miocardio e nella prevenzione di placche aterosclerotiche in pazienti con LDL alto.

\subsection{Farmaci anti epilettici}

L'epilessia è una anomala scarica parossistica dei neuroni corticali dovuta, in alcuni casi, ad un deficit di inibizione~\ce{GABA_A}--mediato e ipereccitabilità glutammato--mediata.

Le crisi si classificano in parziali o convulsioni con locus encefalico specifico e generalizzate. Le generalizzate possono essere caratterizzate da assenza o piccolo male, tonico--cloniche o grande male e miocloniche. Le parziali possono evolvere in grande male. Esiste anche una categoria a se stante per gli spasmi infantili.

Le convulsioni vengono trattate con \index{carbamazepina}carbamazepina e \index{valproato}valproato oppure con \index{clonazepam}clonazepam e \index{fenitoina}fenitoina. 

Le crisi tonico--cloniche vengono trattate con \index{carbamazepina}carbamazepina o \index{valproato}valproato o \index{fenitoina}fenitoina.

Le assenze con \index{etosuccimide}etosuccimide o \index{valproato}valproato.

Le crisi miotoniche con \index{diazepam}diazepam.

Gli spasmi infantili con \index{corticotropina}corticotropina.

Nelle emergenze si usa \index{diazepam}diazepam o altra benzodiazepina insieme a la \index{fosfofenitoina}fosfofenitoina, una molecola simil--\index{fenitoina}fenitoina da usare in IM o IV.

La \index{carbamazepina}carbamazepina è un antidepressivo triciclico che ininbisce i canali~\ce{Na^+} con alta scarica di frequenza evitando cosi il blocco dei neuroni nello stato normale. Blocca anche i canali~\ce{Ca^2+}. Interagisce accellerando il metabolismo di \index{fenitoina}fenitoina e \index{warfarin}warfarin. Sconsigliato in pazienti sotto \index{MAOI}MAOI. 

La \index{fenitoina}fenitoina ha azione simile alla \index{carbamazepina}carbamazepina ma può causare le assenze per cui non va usata in questa patologia. Viene trasportata dall'albumina ma \index{valproato}valproato e \index{sulfonamidi}sulfonamidi hanno maggiore affinità per cui aumentano la concentrazione di farmaco libero. Sono stati rilevati possibili effetti teratogeni.

Il \index{valproato}valproato aumenta i livelli di GABA inibendo due enzimi inattivanti questo neurotrasmettitore. Ha bassa tossicità e non ha effetti sedativi con scari effetti collaterali. Spazza la \index{fenitoina}fenitoina dalle proteine plasmatiche e inibisce il metabolismo di \index{fentobarbital}fentobarbital, usato come cura delle epilessie in età pediatrica, \index{fenitoina}fenitoina, \index{carbamazepina}carbamazepina.

L'\index{etosuccimide}etosuccimide inibisce i canali del~\ce{Ca^2+} a bassa soglia responsabili delle correnti nelle assenze ma può esacerbare crisi tonico--cloniche. Può portare inoltre a nausea, vertigini e reazioni di ipersensibilità.

\newpage
	\chapter{Farmacocinetica}

\section{Emivita}

L'emivita di un farmaco è definita come il tempo necessario a ridurre
il farmaco a \unitfrac{1}{2} della quantità di farmaco presente nell'organismo
allo steady-state.

Presupponendo che la quantità di farmaco nell'organismo abbia un andamento esponenziale decrescente con il tempo, si pu definire questo matematicamente come:

$$
Q(t)=\alpha e^{-\beta t}
$$

Per trovare i due parametri $\alpha$ e $\beta$ consideriamo che a $t=0$
$Q(0)=Q_{\text{TOT}}=\alpha$ e quindi l'equazione sopra si pro scrivere come
$$
Q(t)=Q_{\text{TOT}}e^{-\beta t}
$$
e d'altra parte se consideriamo la velocità di eliminazione del farmaco
al tempo $t$ si ha che\vspace{.5em}

$-\dfrac{\text{d}\,Q(t)}{\text{d}\,t}=v_{\text{elim}}(t)=-Q_{\text{TOT}}(-\beta)e^{-\beta t}$ \vspace{.5em}

Ma d'altra parte, per definizione
$$
\text{CL} = \dfrac{v_\text{ELIM}^\text{STEADY STATE}}{c^\text{STEADY STATE}} =\dfrac{v_\text{ELIM}(0)}{c(0)}
$$
e, a $t=0\Rightarrow v_{\text{elim}}(0)=\text{CL}\cdot c(0)=-Q_{\text{TOT}}(-\beta)$ da cui
$\beta=\dfrac{\text{CL\ensuremath{\cdot}}c(0)}{Q_{\text{TOT}}}$ ma

$$V_\text{DIST} = \dfrac{Q_{\text{TOT}}}{c(0)}$$ 

e quindi \vspace{.5em}

$\beta=\dfrac{\text{CL\ensuremath{\cdot}}\cancel{c(0)}}{V_{\text{DIST}}\cdot\cancel{c(0)}}\Rightarrow\beta=\dfrac{\text{CL}}{V_{\text{DIST}}}$ e quindi

$$
Q(t)=Q_{\text{TOT}}e^{-\frac{\text{CL}}{V_{\text{DIST}}}t}
$$


a $t=t_{\unitfrac{1}{2}}\Rightarrow Q(t_{\unitfrac{1}{2}})=\dfrac{1}{2}\cancel{Q_{\text{TOT}}}=\cancel{Q_{\text{TOT}}}e^{-\frac{\text{CL}}{V_{\text{DIST}}}t_{\unitfrac{1}{2}}}$ 

e passando ai logaritmi naturali

$\ln\dfrac{1}{2}=-\dfrac{\text{CL}}{V_{\text{DIST}}}t_{\unitfrac{1}{2}}\Rightarrow t_{\unitfrac{1}{2}}=\ln\dfrac{1}{2}\cdot\left(-\dfrac{V_{\text{DIST}}}{\text{CL}}\right)=\dfrac{\ln2\cdot V_{\text{DIST}}}{\text{CL}}$

e quindi

\[
t_{\unitfrac{1}{2}}\simeq0.7\cdot\dfrac{V_{\text{DIST}}}{\text{CL}}
\]


	\printindex
\end{document}
