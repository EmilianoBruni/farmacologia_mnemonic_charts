\chapter{Farmaci dell'emostasi}

\section{Farmaci dell'emostasi}

\begin{tikzpicture}
	\tikzset{frontier/.style={distance from root=300pt}} 
	\Tree 
		[ .Emostasi 
			[ .anticoagulanti 
				[ .\node[farmaco]{\index{eparina}eparina};
					{frazionata o LMWH o\\ a basso peso molecolare}
					{non frazionata o HMWM\\(meno usata causa eff. coll.}
				]
				[ .{inibitori della trombina} 
					\node[farmaco]{\index{lepirudina}lepirudina\\ \index{argatroban}argatroban}; 
				]
				[ .orali \node[farmaco]{\index{warfarin}warfarin};  ]
			]
			[ .antiaggreganti 
				[ .FANS ]
				[ .{inibitori \\ della fosfodiesterasi} 
					\node[farmaco]{\index{dipiridamolo}dipiridamolo\\ \index{cilostazolo}cilostazolo };
				]
				[ .{antagonisti recettori ADP} 
					\node[farmaco]{\index{ticlopidina}ticlopidina\\ \index{clopidogrel}clopidogrel }; 
				]
				[ .{inibitori recettore\\ Gp IIb/IIIA} 
					\node[farmaco]{\index{abciximab}abciximab\\ \index{tirofiban}tirofiban\\ \index{eptifibatide}eptifibatide}; 				
				]
			]
			[ .trombolitici 
				[.{Attivatori tissutali\\ del plasminogeno (tPA)}
					\node[farmaco]{\index{urochinasi}urochinasi\\ \index{streptochinasi}streptochinasi }; 
				]
			]
		]
\end{tikzpicture}

\begin{tikzpicture}
		\Tree
		[.\node[farmaco]{\index{eparina}eparina};
			[.meccanismo
				[.{HMWH}
					{si lega a AT3 con \\\upa cinetica del\\ fattore Xa e trombina}
					{causa \upa aPTT per cui\\ continuo monitoraggio}
				]
				[.LMWH
					{si lega a AT3 con \\\upa cinetica del solo\\ fattore Xa quindi\\ no monitoraggio}
				]
			]
			[.assorbimento
				{sottocutaneo}
				{no intramuscolo ove\\ causa ematomi}
			]
			[.metabolismo
				{renale}
			]
			[.{usi clinici}
				{anticoagulante immediato\\(pochi sec. per agire)}
			]
			[.{effetti collaterali}
				{emorragia, trombocitenia,\\ osteoporosi in uso cronico}
				{antidoto: \index{protamina solfato}protamina solfato}
			]
		]
\end{tikzpicture}

\begin{tikzpicture}
	\Tree
	[.{inibitori diretti\\ della trombina}
		[.meccanismo
			{si legano al sito\\ attivo della trombina}
		]
		[.assorbimento
			{parenterale}
		]
		[.{usi clinici}
			{azione anticoagulante}
			{monitoraggio aPTT}
		]
		[.{effetti collaterali}
			{emorragia}
			{formazione anticorpi}
		]
	]
\end{tikzpicture}

\section{Farmaci antianemici}

\begin{tikzpicture}
	\Tree
	[.{anemia}
		[.microcitiche
			{carenza Fe}
			talassemia
			emoglobinopatie
		]
		[.normocitiche
			aplasie
			{anemie emolitiche}
			emorragie
		]
		[.macrocitiche
			{carenza \ce{B_12}}
			{carenza folati}
		]
	]
\end{tikzpicture}

\begin{tikzpicture}
	\Tree
	[.{farmaci}
		[.ferro
			\node[farmaco]{\index{solfato ferroso}solfato ferroso};
			\node[farmaco]{\index{ferro destrano}ferro destrano};
		]
		[.\ce{B_12}
			\node[farmaco]{\index{cianocobalamina}cianocobalamina};
		]
		{acido folico}
		[.{fattori di crescita\\ eritrocitari}
			\node[farmaco]{\index{eritropoietina}eritropoietina};
			\node[farmaco]{\index{darbepoetina $\alpha$}darbepoetina $\alpha$};
		]
		[.{fattori megacariocitari}
			\node[farmaco]{\index{oprelvekin}oprelvekin\\(IL-11 ricombinante)};
		]
	]
\end{tikzpicture}

Chelanti del ferro (\index{desferrioxamina}desferrioxamina) nel caso di intossicazione dal ferro.

\begin{tikzpicture}
	\Tree
	[.{\ce{B_12}}
		[.meccanismo
			{sopperisce alla carenza}
		]
		[.assorbimento
			[.{se causa da malassorbimento\\(a. perniciosa, carenza\\ fattore intrinseco)}
				{parenterale}
			]
			[.{altrimenti}
				os
			]
		]
		[.metabolismo
			{epatico}
		]
		[.{usi clinici}
			{carenza \ce{B_12} si da anche\\ acido folico ma solo\\ \ce{B_12} sopperisce\\ agli effetti neurologici }
		]
		[.{effetti collaterali}
			{diarrea, policitemia\\ insuff. cardiaca, shock anafilattico}
		]
	]
\end{tikzpicture}

\begin{tikzpicture}
	\Tree
	[.{folati}
		[.meccanismo
			{come \ce{B_12}}
		]
		[.assorbimento
			{os}
		]
		[.metabolismo
			{epatico}
		]
		[.{usi clinici}
			{\ce{B_12} ma nessun sintomo\\ neurologico}
		]
		[.{effetti collaterali}
			{\ce{B_12}}
		]
	]
\end{tikzpicture}

\begin{tikzpicture}
	\Tree
	[.{fattori stimolanti\\ eritropoiesi}
		[.meccanismo
			{stimola eritropoiesi che\\ rigenerano gli eritrocito}
		]
		[.assorbimento
			{EV}
			{sottocutaneo}
		]
		[.metabolismo
			{epatico}
		]
		[.{usi clinici}
			{anemie da insuff. renale che\\ causa \dwa EPO sierica}
			{chemioterapici}
			{trapianti midollo osseo}
		]
		[.{effetti collaterali}
			ipertensione
			iperviscosità
			prurito
			convulsioni
		]
	]
\end{tikzpicture}

\newpage
